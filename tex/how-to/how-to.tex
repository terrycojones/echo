\documentstyle [12pt] {article}
% \makeindex

\begin{document}

\newenvironment{shell}{\begin{verse}\begin{sf}}{\end{sf}\end{verse}}
\newcommand{\prompt}{{\bf \%}}

\title {{\bf An Introduction to SFI Echo.}}

\author
{
Terry Jones
\thanks
{
Santa Fe Institute, 1660 Old Pecos Trail, Suite A.
Santa Fe NM 87501.
email: terry@santafe.edu
}
\and
Stephanie Forrest
\thanks
{
Dept. of Computer Science,
University of New Mexico.
Albuquerque NM  87131.
email: forrest@cs.unm.edu
}
}


\pagenumbering{empty}
\maketitle
\newpage

% \pagenumbering{empty}

\pagestyle{plain}
\pagenumbering{roman}
\tableofcontents
\newpage
\pagestyle{headings}
\pagenumbering{arabic}

\section{Introduction}

This report is concerned with an implementation of a family of models
of complex adaptive systems called Echo models. In what follows, you
will find:

\begin{itemize}

\item
An introduction to Echo.

\item
Information on how to obtain, install and run the Echo system.

\item
A description of Echo's graphical interface.

\item
Information on running Echo.

\end{itemize}

\subsection{Echo}

Echo is a model of complex adaptive systems formulated by John Holland
\cite{holland-92a,holland-92b,holland-93}. 
It abstracts away virtually all of the physical details of real
systems and concentrates on a small set of primitive agent--agent and
agent--environment interactions.  The extent to which Echo captures
the essence of real systems is still largely undetermined.  The goal
of Echo is to study how simple interactions among simple agents lead
to emergent high--level phenomena such as the flow of resources in a
system or cooperation and competition in networks of agents (e.g.,
communities, trading networks, or arms races).

An Echo world consists of a lattice of sites. Each is populated by
some number of agents, and there is a measure of locality within each
site.  Sites produce different types of renewable resources; each type
of resource is encoded by a letter (e.g., ``a,'' ``b,'' ``c,'' ``d'').
Different types of agents use different types of resources and can
store these resources internally. Sites charge agents a maintenance
fee or tax. This tax can also be thought of as metabolic cost.

Agents fight, trade and reproduce. Fighting and trading
result in the exchange of resources between agents. There is sexual
and non--sexual reproduction, sexual reproduction
results in offspring whose genomes are a combination of those
of the parents. Each agent's genome encodes various genes
which determine how it will interact with other agents (e.g., which
resource it is willing to trade, what sort of other agents it will
fight or trade with, etc.).  Some of these genes determine phenotypic
traits, or ``tags'' that are visible to other agents.  This
allows the possibility of the evolution of social rules and
potentially of mimicry, a phenomenon frequently observed in
natural ecosystems. The interaction rules rely only on string
matching.

Echo has no explicit fitness function guiding selection and
reproduction.  An agent self--reproduces when it accumulates a
sufficient quantity of each resource to make an exact copy of its
genome.  This cloning is subject to a low rate of mutation.

In preliminary simulations, the Echo system has demonstrated
surprisingly complex behavior (including something resembling a
biological ``arms race'' in which two competing agent types develop
progressively more complex offensive and defensive combat strategies),
ecological dependencies among different species, and sensitivity (in
terms of the number of different phenotypes) to differing levels of
renewable resources.

Ideally, Echo will allow the modeling of a diverse range of complex
adaptive systems without the need for a specialized model for each to
be developed. Typically, the people who know the most about any
particular real--world complex adaptive system are not the people who
can also develop sophisticated models that can be used as tools to
increase understanding. Echo aims to provide a useful modeling tool or
a starting point for the development of a model.

As a cautionary note, one must be a little careful when using the term
``Echo.''  Properly, Echo refers to a large family of models. As
described here, Echo will refer to the implementation developed at the
Santa Fe Institute.

Several versions of the system have been developed by Holland, and
there are significant differences between these. Echo has been
described in
\cite{holland-92a,holland-92b,holland-93}. These descriptions
represent snapshots of ongoing thought about Echo models. The version
implemented here is closest to that described in \cite{holland-92a}.
For further details, refer to the above sources.

\section{Getting Started}

\subsection{A Note On Fonts}
\index{Fonts}

The following conventions are used consistently throughout this
report:

\begin{itemize}

\item
\index{UNIX Commands}
File names and UNIX commands appear in {\bf bold face}. These will
normally be references to files and commands that you might need or
use, not things you will be expected to enter immediately.

\item
\index{UNIX Environment Variables}
Environment variables are shown in {\sc small caps} and are always
completely {\sc uppercase}.

\item
Commands you are expected to type or put in a file are shown in {\sf
sans serif}. This is true even if what you are asked to type is a file
name or an environment variable.

\item
Anything printed by the system appears in a {\sl slanted} font.

\item
Things you will see in the Echo interface are typeset in a {\sl
slanted} font, as for other system output, but, additionally, are
enclosed in ``{\sl double quotes}''. The text so enclosed is exactly
what you can expect to see in Echo's interface. For example, this font
will be used when discussing the various options you'll see in Echo's
popup menus.\index{Pop--up Menus}

\item
\index{UNIX Shell Prompt}
The UNIX shell prompt is a bold percent sign (\prompt) at the start
of a line.

\end{itemize}


\subsection{System Requirements}
\index{System Requirements}

Echo was developed on a SUN SPARC architecture running SUNOS 4.1.3. In
theory it will run on other BSD--based versions of UNIX, but may
require slight modification in some cases. The X Windows \index{X
Windows} system from MIT is required. Development was under X11R5, but
Echo should run without modification under X11R4. If you are using a
SPARC machine, no special widget sets \index{Widgets} are required,
the distribution file contains these. If not, you will need to {\bf
ftp} widgets from several locations -- this is discussed in appendix
\ref{echo-widgets} \index{Echo Widgets}.

\subsection{Porting Echo}
\index{Porting}

Unless a joint research agreement with the Santa Fe Institute exists,
there is currently no guaranteed way to get help porting Echo to a new
architecture. A possible source of help is through electronic mail to
{\bf echo@santafe.edu}. Assistance will probably be contingent on the
project being supported at SFI.

\subsection{Getting Echo Through Anonymous FTP}
\index{Using Ftp}

Echo is available via anonymous ftp from {\bf ftp.santafe.edu} in the
file

\begin{verse}
{\bf /pub/Users/terry/echo/Echo-1.0.tar.Z}
\end{verse}

The entire distribution occupies just over two megabytes of disk
space when uncompressed. You should transfer Echo, uncompress
\index{Uncompress} it and
extract the files using tar\index{Tar}.  If you are not familiar with
the workings of ftp, uncompress or tar, consult appendix
\ref{using-ftp}.


\subsection{Compiling Echo}

Once you have received and unpacked the distribution, you will need to
compile or ``make''\index{Making Echo} it. The {\bf make} command will
take care of the compilation and linking, but first you will need to
make some changes to the file {\bf Makefile}. These should be very
minor, the file contains instructions to help you. You should only
have to change two lines, to indicate where the X libraries and
include files are on your system. If you don't know, you can try to
find them or ask your system manager.

Once you have done this, you can simply type

\begin{shell}
\prompt\ make all
\end{shell}

and make will do the rest. You should see about twenty files get
compiled.  When the make is completed, you should have a file called
{\bf Echo} which is the executable program. This process also creates
you a shell script, {\bf run--echo} to make it simple to run the
executable in the recommended way.

\subsection{Preparing to Run Echo}

\label{simple-setup}

Two additional things need to be done before running Echo.

\begin{itemize}

\item
X resources for Echo need to be installed.

\item
Shell environment variables need to be set.

\end{itemize}

The distribution contains programs that will do this for you if do not
want to do it yourself, or do not know how. If you'd like it taken
care of for you, use

\begin{shell}
\prompt\ make simple--setup
\end{shell}

This will add X resource defaults to your X startup file and set
environment variables in your shell's startup file. If you have used
this option, you should now log out and then log back in, as these
changes will not affect your current login session\footnote{They will
be automatically set for you in all future login sessions.}.

If you'd prefer to do this kind of installation yourself, you should
see the details about what needs to be done in appendix \ref{setting-up}.


\subsection{Running Echo}
\index{Running Echo}
\label{running-echo}

Once you have completed the above, you should be able to run Echo.
Echo writes its textual output to the {\bf xterm} from which it was
invoked.  The X resources \index{X Resources} you just installed
specify the colors, sizes and locations of all the pieces of the Echo
interface.  These pieces are designed to fit around a central {\bf
xterm} window.

There are two ways you can arrange things so that an {\bf xterm} is
created in the right place (and the right color):

\begin{itemize}

\item
The simplest way is to use the shell script {\bf run--echo} that was
made for you when you did the original {\bf make}.  This will do all
the work of setting up an {\bf xterm}, invoking Echo, and making sure
the interface is configured correctly.

\item
Slightly more work, but simpler in the long run is to put Echo into a
popup menu.\index{Pop--up Menus} This is achieved by adding a line to
your window manager's initialization \index{Window Managers}
file. This file's name depends on your window manager. A likely
location is {\bf $\sim$/.twmrc}, or some other file in your home
directory whose name starts with a period ({\bf .}) and ends in the
string ``{\bf wmrc}''.  If you don't know where your window manager
initialization file is or can't find it, ask your system manager. The
file {\bf .window\_manager} in the Echo distribution contains a line
that can probably be usefully inserted in this file\footnote{{\bf
Warning}: it is not as simple as just adding the line to the end of
this file. If you are unsure about what you're doing, get help.}.

\end{itemize}

\subsection{A Sample Echo Run}
\index{Demo Run}
\index{Sample Run}
\index{Running Echo}

This section walks you through a quick Echo demo, without too much
explanation.

Once you have installed the X resources and the shell environment
variables, you are in a position to run Echo. Try it now -- either
from your window manager or with

\begin{shell}
\prompt\ ./run--echo
\end{shell}

You should see a number of windows appear on your screen. These are
explained in the next section. For now notice the main menu bar at the
top of the screen and the {\bf xterm} in the middle.

\begin{itemize}

\item
The first step in running a world is to choose the world you intend to
run. To do this, click on the red menu button at the top of the screen
that says ``{\sl Running}''. Hold the mouse button down and drag the
pointer down to ``{\sl Choose World}'' and let go.\index{Choosing A
World} \index{Pop--up Menus}

\item
A file selector will pop up. This allows you to walk through the
directory structure to find a file containing the specification of a
world you want to run. You should see ``insects'' in the scrollable
region on the right. Click on this word and the line it is on should
change to white on black. Now click on ``{\sl OK}'' to select the
world.

\item
Look at the ``{\sl Species}'' graph.\index{Species Levels Graph} It
now has a legend, showing ``{\sl ant}'', ``{\sl fly}'' and ``{\sl
cat}'' (``cat'' is short for caterpillar). The legend has appeared as
the world you selected contains a site that contains all of these
agent types.

\item
Now select ``{\sl Run 10 Generations}'' from the ``{\sl Running}''
menu.\index{Running Menu} You should see the two graphs update and the
{\bf xterm} will show the text: ``{\sl Run through generation 10.}''

\item
Let's edit the world and remove a few agents.\index{Editing The Stack}
\index{Stack Editing} Click on the ``{\sl Edit}'' menu button and drag
the pointer down to ``{\sl Stack}''. A window will appear showing you
the population of agents present at the site. This is called the
``stack'' at that site. It's a one--dimensional array of agents. Use
the left mouse button to move the small arrowed cursor to the start of
some line. Now hit Control--k twice on your keyboard. You should see
the line in the stack disappear -- and with it some unfortunate
agent. Control--k is an Emacs keybinding that kills the text until the
end of the line. See appendix
\ref{emacs-keybindings} for more information on the keybindings if
these are not familiar to you.

Now that you've removed the agent (or agents if you were feeling
enthusiastic), hit the ``{\sl Amen}'' button at the top of the stack
editor to make it so.

\item
Before restarting the world, choose ``{\sl Set Verbose Level...}''
from the ``{\sl Control}'' menu at the top of the
screen.\index{Verbose Level} Here you can type letters to see certain
types of output in the {\bf xterm} window.  In the small gray window
at the bottom of the panel that popped up, type the following
incantation: ``gsukbx''. You can see what output each of these letters
will produce by reading the text above the small text window. To type
in the text window you'll need to move the mouse into it. Now click on
the ``{\sl OK}'' button to dismiss the window.

\item
Choose ``{\sl Run 1 Generation}'' from the ``{\sl Running}''
menu.\index{Running Menu} You should see various output appear in the
{\bf xterm}. This will likely include information on the agents that
were taxed during the generation and those that went bankrupt. There
is a small chance you'll see a mutation or an agent killed in
combat. The end of the output will show a population summary.

\item
Now we'll run to generation 100. Choose ``{\sl Run Until...}'' from
the ``{\sl Running}'' menu.\index{Running Menu} In the small window
that pops up, enter 100 and then click on ``{\sl OK}''. You should see
the graphs rapidly update to generation 100. You'll also see a lot of
text flashing by in the {\bf xterm} window. Often it is good to turn
off all such output when you are about to run for a large number of
generations.

\item
Select ``{\sl Variant Levels}''\index{Variant Levels Graph} from the
``{\sl Graphs}'' menu. The graph will appear where the ``{\sl
Worldwide Population Level}'' graph was (actually it's just on top of
it). The top line (in red) shows the number of different genomes that
have ever existed in the world. The bottom (blue) line shows the
number that are currently alive. You can compare this number with the
worldwide population level, and it will probably be quite a lot
smaller. Obviously, some genomes have multiple copies -- presumably
because they are doing something right.

\item
Finally, choose ``{\sl Cluster Living Population}''\index{Cluster
Analysis} from the ``{\sl Examine}'' menu.\index{Examine Menu} You
will see a tree displayed in the {\bf xterm} window that indicates how
closely related the agents in the world are.  You will probably need
to resize (or scroll) your {\bf xterm} window to see the whole
tree. This tree will be explained in more detail later.

\item
To exit Echo, choose ``{\sl Exit}''\index{Exiting Echo} from the
``{\sl Control}'' menu.  Of course you can continue to play if you
like.

\end{itemize}

The next section gives more details about the kinds of things you just
saw and did.


\section{The Echo Interface}
\index{Echo Interface}

After starting Echo, you should see seven windows on your
screen. Starting in the center at the top and moving
counter--clockwise, these are the Control window, the World Editor,
the Site Editor, the Agent Editor, a graph showing species levels, a
graph showing the worldwide population level, and an {\bf xterm}
window.

\subsection{The Overall Look And Feel}
\index{Echo Look and Feel}

There are some high level features of the interface that have been
designed to make the look and feel consistent. These are:

\begin{itemize}

\item
\index{Mouse Buttons}
In general, the interface makes no distinction between the buttons on
your mouse (assuming you have more than a single button). You can use
any button you like to push command buttons, or use popup menus.
\index{Pop--up Menus} The only exception to this rule is described
in the next point.

\item
\index{Text Windows}
Some of the areas of the screen are a steely grey color. These are
text windows. This color is used consistently to indicate an area of
the screen where you may type characters. Text in these windows will
always be black. All you need do is move the cursor into the grey area
and start to type. Each of these is in fact a small editor with
Emacs--like keybindings\footnote{If you are not familiar with Emacs,
don't worry, you can move around with mouse button 1 and use the
delete key to get rid of things you don't want. Appendix
\ref{emacs-keybindings} gives a brief introduction to editing with
Emacs.}.

In a text window, the mouse may be used to copy and paste regions in
the same fashion as in normal {\bf xterm} windows. The first button, when
held down, can be used to drag out, and thus copy, a region (which
will be highlighted). The second button pastes the copied text into
the buffer at the cursor's location. The third button can be used to
extend the copied region. This is something you should experiment with
if you are not already familiar with this kind of mouse behavior.

In addition, rapid double and triple clicks with the left mouse button
can be used to select the word the cursor is on or the line the cursor
is on. A single click with the first button just moves the text cursor
to the location of the mouse.

A search window can be popped up by typing Control--S, and the
contents of a file can be read into the buffer using Meta--I, which
pops up a window to read the file name. See the Emacs keybindings in
appendix \ref{emacs-keybindings} for more information on operations in
text windows.

\item
\index{Read Only Output}
Blue areas where white text appears, such as the main output window,
are always read--only.

\item
\index{Command Buttons}
\index{Menu Bars}
Command and menu buttons always have a red background and white text.
These always indicate things you can click the mouse on to have some
action performed. Command buttons carry out an action immediately,
while menu buttons pop up a menu of options.  As mentioned above, no
distinction is made here between the buttons on your mouse -- use
whichever you like. \index{Mouse Buttons}

\item
\index{File Selectors}
There are a number of occasions when Echo will need to get a file name
from the user. To do this, it will display a file selector window.
This window has various components. You may enter the file's name in
the text window at the top of the file selector, and then click on
``{\sl OK}''. Often simpler, is to use the mouse to select the file
you want. The scrollable window labeled ``{\sl Directory Contents}''
contains the list of files in the currently scanned directory. Echo
tries to make sure that the default directory is a useful one (using
the {\sc echo\_location} environment variable).

You can select a file in the scrollable list by simply clicking on it
with the mouse. The second mouse button can be used to scroll the
list. If you click on the entry labeled ``{\sl ../}'' the file
selector will read and display the contents of the parent directory.
The ``{\sl Up}'' button can also be used to move up a level.  If you
click on an entry that is a directory, the file selector will read and
display that directory's contents. Directories can be identified by
the trailing ``{\sl /}'' after their name.

The ``{\sl OK}'' button selects the file and causes the file selector
to disappear. The ``{\sl Cancel}'' button dismisses the file selector
and cancels whatever function was called that made it appear
originally. 

\end{itemize}


\subsection{The Control Window}
\index{Control Window}

The control window consists of five pull down menus. These are labeled
``{\sl Control},'' ``{\sl Edit},'' ``{\sl Running},'' ``{\sl
Graphs},'' and ``{\sl Examine}.''

\begin{itemize}

\item
The ``{\sl Control}'' menu contains three options. The first, ``{\sl
Set Verbose Level...}'' \index{Verbose Level} pops up a window that
contains lines that briefly describe types of output that can be
displayed in the main {\bf xterm} window. At the start of each of
these lines is a key letter. In the small grey window at the bottom
you may enter the letters corresponding to the types of output you
want to see in the {\bf xterm}. This output will appear following each
generation, so you must run at least one generation to see anything.

The ``{\sl Show Seed}'' \index{Random Seeds} option of the ``{\sl
Control}'' menu prints the current random seed value into the {\bf
xterm}. This allows you access at any time to the seed that was used
to set up the current run.

The ``{\sl Exit}''\index{Exiting Echo} option does exactly that, it
exits Echo completely. It does not currently ask for confirmation, so
be careful!

\item
The ``{\sl Edit}'' \index{Editing Menu} menu has seven
options. Most of these are very similar.  ``{\sl Worlds''}
\index{Editing Worlds} will pop up
a file selector to let you choose a world to edit. The file selector
can be used to wander through a directory tree and select a file
containing the specifications of a world. Use the mouse to click on
the name of the file you want (or enter it in the top grey window) and
then click the ``{\sl OK''} button to select it. The world you choose
will appear in the World Editor window in the top left corner of the
screen (more on that soon).

Choosing ``{\sl New World}'' \index{Creating New Worlds} will clear the World
Editor window and let you enter the details of a fresh world.

The options ``{\sl Sites},'' \index{Editing Sites} ``{\sl New Site},''
\index{Creating New Sites} ``{\sl Agents},'' \index{Editing Agents}
and ``{\sl New Agent}'' \index{Creating New Agents} all behave in a
similar fashion. They can be used to prepare for editing the
characteristics of sites and agents in the Site Editor (middle left)
and Agent Editor (bottom left) windows.

The final option in the Editing menu is ``{\sl Stack.''}
\index{Editing The Stack} \index{Stack Editing} This allows
you to edit a site in a running world. Since we have not yet begun to
see what happens when a world is actually running, this operation will
be described later.

\item
The ``{\sl Running}'' \index{Running Menu} menu has nine options. The
first, ``{\sl Choose World}'' \index{Choosing A World} is used to tell
Echo which world you intend to run. This option must be used before an
Echo run can begin. It also pops up a file selector so that you can
choose a world.

The next four options, \index{Running Echo} ``{\sl Run
Indefinitely,}'' ``{\sl Run 1 Generation,}'' ``{\sl Run 10
Generations,}'' and ``{\sl Run Until...}''  are all to do with running
a world for a certain time. They should all be self explanatory. The
last will pop up a window so you can enter a stopping generation
number.

The next two options, ``{\sl Pause''} \index{Pausing A Run} and ``{\sl
Continue''} \index{Continuing A Paused Run} can be used to temporarily
halt and restart a run. If you have used one of the above running
options to run until generation 500 and suddenly decide you need to
turn off the output in the main text window, you can use ``{\sl
Pause}'', turn off the output and then ``{\sl Continue}'' to allow the
run to proceed to generation 500. These two options can be used in
this manner any time Echo is running an experiment.

The ``{\sl Replay''} \index{Replaying A Run} option resets the world
and re--initializes the random number generator so that the run can be
re--done exactly. This is very useful when you would like to try some
experiment and need to stop the world at an earlier point.

The ``{\sl Seed''} \index{Random Seeds} option can be used to enter
the random seed for a run.  The seed should be set {\em before} you
choose the world to run. It is important that you perform these two in
this order. If you do not specify a random seed for the run, one will
be chosen for you. The seed chosen for you is guaranteed to be unique,
and the random number generator has been highly scrutinized for
randomness. The file {\bf random.c} contains a blow--by--blow
description of the search for an acceptable generator.

\item
The ``{\sl Graphs}'' \label{graphs-menu} \index{Graphs Menu} menu can
be used to pop up any of five graphs. Initially there are two places
where graphs appear, and two of the five are shown by default. The
species level graph always appears on the left while the other four
are on its right. Of course, since these are normal X windows, you can
move them around and resize them as you wish.  The five graphs are:

\begin{itemize}
\item
The ``{\sl Species Levels}'' \index{Species Levels Graph} graph shows
a legend (after you choose a world) and plots the population level of
all the descendants of the original members of the ``species''
\index{Species}. It is not really correct to call these groups species
(lineage is more accurate), but I will not go into that here.

\item
The ``{\sl Worldwide Population Level}'' \index{Population Level
Graph} graph shows the total number of agents alive in the world.

\item
The ``{\sl World Resource Levels}'' \index{Resource Levels Graph}
graph shows how many of each resource exist (this is often very dull
viewing).

\item
The ``{\sl Schema Level}'' \index{Schema Levels Graph} graph allows you
to track the level of a schema in the population.

\item
The ``{\sl Variant Levels}'' \index{Variant Levels Graph} graph shows
the number of genomes that have ever existed and the number of genomes
that currently exist.

\end{itemize}

\item
The ``{\sl Examine}'' \index{Examine Menu} menu allows you to look at
some properties of the populations. The first option ``{\sl Choose
Schema to Graph...}''  \index{Graphing A Schema} \index{Choosing A
Schema To Graph} allows you to enter a regular expression
\index{Regular Expressions} 
corresponding to a schema you are interested in. The regular
expression is in UNIX--style and is a pattern of resources that might
be found on an agent's genome.  The simple details of these
expressions are explained in section
\ref{schema-tracking}.  More information on regular expressions can be
found in the UNIX manual page for {\bf egrep}.

The next two options, ``{\sl Cluster living population}'' and ``{\sl
Cluster all individuals}'' \index{Cluster Analysis} both perform
cluster analysis. You may choose to have either the current population
clustered or every genome that ever existed clustered. The output will
appear in the {\bf xterm} window, which should have a scrollbar in
case the output is too long. Clustering is explained in section
\ref{cluster-analysis}.

\end{itemize}

\subsection{The World Editor}
\index{World Editor}
\label{world-editor}

The ``{\sl World Editor}'' is used to make changes to the properties
of a world.  These changes do not affect the currently running
world. You can choose the world you wish to alter (or a new world)
from the ``{\sl Edit}'' menu in the ``{\sl Control}'' window. Worlds
have thirteen properties. Eleven of these are edited by entering their
values into the grey horizontal text windows to the right of their
brief descriptions:

\begin{itemize}

\item
Each world has a ``{\sl Name}''. \index{World Name} Usually this is
the same as the ``{\sl File Name}'' \index{World File Name} in which
the world is stored, but this need not be the case. The file name is
the name of the file in the ``{\sl OBJECTS/worlds}'' \index{Echo
Objects} directory pointed to by the {\sc echo\_location} environment
variable. There can only be one world per file, but many worlds
(stored in files with different names) may have the same name.

\item
Each world has some ``{\sl Number Of Resources}.'' \index{Number Of
Resources} \index{Echo Resources} The resources are named a, b,
c... depending on the number that exist. Typically this is a small
number, say three or four.

\item
The ``{\sl Rows}'' \index{Rows} and ``{\sl Columns}'' \index{Columns}
give the size of the two dimensional array of sites. \index{World
Size} \index{World Geography}

\item
The ``{\sl Trading Fraction}'' \index{Trading Fraction} determines how
much of the excess of a resource an agent gives away when it
trades. Each agent trades a particular resource, and when it gets
involved in a trading relationship with another agent, it uses this
fraction to decide how much to give away.  The excess \index{Trading
Excess} is defined to be the amount of the resource in question over
and above what the agent needs for self--replication purposes.

\item
The ``{\sl Interaction Fraction}'' \index{Interaction Fraction}
determines how many agent--agent interactions will take place each
time step. This number is multiplied by the population size to arrive
at a number of interactions. After this many interactions are
performed, the sites produce resources again and the other aspects of
an Echo cycle are executed.

\item
The ``{\sl Self Replication Fraction}'' \index{Self Replication
Fraction} determines how much of its extra resources a parent will
give to a child when self replicating. Self replication involves
making a copy of one's genome when enough resources have accrued in
the reservoirs. Once this happens, there may be extra resources, some
of which might be given to the child to ensure that it is not too weak
at birth.

\item
The ``{\sl Self Replication Threshold}'' \index{Self Replication
Threshold} determines how many copies of its genome an agent must be
capable of making before it actually makes a single one. Thus if this
value is set at 2, the agent must accumulate twice as many of each
resource in its reservoirs as it has in its genome. In this case, when
it does make a copy of itself, it will have an excess of each resource
equal to what it carries in its genome.  This extra can be divided
between the agent and the new child according to the Self Replication
Fraction described above.

\item
The ``{\sl Maintenance Probability}'' \index{Maintenance}
\index{Taxation} determines the frequency with
which sites charge a maintenance fee. This probability is used per
site per agent per Echo cycle.

\item
The ``{\sl Neighborhood}'' \index{Neighborhood} \index{Migration} can
be set to any of ``NONE'', ``EIGHT'' or ``NEWS'' to indicate how
agents can migrate in the world. The first should be clear, it
disables migration. The second means an agent can move to any of the
eight adjacent sites (assuming it is not in a corner or on an edge)
and the third allows only north, south, east or west moves.

\end{itemize}

In addition to these eleven properties, worlds have an array of {\bf
Sites} and a ``{\sl Combat Matrix}.'' \index{Combat Matrix} These can
both be edited by clicking the button at the top of the World Editor,
which will cause a window to pop up.  The sites window should contain
site file names (see below) in an array the size of the Rows and
Columns as specified in the world's properties above. The combat
matrix is a square array of side length the number of resources in the
world. Its actual use is somewhat complicated and will be described
more fully below under Combat.

\subsection{The Site Editor}
\index{Site Editor}
\label{site-editor}

Sites have ten properties, nine of which can be entered directly into
the Site Editor window in the horizontal grey rectangles:

\begin{itemize}

\item
As with worlds (and agents), sites have a ``{\sl Name}'' \index{Site
Name} and a ``{\sl File Name}'' \index{Site File Name} in which they
are stored. The actual location of the site file is in the ``{\sl
OBJECTS/sites}'' \index{Echo Objects} directory.

\item
Each site has a ``{\sl Mutation Probability}.'' \index{Mutation}
Mutation is performed in a genetic algorithm fashion. Each locus on
each genome is mutated with this probability at the end of each
cycle. The allele at a locus may ``mutate'' to the same allele value.

\item
The ``{\sl Crossover Probability}'' \index{Crossover}
\index{Recombination} determines the probability that
crossing over, or recombination, takes place when two agents reproduce
sexually. If recombination does not take place, the agents are left
untouched.

\item
The ``{\sl Random Death Probability}'' \index{Random Death} is the
probability that an agent is killed without cause at the end of a
cycle. This is typically set very low. After each cycle, every agent
is killed for no reason with this probability.

\item
The ``{\sl Production Function}'' \index{Resource Production}
determines how much of each resource the site produces at the end of
each cycle. These values should be specified separated by white
space. There should be as many of them as there are resources. The
same is true for the next three properties.

\item
The ``{\sl Initial Resource Levels}'' \index{Initial Site Resource
Levels} \index{Resource Levels Initially} are the resource levels that
the site is allocated when the world is initially created.

\item
The ``{\sl Maximums}'' \index{Maximum Resource Levels} \index{Resource
Levels Maximally} determine to what level each resource can grow if it
remains ``on the ground'' (i.e. not picked up by an agent) at a site.

\item
The ``{\sl Maintenance}'' \index{Maintenance} \index{Taxation} is the
tax charged by the site. Each agent is charged this tax after every
cycle according to the maintenance probability set for the world. This
probability is used to determine whether each agent individually gets
taxed, not whether the site will charge all agents if the probability
condition is met.

\end{itemize}

The final property of a site is its initial agent list. This can be
accessed by clicking the mouse on the ``{\sl Agents}'' \index{Site
Agent List} \index{Agents List At A Site} button at the top of the
site editor. A window will pop up. Each line in this window is used to
specify some number of agents. The agent names must be agent file
names. Each name may be followed by white space and a decimal number
indicating the number of agents of that type that should be created
contiguously at that location. The agents in this list form the agent
stack at that site.

\subsection{The Agent Editor}
\index{Agent Editor}
\label{agent-editor}

Agents have eleven properties:

\begin{itemize}

\item
As with worlds and sites, agents have both a ``{\sl Name}''
\index{Agent Name} and a
``{\sl File Name}.'' \index{Agent File Name} The file name simply
references a file in the directory ``{\sl OBJECTS/agents}.''
\index{Echo Objects}

\item
Each agent has a ``{\sl Trading Resource}.'' \index{Echo Resources}
\index{Trading Resources}  This is the resource that
the agent, initially, trades. This may be mutated in the course of a
run.

\item
Each agent is provided with some ``{\sl Initial Resources}.''
\index{Agent Resource Levels Initially} \index{Echo Resources}  This
specifies the resource levels in the agent's reservoir \index{Agent
Reservoirs} when it is created. There should be as many numbers here
as there are resources, each separated by white space.

\item
The ``{\sl Uptake Mask}'' \index{Uptake Masks} \index{Agent Uptake
Mask} determines what resources the agent is able to pick up directly
from the ground at the site. This should be a string of ``1'' or ``0''
characters, one for each of the resources. They should not be
separated by white space. A ``1'' indicates that the agent may pick up
this resource, and a ``0'' that it may not. This mask is subject to
mutation.

\item
The next six properties all specify tags \index{Tags} and conditions.
\index{Conditions} These are
used to determine with whom and how the agent interacts in the world.
They are described in detail elsewhere
\cite{holland-92a,holland-92b}. These ``genes'' can all grow and
shrink (even to zero length) under mutation. \index{Mutation}

\end{itemize}

\subsection{The Graphs}
\index{Graphs}

There is not much to say about the graphs. The individual graphs are
briefly described in section \ref{graphs-menu}. There is space
allocated for two of them, side by side at the bottom of the
screen. Only the species graph is ever displayed on the left. However,
since they are fully functional X windows, you can use your window
manager to resize and reposition them as you wish. The exit button on
each graph window simply closes the window. Graphs can be redisplayed
by selecting the graph in question from the Graph menu in the Control
window. The graphs (currently) update every two hundred generations
and there is no way to retrieve data once it moves off the left of the
graph (other than by replaying the world). This will hopefully be
changed sometime.

\subsection{Textual Output}
\index{X Windows}
\index{Text Windows}

Textual output appears in the {\bf xterm} window. This window should have a
scroll bar so you can examine lengthy output. The {\bf --sl} option to
{\bf xterm} can be used to set the number of lines that are saved off
the top of the screen for scrollback purposes. If you invoke Echo with
the supplied {\bf run--echo} script (or by adding the suggested line
to your window manager's startup file), the {\bf xterm} created will
have a scroll bar and will save two thousand lines of previous text.

\section{Creating A World}
\index{Creating A World}

The Echo distribution comes with several worlds, sites and agents in
the {\bf OBJECTS} directory. You should never need to directly edit
the files under this directory, the world, site and agent editors are
designed to read and write these for you.

These editors are described in sections \ref{world-editor},
\ref{site-editor} and \ref{agent-editor}.

To create a world (including its sites and their agents) from scratch,
choose ``{\sl New World}'' from the ``{\sl Edit}'' menu. This will
display a blank world editor. Fill in the details of your new world,
including a name and file name, and then ``{\sl Save}'' it. Notice
that you need to fill in the sites array. Click on the
``{\sl Sites}'' button to display a text window in which you enter
the site names (actually the site file names). This should be an array
that has as many rows and columns as you specify in the world editor.
For an example, take a look at the world {\bf 4x4--insects} in the
Echo distribution. This is a world with four rows and columns. Its
site array specifies the same site 16 times (the site is also called
{\bf 4x4--insects}.

Another way to create a new world is to copy an existing
one. \index{Copying Worlds} This is easily done. Suppose you wish to
make a copy of the {\bf insects} world. Read it into the world editor
and change its name and file name. Then make the other changes you
want and save the new world. The same principle applies to making
copies of sites and agents. \index{Copying Sites} \index{Copying
Agents}

Note that worlds refer to sites (in the ``{\sl Sites}'' text window of
the world editor) and that sites in turn refer to agents (in the
``{\sl Agents}'' text window of the site editor). This is obvious, but
the fact that it implies a connection between the three editors you'll
be using may not be so clear. If you are unclear about how to create
world, sites and agents, the best way to look is to examine the ones
in the Echo distribution.

If you don't own the Echo distribution files, you will not be able to
save your creations in the {\bf OBJECTS} directory that the
distribution came with. \index{Write Permissions} There is a simple
solution to this: simply copy the distribution's {\bf OBJECTS}
directory elsewhere, and change your {\sc echo\_location} environment
variable to indicate where your personal Echo objects are to be
found. For example, if the Echo distribution is located in {\bf
/usr/local/Echo}, you can create your own objects directory in your
home directory with

\begin{shell}
\prompt\ cd
\prompt\ cp --r /usr/local/Echo/OBJECTS Echo--objects
\end{shell}

and then change {\sc echo\_location} to be the {\bf Echo--objects}
directory in your home directory.

Of course, you don't need to copy the distribution's entire
{\bf OBJECTS} directory, you can just create your own. Echo expects
the directory specified in the {\sc echo\_location} variable to
contain three sub--directories, named {\bf worlds}, {\bf sites}, and
{\bf agents}.

\section{Running A World}
\index{Running A World}

\subsection{Choosing A World To Run}
\index{Choosing A World}

The first step in running a world is choosing which world to run.  To
do this, select the ``{\sl Choose World}'' option from the Running
menu in the Control window. This pops up a file selector showing the
available worlds. Once you choose a world, you can use the various
entries in the ``{\sl Running}'' menu to actually run it.

\subsection{Editing a Running World}
\index{Editing Running Worlds}

Although you cannot yet change attributes of sites and worlds while a
run is in progress, you can make changes to the agents at a site. If
you choose ``{\sl Stack}'' \index{Stack} in the Edit menu, a window
will pop up containing the agents present at the site (you will be
asked to enter site coordinates if you have multiple sites). In this
window you can directly edit the genome of any agent. You can search
for a particular string (use Control--s), you can read in a file of
agents from disk (use Meta--i) or you can use the editor to remove or
replicate some number of agents. Once you are done, you can use the
``{\sl Amen}'' command button to make your new site reality. The help
button will pop up a box describing the genome representation.

\subsection{Verbose Output}
\index{Output}
\index{Text Windows}

The ``{\sl Set Verbose Level...}'' \index{Verbose Level} option in the
``{\sl Control}'' menu \index{Control Menu} can be used to make
informative text appear in the X window. This has already been
described above. Interesting output includes that for the letters {\sf
g} (generation number), \index{Generation Number} {\sf s} (species
summary), \index{Species Summary} {\sf u} (details of mutations),
\index{Mutation} {\sf k} (who is killing whom), and {\sf d} (to see
extinct genomes). \index{Extinct Genomes} \index{Dead Agents}

\subsection{Cluster Analysis}
\index{Cluster Analysis}
\label{cluster-analysis}

It is possible to perform a cluster analysis based on the genetic
distance between the genomes. This can be done for the living genomes
or all those that have ever lived. The output will appear in tree
form in the {\bf xterm} window.

The clustering is done by calculating the genetic distance between all
pairs of agents. This distance is defined as the minimum number of
mutations needed to transform the genome of one agent into the genome
of the other. The two agents that are most closely related are grouped
into a ``cluster''. By also defining the distance from an agent to a
cluster and between separate clusters, it is simple to build a tree
showing how closely related the individuals in a population are. This
is done by successively merging the closest clusters (or individuals)
into a larger cluster until only one remains. This process naturally
defines a hierarchy of cluster relatedness which can be displayed as a
tree.  The clustering algorithm runs in $O(n^3 )$, where $n$ is the
number of individuals at the outset, so you may have to be patient if
you have a large population.


\subsection{Schema Tracking}
\index{Schema Tracking}
\label{schema-tracking}

While the world is not running, you can enter a schema to graph by
selecting that option in the ``{\sl Examine}'' \index{Graphing A
Schema} \index{Choosing A Schema To Graph} menu. Genes are separated
on the chromosome by an underscore (\_). The meta--characters
$\hat{\:}$ and
\$ can be used to tie the regular expression \index{Regular
Expressions} to the beginning and end of the genome
respectively. Square brackets ([\,]) can be used to denote a set of
characters any one of which constitutes a match. A star (*) represents
any number of the preceding expression and a plus (+) represents one
or more of the preceeding expression. All this is very standard
regular expression syntax, and this explanation is meant to be brief
at best.

As an example, we could look for agents that had the string ``aa''
somewhere in their mating tag\footnote{Assume a world with four
resources.}. This is matched by the expression \\
\noindent
$\hat{\:}$[abcd]*\_[abcd]*\_[abcd]*aa

which allows anything (including nothing) in the first two genes and
then anything (including nothing) followed by two a's in the third
gene. The range ``[abcd]'' could also have been represented with
``[a--d]''.

To see the level of this schema in the population, select the Schema
Level graph in the Graphs menu.

\newpage
\section*{Appendices}
\appendix

\section{Using FTP}
\index{Using Ftp}
\label{using-ftp}

The following illustrates how the file may be retrieved. Your UNIX
prompt is a percent sign. What the system prints is shown in a {\sl
slanted} font, and what you type, as usual, is in {\sf sans serif}.

\vspace{0.5in}
\noindent
\begin{sl}
\prompt\ {\sf ftp santafe.edu}                                                  \\
Connected to santafe.edu.                                                       \\
220-  * * * * * * * * * * * * * * * * * * * * * * * * * * * * * * * * * * * * * \\
220-                                                                            \\
220-   Anonymous access to the FTP area at SantaFe.edu is available:            \\
220-                                                                            \\
220-   ftp ftp.santafe.edu                                                      \\
220-   Login: anonymous                                                         \\
220-   Password: (Your email address)                                           \\
220-                                                                            \\
220-  * * * * * * * * * * * * * * * * * * * * * * * * * * * * * * * * * * * * * \\
220-                                                                            \\
220 sfi FTP server (Version 2.0WU(10) Mon Apr 12 10:49:51 MDT 1993) ready.      \\
Name (santafe.edu:terry): {\sf anonymous}                                       \\
331 Guest login ok, send your complete e-mail address as password.              \\
Password: {\sf Enter your email address.}                                       \\
230-                                                                            \\
230-  SFI FTP - SFI Anonymous FTP root directory.                               \\
230-                                                                            \\
230-  Directory: $\sim$ftp           Path: $\sim$ftp                            \\
230-                                                                            \\
230-    Welcome to the FTP area at SantaFe.edu...                               \\
230-                                                                            \\
230-    Everything useful is in the pub directory.  Type ``cd pub'' ...         \\
230-                                                                            \\
230-    If you have any questions or problems with this service,                \\
230-    please send email to $<$ftp@santafe.edu$>$.                             \\
230-                                                                            \\
230-Please read the file README.Z                                               \\
230-  it was last modified on Fri Mar 19 15:29:44 1993 - 179 days ago           \\
230 Guest login ok, access restrictions apply.                                  \\
ftp$>$ {\sf cd pub/Users/terry/echo}                                            \\
250 CWD command successful.                                                     \\
ftp$>$ {\sf binary}                                                             \\
200 Type set to I.                                                              \\
ftp$>$ {\sf get Echo--1.0.tar.Z}                                                \\
200 PORT command successful.                                                    \\
150 Opening BINARY mode data connection for Echo-1.0.tar.Z (925969 bytes).      \\
226 Transfer complete.                                                          \\
local: Echo--1.0.tar.Z remote: Echo--1.0.tar.Z                                  \\
925969 bytes received in 6.1e+02 seconds (5.2 Kbytes/s)                         \\
ftp$>$ {\sf quit}                                                               \\
\end{sl}

Now you have retrieved the entire distribution. The distribution
consists of a number of files and directories, which were archived
into a single file that was then compressed. The next job is to
reverse these steps to recover the original Echo files.  To do this:

\begin{shell}
\prompt\ uncompress Echo--1.0.tar.Z \\*
\prompt\ tar xf Echo--1.0.tar \\*
\end{shell}

This should result in the creation of a new directory, called {\bf
Echo--1.0}. Check that that directory has been created. If so, and you
have received no error messages, it is safe to remove the bundled
distribution file with

\begin{shell}
\prompt\ rm Echo-1.0.tar
\end{shell}

\newpage
\section{Environment Variables and X Resources}
\label{setting-up}

This appendix deals with setting up environment variables and X
resources for those who preferred not to have this done automatically
in section \ref{simple-setup}.

You may wish to automatically install just the X resources, or just
the shell's environment variables. Do this with either

\begin{shell}
\prompt\ make x--setup
\end{shell}

or

\begin{shell}
\prompt\ make sh--setup
\end{shell}

In both cases, the shell script that is invoked (either {\bf
echo--x--setup} or {\bf echo--sh-setup}) tries to find an appropriate
file to append some text to. It is fairly conservative and always
makes a backup copy of any file it alters (in a file whose name ends
with {\bf .bak}).

There are several reasons why you may choose not to have either of
these setups done automatically. If you are an experienced UNIX user,
your shell startup and X resource files are probably not something
you'll feel comfortable having someone else's shell script edit
automatically. In this case, you should be able to decide how you wish
to do what follows.

Here are more details about exactly what needs to be done:

\begin{itemize}

\item
The file {\bf Echo.ad} contains X window \index{X Windows} resource
\index{X Resources} specifications.
These set the various colors of the Echo interface, set the sizes and
locations of the pieces of the interface and so on. If you know what
all this means, you should put the contents of this file somewhere
that the X toolkit will find them when Echo starts.

\item
Echo looks for certain environment variables \index{Environment
Variables} that can be used to influence its behavior. None of these
are required, but at least one is highly recommended.

Echo \index{Echo Objects} comes with a collection of pre--defined
worlds, sites and agents which are located in the {\bf OBJECTS}
directory. If you have installed Echo in the directory {\bf
/usr/local/echo} then you should set an environment variable called
{\sc echo\_location} and give it the value

\begin{verse}
{\bf /usr/local/echo/OBJECTS}
\end{verse}

If you do not do this, you will always have to invoke Echo from the
directory where you installed it for it to see the {\bf OBJECTS}
directory. Eventually you may have your own directory of Echo objects
and you can change this variable to point to it.

The way to set the environment variable depends on the shell you are
using. If you are in csh, the simplest thing to do is to place the
line

\begin{shell}
setenv ECHO\_LOCATION /usr/local/echo/OBJECTS
\end{shell}

in your {\bf $\sim$/.login} file. Then, whenever you log in, the variable
will be automatically set for you. To make it affect the current login
session, you can also type the line at the shell's prompt.

If you are not using csh or a variant of it, you should place the
following lines in your {\bf $\sim$/.profile} file (or equivalent),

\begin{shell}
ECHO\_LOCATION=/usr/local/echo/OBJECTS \\*
export ECHO\_LOCATION \\*
\end{shell}

And you can either log out and in again or type those two lines to the
current shell to have them affect this session.

If you aren't sure what shell you are running, type

\begin{shell}
\prompt\ echo \$SHELL
\end{shell}

and if the output ends in the letters {\sl csh}, then use the first
method above. If not, use the second\footnote{If the shell name ends
in {\sl bash}, the lines should go into your {\bf
$\sim$/.bash\_profile} file if you have one and your {\bf
$\sim$/.profile} if not.}.

The other three environment variables are less important, but will
make Echo start in a more attractive fashion. The variables and their
recommended values are given below.  These settings will make Echo
display a world, a site and an agent when it starts up. Otherwise the
three editing areas will be blank.

These variables can be set in exactly the same way that you set the
variable {\sc echo\_location} above. If you use C shell or a variant
of it, the following goes into your {\bf $\sim$/.login}:

\begin{shell}
setenv ECHO\_WORLD insects \\*
setenv ECHO\_SITE insects \\*
setenv ECHO\_AGENT fly \\*
\end{shell}

Otherwise, the following goes into your {\bf $\sim$/.profile}:

\begin{shell}
ECHO\_WORLD=insects \\*
ECHO\_SITE=insects \\*
ECHO\_AGENT=fly \\*
export ECHO\_WORLD ECHO\_SITE ECHO\_AGENT \\*
\end{shell}

\end{itemize}

\newpage
\section{Emacs Keybindings}
\index{Emacs Keybindings}
\label{emacs-keybindings}

In this section an uppercase {\sf C} will be used to represent the use
of the Control key and an uppercase {\sf M} to represent the Meta
key. Thus, {\sf C--x} indicates that you should hold down the Control
key and while doing so, type an {\sf x}. The Meta key is used in an
identical fashion, e.g.  {\sf M--d} tells you to first hold down Meta
and while holding it, press the {\sf d} key.

Table \ref{emacs-keybindings-table} shows the most useful key bindings
present in all text windows.

\begin{table}
\begin{center}
\begin{tabular}{||l|l||l|l||}
\hline
{\sf C--a} & Beginning Of Line.         & {\sf M--b}   & Backward Word. \\
{\sf C--b} & Backward Character.        & {\sf M--d}   & Delete Next Word. \\
{\sf C--d} & Delete Next Character.     & {\sf M--f}   & Forward Word. \\
{\sf C--e} & End Of Line.               & {\sf M--i}   & Insert File. \\
{\sf C--f} & Forward Character.         & {\sf M--v}   & Scroll Backwards. \\
{\sf C--h} & Delete Previous Character. & {\sf M--$<$} & Beginning Of File. \\
{\sf C--k} & Kill To End Of Line.       & {\sf M--$>$} & End Of File. \\
{\sf C--l} & Redraw. & & \\
{\sf C--n} & Next Line. & & \\
{\sf C--p} & Previous Line. & & \\
{\sf C--r} & Search Backwards. & & \\
{\sf C--s} & Search Forwards. & & \\
{\sf C--t} & Transpose Characters. & & \\
{\sf C--v} & Scroll Forward. & & \\
{\sf C--w} & Delete Selected Region. & & \\
{\sf C--y} & Paste Deleted Region. & & \\
\hline
\end{tabular}
\end{center}
\caption{Useful Emacs Keybindings For Text Windows}
\label{emacs-keybindings-table}
\end{table}

\newpage
\section{Echo Widgets}
\index{Echo Widgets}
\label{echo-widgets}

If you are not installing Echo on a SPARC machine, you will need to
obtain and install various widget sets on your machine before you can
make Echo. These are all freely available via {\bf ftp}. If you are
not familiar with this sort of installation, it might be best to
consult your system administrator.

The widget packages all create libraries that Echo must be linked with
when it is compiled. The {\bf Makefile} definition of {\bf LIBS} in
the distribution assumes that these libraries can be found in the {\bf
WIDGETS} directory. In fact, these libraries can be installed anywhere
that is convenient, as long as the {\bf Makefile} is altered to
reflect their location.

The various widget libraries and their {\bf ftp} locations are as
follows:

\begin{itemize}

\item {\bf Athena Plotter Widgets} \\
Machine: {\bf ftp.uni--paderborn.de}, Location: {\bf /unix/tools}.

\item {\bf Free Widget Foundation Widgets} \\
Machine: {\bf ftp.let.rug.nl}, Location: {\bf /ftp/pub/FWF/fwf.tar.Z}.

\item {\bf 3D Athena Widgets} \\
Machine: {\bf ftp.x.org}, Location: {\bf /contrib/Xaw3d}.

\end{itemize}

Each of these widget sets comes with instructions on how to make and
install the libraries that support the various widgets. The FWF and
Xaw3d widgets are available for {\bf ftp} from many locations. Use
{\bf xarchie} to find other sites.

Once these libraries have been created and Echo's {\bf Makefile} knows
where to find them, you should be able to proceed with the Echo
installation itself.

I apologize if you have to go through this procedure. It is the result
of using the X toolkit and the need to find useful (free) widgets. I
hope this will be solved in later Echo versions by using {\bf TCL} --
though this will require that the machine receiving Echo has (or
obtains) the {\bf TCL} libraries...

\newpage
\begin{thebibliography}{99}

\bibitem {holland-92a}
Holland, John H. (1992a).
{\em The Echo Model},
in ``Proposal for a Research Program in Adaptive Computation''.
Santa Fe Institute, July 1992.

\bibitem {holland-92b}
Holland, John H. (1992b).
{\em Adaptation in Natural and Artificial Systems}, 2nd Ed.,
MIT Press, 1992.

\bibitem {holland-93}
Holland, John H. (1993).
{\em ECHOING EMERGENCE : Objectives, Rough Definitions, and
Speculations for Echo-class Models},
To appear in ``Integrative Themes'', George Cowan, David Pines and
David Melzner Eds. Santa Fe Institute Studies in the Sciences of
Complexity, Proc. Vol XIX. Reading, MA: Addison--Wesley 1993.

\end {thebibliography}

\documentstyle [12pt] {article}
% \makeindex

\begin{document}

\newenvironment{shell}{\begin{verse}\begin{sf}}{\end{sf}\end{verse}}
\newcommand{\prompt}{{\bf \%}}

\title {{\bf An Introduction to SFI Echo.}}

\author
{
Terry Jones
\thanks
{
Santa Fe Institute, 1660 Old Pecos Trail, Suite A.
Santa Fe NM 87501.
email: terry@santafe.edu
}
\and
Stephanie Forrest
\thanks
{
Dept. of Computer Science,
University of New Mexico.
Albuquerque NM  87131.
email: forrest@cs.unm.edu
}
}


\pagenumbering{empty}
\maketitle
\newpage

% \pagenumbering{empty}

\pagestyle{plain}
\pagenumbering{roman}
\tableofcontents
\newpage
\pagestyle{headings}
\pagenumbering{arabic}

\section{Introduction}

This report is concerned with an implementation of a family of models
of complex adaptive systems called Echo models. In what follows, you
will find:

\begin{itemize}

\item
An introduction to Echo.

\item
Information on how to obtain, install and run the Echo system.

\item
A description of Echo's graphical interface.

\item
Information on running Echo.

\end{itemize}

\subsection{Echo}

Echo is a model of complex adaptive systems formulated by John Holland
\cite{holland-92a,holland-92b,holland-93}. 
It abstracts away virtually all of the physical details of real
systems and concentrates on a small set of primitive agent--agent and
agent--environment interactions.  The extent to which Echo captures
the essence of real systems is still largely undetermined.  The goal
of Echo is to study how simple interactions among simple agents lead
to emergent high--level phenomena such as the flow of resources in a
system or cooperation and competition in networks of agents (e.g.,
communities, trading networks, or arms races).

An Echo world consists of a lattice of sites. Each is populated by
some number of agents, and there is a measure of locality within each
site.  Sites produce different types of renewable resources; each type
of resource is encoded by a letter (e.g., ``a,'' ``b,'' ``c,'' ``d'').
Different types of agents use different types of resources and can
store these resources internally. Sites charge agents a maintenance
fee or tax. This tax can also be thought of as metabolic cost.

Agents fight, trade and reproduce. Fighting and trading
result in the exchange of resources between agents. There is sexual
and non--sexual reproduction, sexual reproduction
results in offspring whose genomes are a combination of those
of the parents. Each agent's genome encodes various genes
which determine how it will interact with other agents (e.g., which
resource it is willing to trade, what sort of other agents it will
fight or trade with, etc.).  Some of these genes determine phenotypic
traits, or ``tags'' that are visible to other agents.  This
allows the possibility of the evolution of social rules and
potentially of mimicry, a phenomenon frequently observed in
natural ecosystems. The interaction rules rely only on string
matching.

Echo has no explicit fitness function guiding selection and
reproduction.  An agent self--reproduces when it accumulates a
sufficient quantity of each resource to make an exact copy of its
genome.  This cloning is subject to a low rate of mutation.

In preliminary simulations, the Echo system has demonstrated
surprisingly complex behavior (including something resembling a
biological ``arms race'' in which two competing agent types develop
progressively more complex offensive and defensive combat strategies),
ecological dependencies among different species, and sensitivity (in
terms of the number of different phenotypes) to differing levels of
renewable resources.

Ideally, Echo will allow the modeling of a diverse range of complex
adaptive systems without the need for a specialized model for each to
be developed. Typically, the people who know the most about any
particular real--world complex adaptive system are not the people who
can also develop sophisticated models that can be used as tools to
increase understanding. Echo aims to provide a useful modeling tool or
a starting point for the development of a model.

As a cautionary note, one must be a little careful when using the term
``Echo.''  Properly, Echo refers to a large family of models. As
described here, Echo will refer to the implementation developed at the
Santa Fe Institute.

Several versions of the system have been developed by Holland, and
there are significant differences between these. Echo has been
described in
\cite{holland-92a,holland-92b,holland-93}. These descriptions
represent snapshots of ongoing thought about Echo models. The version
implemented here is closest to that described in \cite{holland-92a}.
For further details, refer to the above sources.

\section{Getting Started}

\subsection{A Note On Fonts}
\index{Fonts}

The following conventions are used consistently throughout this
report:

\begin{itemize}

\item
\index{UNIX Commands}
File names and UNIX commands appear in {\bf bold face}. These will
normally be references to files and commands that you might need or
use, not things you will be expected to enter immediately.

\item
\index{UNIX Environment Variables}
Environment variables are shown in {\sc small caps} and are always
completely {\sc uppercase}.

\item
Commands you are expected to type or put in a file are shown in {\sf
sans serif}. This is true even if what you are asked to type is a file
name or an environment variable.

\item
Anything printed by the system appears in a {\sl slanted} font.

\item
Things you will see in the Echo interface are typeset in a {\sl
slanted} font, as for other system output, but, additionally, are
enclosed in ``{\sl double quotes}''. The text so enclosed is exactly
what you can expect to see in Echo's interface. For example, this font
will be used when discussing the various options you'll see in Echo's
popup menus.\index{Pop--up Menus}

\item
\index{UNIX Shell Prompt}
The UNIX shell prompt is a bold percent sign (\prompt) at the start
of a line.

\end{itemize}


\subsection{System Requirements}
\index{System Requirements}

Echo was developed on a SUN SPARC architecture running SUNOS 4.1.3. In
theory it will run on other BSD--based versions of UNIX, but may
require slight modification in some cases. The X Windows \index{X
Windows} system from MIT is required. Development was under X11R5, but
Echo should run without modification under X11R4. If you are using a
SPARC machine, no special widget sets \index{Widgets} are required,
the distribution file contains these. If not, you will need to {\bf
ftp} widgets from several locations -- this is discussed in appendix
\ref{echo-widgets} \index{Echo Widgets}.

\subsection{Porting Echo}
\index{Porting}

Unless a joint research agreement with the Santa Fe Institute exists,
there is currently no guaranteed way to get help porting Echo to a new
architecture. A possible source of help is through electronic mail to
{\bf echo@santafe.edu}. Assistance will probably be contingent on the
project being supported at SFI.

\subsection{Getting Echo Through Anonymous FTP}
\index{Using Ftp}

Echo is available via anonymous ftp from {\bf ftp.santafe.edu} in the
file

\begin{verse}
{\bf /pub/Users/terry/echo/Echo-1.0.tar.Z}
\end{verse}

The entire distribution occupies just over two megabytes of disk
space when uncompressed. You should transfer Echo, uncompress
\index{Uncompress} it and
extract the files using tar\index{Tar}.  If you are not familiar with
the workings of ftp, uncompress or tar, consult appendix
\ref{using-ftp}.


\subsection{Compiling Echo}

Once you have received and unpacked the distribution, you will need to
compile or ``make''\index{Making Echo} it. The {\bf make} command will
take care of the compilation and linking, but first you will need to
make some changes to the file {\bf Makefile}. These should be very
minor, the file contains instructions to help you. You should only
have to change two lines, to indicate where the X libraries and
include files are on your system. If you don't know, you can try to
find them or ask your system manager.

Once you have done this, you can simply type

\begin{shell}
\prompt\ make all
\end{shell}

and make will do the rest. You should see about twenty files get
compiled.  When the make is completed, you should have a file called
{\bf Echo} which is the executable program. This process also creates
you a shell script, {\bf run--echo} to make it simple to run the
executable in the recommended way.

\subsection{Preparing to Run Echo}

\label{simple-setup}

Two additional things need to be done before running Echo.

\begin{itemize}

\item
X resources for Echo need to be installed.

\item
Shell environment variables need to be set.

\end{itemize}

The distribution contains programs that will do this for you if do not
want to do it yourself, or do not know how. If you'd like it taken
care of for you, use

\begin{shell}
\prompt\ make simple--setup
\end{shell}

This will add X resource defaults to your X startup file and set
environment variables in your shell's startup file. If you have used
this option, you should now log out and then log back in, as these
changes will not affect your current login session\footnote{They will
be automatically set for you in all future login sessions.}.

If you'd prefer to do this kind of installation yourself, you should
see the details about what needs to be done in appendix \ref{setting-up}.


\subsection{Running Echo}
\index{Running Echo}
\label{running-echo}

Once you have completed the above, you should be able to run Echo.
Echo writes its textual output to the {\bf xterm} from which it was
invoked.  The X resources \index{X Resources} you just installed
specify the colors, sizes and locations of all the pieces of the Echo
interface.  These pieces are designed to fit around a central {\bf
xterm} window.

There are two ways you can arrange things so that an {\bf xterm} is
created in the right place (and the right color):

\begin{itemize}

\item
The simplest way is to use the shell script {\bf run--echo} that was
made for you when you did the original {\bf make}.  This will do all
the work of setting up an {\bf xterm}, invoking Echo, and making sure
the interface is configured correctly.

\item
Slightly more work, but simpler in the long run is to put Echo into a
popup menu.\index{Pop--up Menus} This is achieved by adding a line to
your window manager's initialization \index{Window Managers}
file. This file's name depends on your window manager. A likely
location is {\bf $\sim$/.twmrc}, or some other file in your home
directory whose name starts with a period ({\bf .}) and ends in the
string ``{\bf wmrc}''.  If you don't know where your window manager
initialization file is or can't find it, ask your system manager. The
file {\bf .window\_manager} in the Echo distribution contains a line
that can probably be usefully inserted in this file\footnote{{\bf
Warning}: it is not as simple as just adding the line to the end of
this file. If you are unsure about what you're doing, get help.}.

\end{itemize}

\subsection{A Sample Echo Run}
\index{Demo Run}
\index{Sample Run}
\index{Running Echo}

This section walks you through a quick Echo demo, without too much
explanation.

Once you have installed the X resources and the shell environment
variables, you are in a position to run Echo. Try it now -- either
from your window manager or with

\begin{shell}
\prompt\ ./run--echo
\end{shell}

You should see a number of windows appear on your screen. These are
explained in the next section. For now notice the main menu bar at the
top of the screen and the {\bf xterm} in the middle.

\begin{itemize}

\item
The first step in running a world is to choose the world you intend to
run. To do this, click on the red menu button at the top of the screen
that says ``{\sl Running}''. Hold the mouse button down and drag the
pointer down to ``{\sl Choose World}'' and let go.\index{Choosing A
World} \index{Pop--up Menus}

\item
A file selector will pop up. This allows you to walk through the
directory structure to find a file containing the specification of a
world you want to run. You should see ``insects'' in the scrollable
region on the right. Click on this word and the line it is on should
change to white on black. Now click on ``{\sl OK}'' to select the
world.

\item
Look at the ``{\sl Species}'' graph.\index{Species Levels Graph} It
now has a legend, showing ``{\sl ant}'', ``{\sl fly}'' and ``{\sl
cat}'' (``cat'' is short for caterpillar). The legend has appeared as
the world you selected contains a site that contains all of these
agent types.

\item
Now select ``{\sl Run 10 Generations}'' from the ``{\sl Running}''
menu.\index{Running Menu} You should see the two graphs update and the
{\bf xterm} will show the text: ``{\sl Run through generation 10.}''

\item
Let's edit the world and remove a few agents.\index{Editing The Stack}
\index{Stack Editing} Click on the ``{\sl Edit}'' menu button and drag
the pointer down to ``{\sl Stack}''. A window will appear showing you
the population of agents present at the site. This is called the
``stack'' at that site. It's a one--dimensional array of agents. Use
the left mouse button to move the small arrowed cursor to the start of
some line. Now hit Control--k twice on your keyboard. You should see
the line in the stack disappear -- and with it some unfortunate
agent. Control--k is an Emacs keybinding that kills the text until the
end of the line. See appendix
\ref{emacs-keybindings} for more information on the keybindings if
these are not familiar to you.

Now that you've removed the agent (or agents if you were feeling
enthusiastic), hit the ``{\sl Amen}'' button at the top of the stack
editor to make it so.

\item
Before restarting the world, choose ``{\sl Set Verbose Level...}''
from the ``{\sl Control}'' menu at the top of the
screen.\index{Verbose Level} Here you can type letters to see certain
types of output in the {\bf xterm} window.  In the small gray window
at the bottom of the panel that popped up, type the following
incantation: ``gsukbx''. You can see what output each of these letters
will produce by reading the text above the small text window. To type
in the text window you'll need to move the mouse into it. Now click on
the ``{\sl OK}'' button to dismiss the window.

\item
Choose ``{\sl Run 1 Generation}'' from the ``{\sl Running}''
menu.\index{Running Menu} You should see various output appear in the
{\bf xterm}. This will likely include information on the agents that
were taxed during the generation and those that went bankrupt. There
is a small chance you'll see a mutation or an agent killed in
combat. The end of the output will show a population summary.

\item
Now we'll run to generation 100. Choose ``{\sl Run Until...}'' from
the ``{\sl Running}'' menu.\index{Running Menu} In the small window
that pops up, enter 100 and then click on ``{\sl OK}''. You should see
the graphs rapidly update to generation 100. You'll also see a lot of
text flashing by in the {\bf xterm} window. Often it is good to turn
off all such output when you are about to run for a large number of
generations.

\item
Select ``{\sl Variant Levels}''\index{Variant Levels Graph} from the
``{\sl Graphs}'' menu. The graph will appear where the ``{\sl
Worldwide Population Level}'' graph was (actually it's just on top of
it). The top line (in red) shows the number of different genomes that
have ever existed in the world. The bottom (blue) line shows the
number that are currently alive. You can compare this number with the
worldwide population level, and it will probably be quite a lot
smaller. Obviously, some genomes have multiple copies -- presumably
because they are doing something right.

\item
Finally, choose ``{\sl Cluster Living Population}''\index{Cluster
Analysis} from the ``{\sl Examine}'' menu.\index{Examine Menu} You
will see a tree displayed in the {\bf xterm} window that indicates how
closely related the agents in the world are.  You will probably need
to resize (or scroll) your {\bf xterm} window to see the whole
tree. This tree will be explained in more detail later.

\item
To exit Echo, choose ``{\sl Exit}''\index{Exiting Echo} from the
``{\sl Control}'' menu.  Of course you can continue to play if you
like.

\end{itemize}

The next section gives more details about the kinds of things you just
saw and did.


\section{The Echo Interface}
\index{Echo Interface}

After starting Echo, you should see seven windows on your
screen. Starting in the center at the top and moving
counter--clockwise, these are the Control window, the World Editor,
the Site Editor, the Agent Editor, a graph showing species levels, a
graph showing the worldwide population level, and an {\bf xterm}
window.

\subsection{The Overall Look And Feel}
\index{Echo Look and Feel}

There are some high level features of the interface that have been
designed to make the look and feel consistent. These are:

\begin{itemize}

\item
\index{Mouse Buttons}
In general, the interface makes no distinction between the buttons on
your mouse (assuming you have more than a single button). You can use
any button you like to push command buttons, or use popup menus.
\index{Pop--up Menus} The only exception to this rule is described
in the next point.

\item
\index{Text Windows}
Some of the areas of the screen are a steely grey color. These are
text windows. This color is used consistently to indicate an area of
the screen where you may type characters. Text in these windows will
always be black. All you need do is move the cursor into the grey area
and start to type. Each of these is in fact a small editor with
Emacs--like keybindings\footnote{If you are not familiar with Emacs,
don't worry, you can move around with mouse button 1 and use the
delete key to get rid of things you don't want. Appendix
\ref{emacs-keybindings} gives a brief introduction to editing with
Emacs.}.

In a text window, the mouse may be used to copy and paste regions in
the same fashion as in normal {\bf xterm} windows. The first button, when
held down, can be used to drag out, and thus copy, a region (which
will be highlighted). The second button pastes the copied text into
the buffer at the cursor's location. The third button can be used to
extend the copied region. This is something you should experiment with
if you are not already familiar with this kind of mouse behavior.

In addition, rapid double and triple clicks with the left mouse button
can be used to select the word the cursor is on or the line the cursor
is on. A single click with the first button just moves the text cursor
to the location of the mouse.

A search window can be popped up by typing Control--S, and the
contents of a file can be read into the buffer using Meta--I, which
pops up a window to read the file name. See the Emacs keybindings in
appendix \ref{emacs-keybindings} for more information on operations in
text windows.

\item
\index{Read Only Output}
Blue areas where white text appears, such as the main output window,
are always read--only.

\item
\index{Command Buttons}
\index{Menu Bars}
Command and menu buttons always have a red background and white text.
These always indicate things you can click the mouse on to have some
action performed. Command buttons carry out an action immediately,
while menu buttons pop up a menu of options.  As mentioned above, no
distinction is made here between the buttons on your mouse -- use
whichever you like. \index{Mouse Buttons}

\item
\index{File Selectors}
There are a number of occasions when Echo will need to get a file name
from the user. To do this, it will display a file selector window.
This window has various components. You may enter the file's name in
the text window at the top of the file selector, and then click on
``{\sl OK}''. Often simpler, is to use the mouse to select the file
you want. The scrollable window labeled ``{\sl Directory Contents}''
contains the list of files in the currently scanned directory. Echo
tries to make sure that the default directory is a useful one (using
the {\sc echo\_location} environment variable).

You can select a file in the scrollable list by simply clicking on it
with the mouse. The second mouse button can be used to scroll the
list. If you click on the entry labeled ``{\sl ../}'' the file
selector will read and display the contents of the parent directory.
The ``{\sl Up}'' button can also be used to move up a level.  If you
click on an entry that is a directory, the file selector will read and
display that directory's contents. Directories can be identified by
the trailing ``{\sl /}'' after their name.

The ``{\sl OK}'' button selects the file and causes the file selector
to disappear. The ``{\sl Cancel}'' button dismisses the file selector
and cancels whatever function was called that made it appear
originally. 

\end{itemize}


\subsection{The Control Window}
\index{Control Window}

The control window consists of five pull down menus. These are labeled
``{\sl Control},'' ``{\sl Edit},'' ``{\sl Running},'' ``{\sl
Graphs},'' and ``{\sl Examine}.''

\begin{itemize}

\item
The ``{\sl Control}'' menu contains three options. The first, ``{\sl
Set Verbose Level...}'' \index{Verbose Level} pops up a window that
contains lines that briefly describe types of output that can be
displayed in the main {\bf xterm} window. At the start of each of
these lines is a key letter. In the small grey window at the bottom
you may enter the letters corresponding to the types of output you
want to see in the {\bf xterm}. This output will appear following each
generation, so you must run at least one generation to see anything.

The ``{\sl Show Seed}'' \index{Random Seeds} option of the ``{\sl
Control}'' menu prints the current random seed value into the {\bf
xterm}. This allows you access at any time to the seed that was used
to set up the current run.

The ``{\sl Exit}''\index{Exiting Echo} option does exactly that, it
exits Echo completely. It does not currently ask for confirmation, so
be careful!

\item
The ``{\sl Edit}'' \index{Editing Menu} menu has seven
options. Most of these are very similar.  ``{\sl Worlds''}
\index{Editing Worlds} will pop up
a file selector to let you choose a world to edit. The file selector
can be used to wander through a directory tree and select a file
containing the specifications of a world. Use the mouse to click on
the name of the file you want (or enter it in the top grey window) and
then click the ``{\sl OK''} button to select it. The world you choose
will appear in the World Editor window in the top left corner of the
screen (more on that soon).

Choosing ``{\sl New World}'' \index{Creating New Worlds} will clear the World
Editor window and let you enter the details of a fresh world.

The options ``{\sl Sites},'' \index{Editing Sites} ``{\sl New Site},''
\index{Creating New Sites} ``{\sl Agents},'' \index{Editing Agents}
and ``{\sl New Agent}'' \index{Creating New Agents} all behave in a
similar fashion. They can be used to prepare for editing the
characteristics of sites and agents in the Site Editor (middle left)
and Agent Editor (bottom left) windows.

The final option in the Editing menu is ``{\sl Stack.''}
\index{Editing The Stack} \index{Stack Editing} This allows
you to edit a site in a running world. Since we have not yet begun to
see what happens when a world is actually running, this operation will
be described later.

\item
The ``{\sl Running}'' \index{Running Menu} menu has nine options. The
first, ``{\sl Choose World}'' \index{Choosing A World} is used to tell
Echo which world you intend to run. This option must be used before an
Echo run can begin. It also pops up a file selector so that you can
choose a world.

The next four options, \index{Running Echo} ``{\sl Run
Indefinitely,}'' ``{\sl Run 1 Generation,}'' ``{\sl Run 10
Generations,}'' and ``{\sl Run Until...}''  are all to do with running
a world for a certain time. They should all be self explanatory. The
last will pop up a window so you can enter a stopping generation
number.

The next two options, ``{\sl Pause''} \index{Pausing A Run} and ``{\sl
Continue''} \index{Continuing A Paused Run} can be used to temporarily
halt and restart a run. If you have used one of the above running
options to run until generation 500 and suddenly decide you need to
turn off the output in the main text window, you can use ``{\sl
Pause}'', turn off the output and then ``{\sl Continue}'' to allow the
run to proceed to generation 500. These two options can be used in
this manner any time Echo is running an experiment.

The ``{\sl Replay''} \index{Replaying A Run} option resets the world
and re--initializes the random number generator so that the run can be
re--done exactly. This is very useful when you would like to try some
experiment and need to stop the world at an earlier point.

The ``{\sl Seed''} \index{Random Seeds} option can be used to enter
the random seed for a run.  The seed should be set {\em before} you
choose the world to run. It is important that you perform these two in
this order. If you do not specify a random seed for the run, one will
be chosen for you. The seed chosen for you is guaranteed to be unique,
and the random number generator has been highly scrutinized for
randomness. The file {\bf random.c} contains a blow--by--blow
description of the search for an acceptable generator.

\item
The ``{\sl Graphs}'' \label{graphs-menu} \index{Graphs Menu} menu can
be used to pop up any of five graphs. Initially there are two places
where graphs appear, and two of the five are shown by default. The
species level graph always appears on the left while the other four
are on its right. Of course, since these are normal X windows, you can
move them around and resize them as you wish.  The five graphs are:

\begin{itemize}
\item
The ``{\sl Species Levels}'' \index{Species Levels Graph} graph shows
a legend (after you choose a world) and plots the population level of
all the descendants of the original members of the ``species''
\index{Species}. It is not really correct to call these groups species
(lineage is more accurate), but I will not go into that here.

\item
The ``{\sl Worldwide Population Level}'' \index{Population Level
Graph} graph shows the total number of agents alive in the world.

\item
The ``{\sl World Resource Levels}'' \index{Resource Levels Graph}
graph shows how many of each resource exist (this is often very dull
viewing).

\item
The ``{\sl Schema Level}'' \index{Schema Levels Graph} graph allows you
to track the level of a schema in the population.

\item
The ``{\sl Variant Levels}'' \index{Variant Levels Graph} graph shows
the number of genomes that have ever existed and the number of genomes
that currently exist.

\end{itemize}

\item
The ``{\sl Examine}'' \index{Examine Menu} menu allows you to look at
some properties of the populations. The first option ``{\sl Choose
Schema to Graph...}''  \index{Graphing A Schema} \index{Choosing A
Schema To Graph} allows you to enter a regular expression
\index{Regular Expressions} 
corresponding to a schema you are interested in. The regular
expression is in UNIX--style and is a pattern of resources that might
be found on an agent's genome.  The simple details of these
expressions are explained in section
\ref{schema-tracking}.  More information on regular expressions can be
found in the UNIX manual page for {\bf egrep}.

The next two options, ``{\sl Cluster living population}'' and ``{\sl
Cluster all individuals}'' \index{Cluster Analysis} both perform
cluster analysis. You may choose to have either the current population
clustered or every genome that ever existed clustered. The output will
appear in the {\bf xterm} window, which should have a scrollbar in
case the output is too long. Clustering is explained in section
\ref{cluster-analysis}.

\end{itemize}

\subsection{The World Editor}
\index{World Editor}
\label{world-editor}

The ``{\sl World Editor}'' is used to make changes to the properties
of a world.  These changes do not affect the currently running
world. You can choose the world you wish to alter (or a new world)
from the ``{\sl Edit}'' menu in the ``{\sl Control}'' window. Worlds
have thirteen properties. Eleven of these are edited by entering their
values into the grey horizontal text windows to the right of their
brief descriptions:

\begin{itemize}

\item
Each world has a ``{\sl Name}''. \index{World Name} Usually this is
the same as the ``{\sl File Name}'' \index{World File Name} in which
the world is stored, but this need not be the case. The file name is
the name of the file in the ``{\sl OBJECTS/worlds}'' \index{Echo
Objects} directory pointed to by the {\sc echo\_location} environment
variable. There can only be one world per file, but many worlds
(stored in files with different names) may have the same name.

\item
Each world has some ``{\sl Number Of Resources}.'' \index{Number Of
Resources} \index{Echo Resources} The resources are named a, b,
c... depending on the number that exist. Typically this is a small
number, say three or four.

\item
The ``{\sl Rows}'' \index{Rows} and ``{\sl Columns}'' \index{Columns}
give the size of the two dimensional array of sites. \index{World
Size} \index{World Geography}

\item
The ``{\sl Trading Fraction}'' \index{Trading Fraction} determines how
much of the excess of a resource an agent gives away when it
trades. Each agent trades a particular resource, and when it gets
involved in a trading relationship with another agent, it uses this
fraction to decide how much to give away.  The excess \index{Trading
Excess} is defined to be the amount of the resource in question over
and above what the agent needs for self--replication purposes.

\item
The ``{\sl Interaction Fraction}'' \index{Interaction Fraction}
determines how many agent--agent interactions will take place each
time step. This number is multiplied by the population size to arrive
at a number of interactions. After this many interactions are
performed, the sites produce resources again and the other aspects of
an Echo cycle are executed.

\item
The ``{\sl Self Replication Fraction}'' \index{Self Replication
Fraction} determines how much of its extra resources a parent will
give to a child when self replicating. Self replication involves
making a copy of one's genome when enough resources have accrued in
the reservoirs. Once this happens, there may be extra resources, some
of which might be given to the child to ensure that it is not too weak
at birth.

\item
The ``{\sl Self Replication Threshold}'' \index{Self Replication
Threshold} determines how many copies of its genome an agent must be
capable of making before it actually makes a single one. Thus if this
value is set at 2, the agent must accumulate twice as many of each
resource in its reservoirs as it has in its genome. In this case, when
it does make a copy of itself, it will have an excess of each resource
equal to what it carries in its genome.  This extra can be divided
between the agent and the new child according to the Self Replication
Fraction described above.

\item
The ``{\sl Maintenance Probability}'' \index{Maintenance}
\index{Taxation} determines the frequency with
which sites charge a maintenance fee. This probability is used per
site per agent per Echo cycle.

\item
The ``{\sl Neighborhood}'' \index{Neighborhood} \index{Migration} can
be set to any of ``NONE'', ``EIGHT'' or ``NEWS'' to indicate how
agents can migrate in the world. The first should be clear, it
disables migration. The second means an agent can move to any of the
eight adjacent sites (assuming it is not in a corner or on an edge)
and the third allows only north, south, east or west moves.

\end{itemize}

In addition to these eleven properties, worlds have an array of {\bf
Sites} and a ``{\sl Combat Matrix}.'' \index{Combat Matrix} These can
both be edited by clicking the button at the top of the World Editor,
which will cause a window to pop up.  The sites window should contain
site file names (see below) in an array the size of the Rows and
Columns as specified in the world's properties above. The combat
matrix is a square array of side length the number of resources in the
world. Its actual use is somewhat complicated and will be described
more fully below under Combat.

\subsection{The Site Editor}
\index{Site Editor}
\label{site-editor}

Sites have ten properties, nine of which can be entered directly into
the Site Editor window in the horizontal grey rectangles:

\begin{itemize}

\item
As with worlds (and agents), sites have a ``{\sl Name}'' \index{Site
Name} and a ``{\sl File Name}'' \index{Site File Name} in which they
are stored. The actual location of the site file is in the ``{\sl
OBJECTS/sites}'' \index{Echo Objects} directory.

\item
Each site has a ``{\sl Mutation Probability}.'' \index{Mutation}
Mutation is performed in a genetic algorithm fashion. Each locus on
each genome is mutated with this probability at the end of each
cycle. The allele at a locus may ``mutate'' to the same allele value.

\item
The ``{\sl Crossover Probability}'' \index{Crossover}
\index{Recombination} determines the probability that
crossing over, or recombination, takes place when two agents reproduce
sexually. If recombination does not take place, the agents are left
untouched.

\item
The ``{\sl Random Death Probability}'' \index{Random Death} is the
probability that an agent is killed without cause at the end of a
cycle. This is typically set very low. After each cycle, every agent
is killed for no reason with this probability.

\item
The ``{\sl Production Function}'' \index{Resource Production}
determines how much of each resource the site produces at the end of
each cycle. These values should be specified separated by white
space. There should be as many of them as there are resources. The
same is true for the next three properties.

\item
The ``{\sl Initial Resource Levels}'' \index{Initial Site Resource
Levels} \index{Resource Levels Initially} are the resource levels that
the site is allocated when the world is initially created.

\item
The ``{\sl Maximums}'' \index{Maximum Resource Levels} \index{Resource
Levels Maximally} determine to what level each resource can grow if it
remains ``on the ground'' (i.e. not picked up by an agent) at a site.

\item
The ``{\sl Maintenance}'' \index{Maintenance} \index{Taxation} is the
tax charged by the site. Each agent is charged this tax after every
cycle according to the maintenance probability set for the world. This
probability is used to determine whether each agent individually gets
taxed, not whether the site will charge all agents if the probability
condition is met.

\end{itemize}

The final property of a site is its initial agent list. This can be
accessed by clicking the mouse on the ``{\sl Agents}'' \index{Site
Agent List} \index{Agents List At A Site} button at the top of the
site editor. A window will pop up. Each line in this window is used to
specify some number of agents. The agent names must be agent file
names. Each name may be followed by white space and a decimal number
indicating the number of agents of that type that should be created
contiguously at that location. The agents in this list form the agent
stack at that site.

\subsection{The Agent Editor}
\index{Agent Editor}
\label{agent-editor}

Agents have eleven properties:

\begin{itemize}

\item
As with worlds and sites, agents have both a ``{\sl Name}''
\index{Agent Name} and a
``{\sl File Name}.'' \index{Agent File Name} The file name simply
references a file in the directory ``{\sl OBJECTS/agents}.''
\index{Echo Objects}

\item
Each agent has a ``{\sl Trading Resource}.'' \index{Echo Resources}
\index{Trading Resources}  This is the resource that
the agent, initially, trades. This may be mutated in the course of a
run.

\item
Each agent is provided with some ``{\sl Initial Resources}.''
\index{Agent Resource Levels Initially} \index{Echo Resources}  This
specifies the resource levels in the agent's reservoir \index{Agent
Reservoirs} when it is created. There should be as many numbers here
as there are resources, each separated by white space.

\item
The ``{\sl Uptake Mask}'' \index{Uptake Masks} \index{Agent Uptake
Mask} determines what resources the agent is able to pick up directly
from the ground at the site. This should be a string of ``1'' or ``0''
characters, one for each of the resources. They should not be
separated by white space. A ``1'' indicates that the agent may pick up
this resource, and a ``0'' that it may not. This mask is subject to
mutation.

\item
The next six properties all specify tags \index{Tags} and conditions.
\index{Conditions} These are
used to determine with whom and how the agent interacts in the world.
They are described in detail elsewhere
\cite{holland-92a,holland-92b}. These ``genes'' can all grow and
shrink (even to zero length) under mutation. \index{Mutation}

\end{itemize}

\subsection{The Graphs}
\index{Graphs}

There is not much to say about the graphs. The individual graphs are
briefly described in section \ref{graphs-menu}. There is space
allocated for two of them, side by side at the bottom of the
screen. Only the species graph is ever displayed on the left. However,
since they are fully functional X windows, you can use your window
manager to resize and reposition them as you wish. The exit button on
each graph window simply closes the window. Graphs can be redisplayed
by selecting the graph in question from the Graph menu in the Control
window. The graphs (currently) update every two hundred generations
and there is no way to retrieve data once it moves off the left of the
graph (other than by replaying the world). This will hopefully be
changed sometime.

\subsection{Textual Output}
\index{X Windows}
\index{Text Windows}

Textual output appears in the {\bf xterm} window. This window should have a
scroll bar so you can examine lengthy output. The {\bf --sl} option to
{\bf xterm} can be used to set the number of lines that are saved off
the top of the screen for scrollback purposes. If you invoke Echo with
the supplied {\bf run--echo} script (or by adding the suggested line
to your window manager's startup file), the {\bf xterm} created will
have a scroll bar and will save two thousand lines of previous text.

\section{Creating A World}
\index{Creating A World}

The Echo distribution comes with several worlds, sites and agents in
the {\bf OBJECTS} directory. You should never need to directly edit
the files under this directory, the world, site and agent editors are
designed to read and write these for you.

These editors are described in sections \ref{world-editor},
\ref{site-editor} and \ref{agent-editor}.

To create a world (including its sites and their agents) from scratch,
choose ``{\sl New World}'' from the ``{\sl Edit}'' menu. This will
display a blank world editor. Fill in the details of your new world,
including a name and file name, and then ``{\sl Save}'' it. Notice
that you need to fill in the sites array. Click on the
``{\sl Sites}'' button to display a text window in which you enter
the site names (actually the site file names). This should be an array
that has as many rows and columns as you specify in the world editor.
For an example, take a look at the world {\bf 4x4--insects} in the
Echo distribution. This is a world with four rows and columns. Its
site array specifies the same site 16 times (the site is also called
{\bf 4x4--insects}.

Another way to create a new world is to copy an existing
one. \index{Copying Worlds} This is easily done. Suppose you wish to
make a copy of the {\bf insects} world. Read it into the world editor
and change its name and file name. Then make the other changes you
want and save the new world. The same principle applies to making
copies of sites and agents. \index{Copying Sites} \index{Copying
Agents}

Note that worlds refer to sites (in the ``{\sl Sites}'' text window of
the world editor) and that sites in turn refer to agents (in the
``{\sl Agents}'' text window of the site editor). This is obvious, but
the fact that it implies a connection between the three editors you'll
be using may not be so clear. If you are unclear about how to create
world, sites and agents, the best way to look is to examine the ones
in the Echo distribution.

If you don't own the Echo distribution files, you will not be able to
save your creations in the {\bf OBJECTS} directory that the
distribution came with. \index{Write Permissions} There is a simple
solution to this: simply copy the distribution's {\bf OBJECTS}
directory elsewhere, and change your {\sc echo\_location} environment
variable to indicate where your personal Echo objects are to be
found. For example, if the Echo distribution is located in {\bf
/usr/local/Echo}, you can create your own objects directory in your
home directory with

\begin{shell}
\prompt\ cd
\prompt\ cp --r /usr/local/Echo/OBJECTS Echo--objects
\end{shell}

and then change {\sc echo\_location} to be the {\bf Echo--objects}
directory in your home directory.

Of course, you don't need to copy the distribution's entire
{\bf OBJECTS} directory, you can just create your own. Echo expects
the directory specified in the {\sc echo\_location} variable to
contain three sub--directories, named {\bf worlds}, {\bf sites}, and
{\bf agents}.

\section{Running A World}
\index{Running A World}

\subsection{Choosing A World To Run}
\index{Choosing A World}

The first step in running a world is choosing which world to run.  To
do this, select the ``{\sl Choose World}'' option from the Running
menu in the Control window. This pops up a file selector showing the
available worlds. Once you choose a world, you can use the various
entries in the ``{\sl Running}'' menu to actually run it.

\subsection{Editing a Running World}
\index{Editing Running Worlds}

Although you cannot yet change attributes of sites and worlds while a
run is in progress, you can make changes to the agents at a site. If
you choose ``{\sl Stack}'' \index{Stack} in the Edit menu, a window
will pop up containing the agents present at the site (you will be
asked to enter site coordinates if you have multiple sites). In this
window you can directly edit the genome of any agent. You can search
for a particular string (use Control--s), you can read in a file of
agents from disk (use Meta--i) or you can use the editor to remove or
replicate some number of agents. Once you are done, you can use the
``{\sl Amen}'' command button to make your new site reality. The help
button will pop up a box describing the genome representation.

\subsection{Verbose Output}
\index{Output}
\index{Text Windows}

The ``{\sl Set Verbose Level...}'' \index{Verbose Level} option in the
``{\sl Control}'' menu \index{Control Menu} can be used to make
informative text appear in the X window. This has already been
described above. Interesting output includes that for the letters {\sf
g} (generation number), \index{Generation Number} {\sf s} (species
summary), \index{Species Summary} {\sf u} (details of mutations),
\index{Mutation} {\sf k} (who is killing whom), and {\sf d} (to see
extinct genomes). \index{Extinct Genomes} \index{Dead Agents}

\subsection{Cluster Analysis}
\index{Cluster Analysis}
\label{cluster-analysis}

It is possible to perform a cluster analysis based on the genetic
distance between the genomes. This can be done for the living genomes
or all those that have ever lived. The output will appear in tree
form in the {\bf xterm} window.

The clustering is done by calculating the genetic distance between all
pairs of agents. This distance is defined as the minimum number of
mutations needed to transform the genome of one agent into the genome
of the other. The two agents that are most closely related are grouped
into a ``cluster''. By also defining the distance from an agent to a
cluster and between separate clusters, it is simple to build a tree
showing how closely related the individuals in a population are. This
is done by successively merging the closest clusters (or individuals)
into a larger cluster until only one remains. This process naturally
defines a hierarchy of cluster relatedness which can be displayed as a
tree.  The clustering algorithm runs in $O(n^3 )$, where $n$ is the
number of individuals at the outset, so you may have to be patient if
you have a large population.


\subsection{Schema Tracking}
\index{Schema Tracking}
\label{schema-tracking}

While the world is not running, you can enter a schema to graph by
selecting that option in the ``{\sl Examine}'' \index{Graphing A
Schema} \index{Choosing A Schema To Graph} menu. Genes are separated
on the chromosome by an underscore (\_). The meta--characters
$\hat{\:}$ and
\$ can be used to tie the regular expression \index{Regular
Expressions} to the beginning and end of the genome
respectively. Square brackets ([\,]) can be used to denote a set of
characters any one of which constitutes a match. A star (*) represents
any number of the preceding expression and a plus (+) represents one
or more of the preceeding expression. All this is very standard
regular expression syntax, and this explanation is meant to be brief
at best.

As an example, we could look for agents that had the string ``aa''
somewhere in their mating tag\footnote{Assume a world with four
resources.}. This is matched by the expression \\
\noindent
$\hat{\:}$[abcd]*\_[abcd]*\_[abcd]*aa

which allows anything (including nothing) in the first two genes and
then anything (including nothing) followed by two a's in the third
gene. The range ``[abcd]'' could also have been represented with
``[a--d]''.

To see the level of this schema in the population, select the Schema
Level graph in the Graphs menu.

\newpage
\section*{Appendices}
\appendix

\section{Using FTP}
\index{Using Ftp}
\label{using-ftp}

The following illustrates how the file may be retrieved. Your UNIX
prompt is a percent sign. What the system prints is shown in a {\sl
slanted} font, and what you type, as usual, is in {\sf sans serif}.

\vspace{0.5in}
\noindent
\begin{sl}
\prompt\ {\sf ftp santafe.edu}                                                  \\
Connected to santafe.edu.                                                       \\
220-  * * * * * * * * * * * * * * * * * * * * * * * * * * * * * * * * * * * * * \\
220-                                                                            \\
220-   Anonymous access to the FTP area at SantaFe.edu is available:            \\
220-                                                                            \\
220-   ftp ftp.santafe.edu                                                      \\
220-   Login: anonymous                                                         \\
220-   Password: (Your email address)                                           \\
220-                                                                            \\
220-  * * * * * * * * * * * * * * * * * * * * * * * * * * * * * * * * * * * * * \\
220-                                                                            \\
220 sfi FTP server (Version 2.0WU(10) Mon Apr 12 10:49:51 MDT 1993) ready.      \\
Name (santafe.edu:terry): {\sf anonymous}                                       \\
331 Guest login ok, send your complete e-mail address as password.              \\
Password: {\sf Enter your email address.}                                       \\
230-                                                                            \\
230-  SFI FTP - SFI Anonymous FTP root directory.                               \\
230-                                                                            \\
230-  Directory: $\sim$ftp           Path: $\sim$ftp                            \\
230-                                                                            \\
230-    Welcome to the FTP area at SantaFe.edu...                               \\
230-                                                                            \\
230-    Everything useful is in the pub directory.  Type ``cd pub'' ...         \\
230-                                                                            \\
230-    If you have any questions or problems with this service,                \\
230-    please send email to $<$ftp@santafe.edu$>$.                             \\
230-                                                                            \\
230-Please read the file README.Z                                               \\
230-  it was last modified on Fri Mar 19 15:29:44 1993 - 179 days ago           \\
230 Guest login ok, access restrictions apply.                                  \\
ftp$>$ {\sf cd pub/Users/terry/echo}                                            \\
250 CWD command successful.                                                     \\
ftp$>$ {\sf binary}                                                             \\
200 Type set to I.                                                              \\
ftp$>$ {\sf get Echo--1.0.tar.Z}                                                \\
200 PORT command successful.                                                    \\
150 Opening BINARY mode data connection for Echo-1.0.tar.Z (925969 bytes).      \\
226 Transfer complete.                                                          \\
local: Echo--1.0.tar.Z remote: Echo--1.0.tar.Z                                  \\
925969 bytes received in 6.1e+02 seconds (5.2 Kbytes/s)                         \\
ftp$>$ {\sf quit}                                                               \\
\end{sl}

Now you have retrieved the entire distribution. The distribution
consists of a number of files and directories, which were archived
into a single file that was then compressed. The next job is to
reverse these steps to recover the original Echo files.  To do this:

\begin{shell}
\prompt\ uncompress Echo--1.0.tar.Z \\*
\prompt\ tar xf Echo--1.0.tar \\*
\end{shell}

This should result in the creation of a new directory, called {\bf
Echo--1.0}. Check that that directory has been created. If so, and you
have received no error messages, it is safe to remove the bundled
distribution file with

\begin{shell}
\prompt\ rm Echo-1.0.tar
\end{shell}

\newpage
\section{Environment Variables and X Resources}
\label{setting-up}

This appendix deals with setting up environment variables and X
resources for those who preferred not to have this done automatically
in section \ref{simple-setup}.

You may wish to automatically install just the X resources, or just
the shell's environment variables. Do this with either

\begin{shell}
\prompt\ make x--setup
\end{shell}

or

\begin{shell}
\prompt\ make sh--setup
\end{shell}

In both cases, the shell script that is invoked (either {\bf
echo--x--setup} or {\bf echo--sh-setup}) tries to find an appropriate
file to append some text to. It is fairly conservative and always
makes a backup copy of any file it alters (in a file whose name ends
with {\bf .bak}).

There are several reasons why you may choose not to have either of
these setups done automatically. If you are an experienced UNIX user,
your shell startup and X resource files are probably not something
you'll feel comfortable having someone else's shell script edit
automatically. In this case, you should be able to decide how you wish
to do what follows.

Here are more details about exactly what needs to be done:

\begin{itemize}

\item
The file {\bf Echo.ad} contains X window \index{X Windows} resource
\index{X Resources} specifications.
These set the various colors of the Echo interface, set the sizes and
locations of the pieces of the interface and so on. If you know what
all this means, you should put the contents of this file somewhere
that the X toolkit will find them when Echo starts.

\item
Echo looks for certain environment variables \index{Environment
Variables} that can be used to influence its behavior. None of these
are required, but at least one is highly recommended.

Echo \index{Echo Objects} comes with a collection of pre--defined
worlds, sites and agents which are located in the {\bf OBJECTS}
directory. If you have installed Echo in the directory {\bf
/usr/local/echo} then you should set an environment variable called
{\sc echo\_location} and give it the value

\begin{verse}
{\bf /usr/local/echo/OBJECTS}
\end{verse}

If you do not do this, you will always have to invoke Echo from the
directory where you installed it for it to see the {\bf OBJECTS}
directory. Eventually you may have your own directory of Echo objects
and you can change this variable to point to it.

The way to set the environment variable depends on the shell you are
using. If you are in csh, the simplest thing to do is to place the
line

\begin{shell}
setenv ECHO\_LOCATION /usr/local/echo/OBJECTS
\end{shell}

in your {\bf $\sim$/.login} file. Then, whenever you log in, the variable
will be automatically set for you. To make it affect the current login
session, you can also type the line at the shell's prompt.

If you are not using csh or a variant of it, you should place the
following lines in your {\bf $\sim$/.profile} file (or equivalent),

\begin{shell}
ECHO\_LOCATION=/usr/local/echo/OBJECTS \\*
export ECHO\_LOCATION \\*
\end{shell}

And you can either log out and in again or type those two lines to the
current shell to have them affect this session.

If you aren't sure what shell you are running, type

\begin{shell}
\prompt\ echo \$SHELL
\end{shell}

and if the output ends in the letters {\sl csh}, then use the first
method above. If not, use the second\footnote{If the shell name ends
in {\sl bash}, the lines should go into your {\bf
$\sim$/.bash\_profile} file if you have one and your {\bf
$\sim$/.profile} if not.}.

The other three environment variables are less important, but will
make Echo start in a more attractive fashion. The variables and their
recommended values are given below.  These settings will make Echo
display a world, a site and an agent when it starts up. Otherwise the
three editing areas will be blank.

These variables can be set in exactly the same way that you set the
variable {\sc echo\_location} above. If you use C shell or a variant
of it, the following goes into your {\bf $\sim$/.login}:

\begin{shell}
setenv ECHO\_WORLD insects \\*
setenv ECHO\_SITE insects \\*
setenv ECHO\_AGENT fly \\*
\end{shell}

Otherwise, the following goes into your {\bf $\sim$/.profile}:

\begin{shell}
ECHO\_WORLD=insects \\*
ECHO\_SITE=insects \\*
ECHO\_AGENT=fly \\*
export ECHO\_WORLD ECHO\_SITE ECHO\_AGENT \\*
\end{shell}

\end{itemize}

\newpage
\section{Emacs Keybindings}
\index{Emacs Keybindings}
\label{emacs-keybindings}

In this section an uppercase {\sf C} will be used to represent the use
of the Control key and an uppercase {\sf M} to represent the Meta
key. Thus, {\sf C--x} indicates that you should hold down the Control
key and while doing so, type an {\sf x}. The Meta key is used in an
identical fashion, e.g.  {\sf M--d} tells you to first hold down Meta
and while holding it, press the {\sf d} key.

Table \ref{emacs-keybindings-table} shows the most useful key bindings
present in all text windows.

\begin{table}
\begin{center}
\begin{tabular}{||l|l||l|l||}
\hline
{\sf C--a} & Beginning Of Line.         & {\sf M--b}   & Backward Word. \\
{\sf C--b} & Backward Character.        & {\sf M--d}   & Delete Next Word. \\
{\sf C--d} & Delete Next Character.     & {\sf M--f}   & Forward Word. \\
{\sf C--e} & End Of Line.               & {\sf M--i}   & Insert File. \\
{\sf C--f} & Forward Character.         & {\sf M--v}   & Scroll Backwards. \\
{\sf C--h} & Delete Previous Character. & {\sf M--$<$} & Beginning Of File. \\
{\sf C--k} & Kill To End Of Line.       & {\sf M--$>$} & End Of File. \\
{\sf C--l} & Redraw. & & \\
{\sf C--n} & Next Line. & & \\
{\sf C--p} & Previous Line. & & \\
{\sf C--r} & Search Backwards. & & \\
{\sf C--s} & Search Forwards. & & \\
{\sf C--t} & Transpose Characters. & & \\
{\sf C--v} & Scroll Forward. & & \\
{\sf C--w} & Delete Selected Region. & & \\
{\sf C--y} & Paste Deleted Region. & & \\
\hline
\end{tabular}
\end{center}
\caption{Useful Emacs Keybindings For Text Windows}
\label{emacs-keybindings-table}
\end{table}

\newpage
\section{Echo Widgets}
\index{Echo Widgets}
\label{echo-widgets}

If you are not installing Echo on a SPARC machine, you will need to
obtain and install various widget sets on your machine before you can
make Echo. These are all freely available via {\bf ftp}. If you are
not familiar with this sort of installation, it might be best to
consult your system administrator.

The widget packages all create libraries that Echo must be linked with
when it is compiled. The {\bf Makefile} definition of {\bf LIBS} in
the distribution assumes that these libraries can be found in the {\bf
WIDGETS} directory. In fact, these libraries can be installed anywhere
that is convenient, as long as the {\bf Makefile} is altered to
reflect their location.

The various widget libraries and their {\bf ftp} locations are as
follows:

\begin{itemize}

\item {\bf Athena Plotter Widgets} \\
Machine: {\bf ftp.uni--paderborn.de}, Location: {\bf /unix/tools}.

\item {\bf Free Widget Foundation Widgets} \\
Machine: {\bf ftp.let.rug.nl}, Location: {\bf /ftp/pub/FWF/fwf.tar.Z}.

\item {\bf 3D Athena Widgets} \\
Machine: {\bf ftp.x.org}, Location: {\bf /contrib/Xaw3d}.

\end{itemize}

Each of these widget sets comes with instructions on how to make and
install the libraries that support the various widgets. The FWF and
Xaw3d widgets are available for {\bf ftp} from many locations. Use
{\bf xarchie} to find other sites.

Once these libraries have been created and Echo's {\bf Makefile} knows
where to find them, you should be able to proceed with the Echo
installation itself.

I apologize if you have to go through this procedure. It is the result
of using the X toolkit and the need to find useful (free) widgets. I
hope this will be solved in later Echo versions by using {\bf TCL} --
though this will require that the machine receiving Echo has (or
obtains) the {\bf TCL} libraries...

\newpage
\begin{thebibliography}{99}

\bibitem {holland-92a}
Holland, John H. (1992a).
{\em The Echo Model},
in ``Proposal for a Research Program in Adaptive Computation''.
Santa Fe Institute, July 1992.

\bibitem {holland-92b}
Holland, John H. (1992b).
{\em Adaptation in Natural and Artificial Systems}, 2nd Ed.,
MIT Press, 1992.

\bibitem {holland-93}
Holland, John H. (1993).
{\em ECHOING EMERGENCE : Objectives, Rough Definitions, and
Speculations for Echo-class Models},
To appear in ``Integrative Themes'', George Cowan, David Pines and
David Melzner Eds. Santa Fe Institute Studies in the Sciences of
Complexity, Proc. Vol XIX. Reading, MA: Addison--Wesley 1993.

\end {thebibliography}

\documentstyle [12pt] {article}
% \makeindex

\begin{document}

\newenvironment{shell}{\begin{verse}\begin{sf}}{\end{sf}\end{verse}}
\newcommand{\prompt}{{\bf \%}}

\title {{\bf An Introduction to SFI Echo.}}

\author
{
Terry Jones
\thanks
{
Santa Fe Institute, 1660 Old Pecos Trail, Suite A.
Santa Fe NM 87501.
email: terry@santafe.edu
}
\and
Stephanie Forrest
\thanks
{
Dept. of Computer Science,
University of New Mexico.
Albuquerque NM  87131.
email: forrest@cs.unm.edu
}
}


\pagenumbering{empty}
\maketitle
\newpage

% \pagenumbering{empty}

\pagestyle{plain}
\pagenumbering{roman}
\tableofcontents
\newpage
\pagestyle{headings}
\pagenumbering{arabic}

\section{Introduction}

This report is concerned with an implementation of a family of models
of complex adaptive systems called Echo models. In what follows, you
will find:

\begin{itemize}

\item
An introduction to Echo.

\item
Information on how to obtain, install and run the Echo system.

\item
A description of Echo's graphical interface.

\item
Information on running Echo.

\end{itemize}

\subsection{Echo}

Echo is a model of complex adaptive systems formulated by John Holland
\cite{holland-92a,holland-92b,holland-93}. 
It abstracts away virtually all of the physical details of real
systems and concentrates on a small set of primitive agent--agent and
agent--environment interactions.  The extent to which Echo captures
the essence of real systems is still largely undetermined.  The goal
of Echo is to study how simple interactions among simple agents lead
to emergent high--level phenomena such as the flow of resources in a
system or cooperation and competition in networks of agents (e.g.,
communities, trading networks, or arms races).

An Echo world consists of a lattice of sites. Each is populated by
some number of agents, and there is a measure of locality within each
site.  Sites produce different types of renewable resources; each type
of resource is encoded by a letter (e.g., ``a,'' ``b,'' ``c,'' ``d'').
Different types of agents use different types of resources and can
store these resources internally. Sites charge agents a maintenance
fee or tax. This tax can also be thought of as metabolic cost.

Agents fight, trade and reproduce. Fighting and trading
result in the exchange of resources between agents. There is sexual
and non--sexual reproduction, sexual reproduction
results in offspring whose genomes are a combination of those
of the parents. Each agent's genome encodes various genes
which determine how it will interact with other agents (e.g., which
resource it is willing to trade, what sort of other agents it will
fight or trade with, etc.).  Some of these genes determine phenotypic
traits, or ``tags'' that are visible to other agents.  This
allows the possibility of the evolution of social rules and
potentially of mimicry, a phenomenon frequently observed in
natural ecosystems. The interaction rules rely only on string
matching.

Echo has no explicit fitness function guiding selection and
reproduction.  An agent self--reproduces when it accumulates a
sufficient quantity of each resource to make an exact copy of its
genome.  This cloning is subject to a low rate of mutation.

In preliminary simulations, the Echo system has demonstrated
surprisingly complex behavior (including something resembling a
biological ``arms race'' in which two competing agent types develop
progressively more complex offensive and defensive combat strategies),
ecological dependencies among different species, and sensitivity (in
terms of the number of different phenotypes) to differing levels of
renewable resources.

Ideally, Echo will allow the modeling of a diverse range of complex
adaptive systems without the need for a specialized model for each to
be developed. Typically, the people who know the most about any
particular real--world complex adaptive system are not the people who
can also develop sophisticated models that can be used as tools to
increase understanding. Echo aims to provide a useful modeling tool or
a starting point for the development of a model.

As a cautionary note, one must be a little careful when using the term
``Echo.''  Properly, Echo refers to a large family of models. As
described here, Echo will refer to the implementation developed at the
Santa Fe Institute.

Several versions of the system have been developed by Holland, and
there are significant differences between these. Echo has been
described in
\cite{holland-92a,holland-92b,holland-93}. These descriptions
represent snapshots of ongoing thought about Echo models. The version
implemented here is closest to that described in \cite{holland-92a}.
For further details, refer to the above sources.

\section{Getting Started}

\subsection{A Note On Fonts}
\index{Fonts}

The following conventions are used consistently throughout this
report:

\begin{itemize}

\item
\index{UNIX Commands}
File names and UNIX commands appear in {\bf bold face}. These will
normally be references to files and commands that you might need or
use, not things you will be expected to enter immediately.

\item
\index{UNIX Environment Variables}
Environment variables are shown in {\sc small caps} and are always
completely {\sc uppercase}.

\item
Commands you are expected to type or put in a file are shown in {\sf
sans serif}. This is true even if what you are asked to type is a file
name or an environment variable.

\item
Anything printed by the system appears in a {\sl slanted} font.

\item
Things you will see in the Echo interface are typeset in a {\sl
slanted} font, as for other system output, but, additionally, are
enclosed in ``{\sl double quotes}''. The text so enclosed is exactly
what you can expect to see in Echo's interface. For example, this font
will be used when discussing the various options you'll see in Echo's
popup menus.\index{Pop--up Menus}

\item
\index{UNIX Shell Prompt}
The UNIX shell prompt is a bold percent sign (\prompt) at the start
of a line.

\end{itemize}


\subsection{System Requirements}
\index{System Requirements}

Echo was developed on a SUN SPARC architecture running SUNOS 4.1.3. In
theory it will run on other BSD--based versions of UNIX, but may
require slight modification in some cases. The X Windows \index{X
Windows} system from MIT is required. Development was under X11R5, but
Echo should run without modification under X11R4. If you are using a
SPARC machine, no special widget sets \index{Widgets} are required,
the distribution file contains these. If not, you will need to {\bf
ftp} widgets from several locations -- this is discussed in appendix
\ref{echo-widgets} \index{Echo Widgets}.

\subsection{Porting Echo}
\index{Porting}

Unless a joint research agreement with the Santa Fe Institute exists,
there is currently no guaranteed way to get help porting Echo to a new
architecture. A possible source of help is through electronic mail to
{\bf echo@santafe.edu}. Assistance will probably be contingent on the
project being supported at SFI.

\subsection{Getting Echo Through Anonymous FTP}
\index{Using Ftp}

Echo is available via anonymous ftp from {\bf ftp.santafe.edu} in the
file

\begin{verse}
{\bf /pub/Users/terry/echo/Echo-1.0.tar.Z}
\end{verse}

The entire distribution occupies just over two megabytes of disk
space when uncompressed. You should transfer Echo, uncompress
\index{Uncompress} it and
extract the files using tar\index{Tar}.  If you are not familiar with
the workings of ftp, uncompress or tar, consult appendix
\ref{using-ftp}.


\subsection{Compiling Echo}

Once you have received and unpacked the distribution, you will need to
compile or ``make''\index{Making Echo} it. The {\bf make} command will
take care of the compilation and linking, but first you will need to
make some changes to the file {\bf Makefile}. These should be very
minor, the file contains instructions to help you. You should only
have to change two lines, to indicate where the X libraries and
include files are on your system. If you don't know, you can try to
find them or ask your system manager.

Once you have done this, you can simply type

\begin{shell}
\prompt\ make all
\end{shell}

and make will do the rest. You should see about twenty files get
compiled.  When the make is completed, you should have a file called
{\bf Echo} which is the executable program. This process also creates
you a shell script, {\bf run--echo} to make it simple to run the
executable in the recommended way.

\subsection{Preparing to Run Echo}

\label{simple-setup}

Two additional things need to be done before running Echo.

\begin{itemize}

\item
X resources for Echo need to be installed.

\item
Shell environment variables need to be set.

\end{itemize}

The distribution contains programs that will do this for you if do not
want to do it yourself, or do not know how. If you'd like it taken
care of for you, use

\begin{shell}
\prompt\ make simple--setup
\end{shell}

This will add X resource defaults to your X startup file and set
environment variables in your shell's startup file. If you have used
this option, you should now log out and then log back in, as these
changes will not affect your current login session\footnote{They will
be automatically set for you in all future login sessions.}.

If you'd prefer to do this kind of installation yourself, you should
see the details about what needs to be done in appendix \ref{setting-up}.


\subsection{Running Echo}
\index{Running Echo}
\label{running-echo}

Once you have completed the above, you should be able to run Echo.
Echo writes its textual output to the {\bf xterm} from which it was
invoked.  The X resources \index{X Resources} you just installed
specify the colors, sizes and locations of all the pieces of the Echo
interface.  These pieces are designed to fit around a central {\bf
xterm} window.

There are two ways you can arrange things so that an {\bf xterm} is
created in the right place (and the right color):

\begin{itemize}

\item
The simplest way is to use the shell script {\bf run--echo} that was
made for you when you did the original {\bf make}.  This will do all
the work of setting up an {\bf xterm}, invoking Echo, and making sure
the interface is configured correctly.

\item
Slightly more work, but simpler in the long run is to put Echo into a
popup menu.\index{Pop--up Menus} This is achieved by adding a line to
your window manager's initialization \index{Window Managers}
file. This file's name depends on your window manager. A likely
location is {\bf $\sim$/.twmrc}, or some other file in your home
directory whose name starts with a period ({\bf .}) and ends in the
string ``{\bf wmrc}''.  If you don't know where your window manager
initialization file is or can't find it, ask your system manager. The
file {\bf .window\_manager} in the Echo distribution contains a line
that can probably be usefully inserted in this file\footnote{{\bf
Warning}: it is not as simple as just adding the line to the end of
this file. If you are unsure about what you're doing, get help.}.

\end{itemize}

\subsection{A Sample Echo Run}
\index{Demo Run}
\index{Sample Run}
\index{Running Echo}

This section walks you through a quick Echo demo, without too much
explanation.

Once you have installed the X resources and the shell environment
variables, you are in a position to run Echo. Try it now -- either
from your window manager or with

\begin{shell}
\prompt\ ./run--echo
\end{shell}

You should see a number of windows appear on your screen. These are
explained in the next section. For now notice the main menu bar at the
top of the screen and the {\bf xterm} in the middle.

\begin{itemize}

\item
The first step in running a world is to choose the world you intend to
run. To do this, click on the red menu button at the top of the screen
that says ``{\sl Running}''. Hold the mouse button down and drag the
pointer down to ``{\sl Choose World}'' and let go.\index{Choosing A
World} \index{Pop--up Menus}

\item
A file selector will pop up. This allows you to walk through the
directory structure to find a file containing the specification of a
world you want to run. You should see ``insects'' in the scrollable
region on the right. Click on this word and the line it is on should
change to white on black. Now click on ``{\sl OK}'' to select the
world.

\item
Look at the ``{\sl Species}'' graph.\index{Species Levels Graph} It
now has a legend, showing ``{\sl ant}'', ``{\sl fly}'' and ``{\sl
cat}'' (``cat'' is short for caterpillar). The legend has appeared as
the world you selected contains a site that contains all of these
agent types.

\item
Now select ``{\sl Run 10 Generations}'' from the ``{\sl Running}''
menu.\index{Running Menu} You should see the two graphs update and the
{\bf xterm} will show the text: ``{\sl Run through generation 10.}''

\item
Let's edit the world and remove a few agents.\index{Editing The Stack}
\index{Stack Editing} Click on the ``{\sl Edit}'' menu button and drag
the pointer down to ``{\sl Stack}''. A window will appear showing you
the population of agents present at the site. This is called the
``stack'' at that site. It's a one--dimensional array of agents. Use
the left mouse button to move the small arrowed cursor to the start of
some line. Now hit Control--k twice on your keyboard. You should see
the line in the stack disappear -- and with it some unfortunate
agent. Control--k is an Emacs keybinding that kills the text until the
end of the line. See appendix
\ref{emacs-keybindings} for more information on the keybindings if
these are not familiar to you.

Now that you've removed the agent (or agents if you were feeling
enthusiastic), hit the ``{\sl Amen}'' button at the top of the stack
editor to make it so.

\item
Before restarting the world, choose ``{\sl Set Verbose Level...}''
from the ``{\sl Control}'' menu at the top of the
screen.\index{Verbose Level} Here you can type letters to see certain
types of output in the {\bf xterm} window.  In the small gray window
at the bottom of the panel that popped up, type the following
incantation: ``gsukbx''. You can see what output each of these letters
will produce by reading the text above the small text window. To type
in the text window you'll need to move the mouse into it. Now click on
the ``{\sl OK}'' button to dismiss the window.

\item
Choose ``{\sl Run 1 Generation}'' from the ``{\sl Running}''
menu.\index{Running Menu} You should see various output appear in the
{\bf xterm}. This will likely include information on the agents that
were taxed during the generation and those that went bankrupt. There
is a small chance you'll see a mutation or an agent killed in
combat. The end of the output will show a population summary.

\item
Now we'll run to generation 100. Choose ``{\sl Run Until...}'' from
the ``{\sl Running}'' menu.\index{Running Menu} In the small window
that pops up, enter 100 and then click on ``{\sl OK}''. You should see
the graphs rapidly update to generation 100. You'll also see a lot of
text flashing by in the {\bf xterm} window. Often it is good to turn
off all such output when you are about to run for a large number of
generations.

\item
Select ``{\sl Variant Levels}''\index{Variant Levels Graph} from the
``{\sl Graphs}'' menu. The graph will appear where the ``{\sl
Worldwide Population Level}'' graph was (actually it's just on top of
it). The top line (in red) shows the number of different genomes that
have ever existed in the world. The bottom (blue) line shows the
number that are currently alive. You can compare this number with the
worldwide population level, and it will probably be quite a lot
smaller. Obviously, some genomes have multiple copies -- presumably
because they are doing something right.

\item
Finally, choose ``{\sl Cluster Living Population}''\index{Cluster
Analysis} from the ``{\sl Examine}'' menu.\index{Examine Menu} You
will see a tree displayed in the {\bf xterm} window that indicates how
closely related the agents in the world are.  You will probably need
to resize (or scroll) your {\bf xterm} window to see the whole
tree. This tree will be explained in more detail later.

\item
To exit Echo, choose ``{\sl Exit}''\index{Exiting Echo} from the
``{\sl Control}'' menu.  Of course you can continue to play if you
like.

\end{itemize}

The next section gives more details about the kinds of things you just
saw and did.


\section{The Echo Interface}
\index{Echo Interface}

After starting Echo, you should see seven windows on your
screen. Starting in the center at the top and moving
counter--clockwise, these are the Control window, the World Editor,
the Site Editor, the Agent Editor, a graph showing species levels, a
graph showing the worldwide population level, and an {\bf xterm}
window.

\subsection{The Overall Look And Feel}
\index{Echo Look and Feel}

There are some high level features of the interface that have been
designed to make the look and feel consistent. These are:

\begin{itemize}

\item
\index{Mouse Buttons}
In general, the interface makes no distinction between the buttons on
your mouse (assuming you have more than a single button). You can use
any button you like to push command buttons, or use popup menus.
\index{Pop--up Menus} The only exception to this rule is described
in the next point.

\item
\index{Text Windows}
Some of the areas of the screen are a steely grey color. These are
text windows. This color is used consistently to indicate an area of
the screen where you may type characters. Text in these windows will
always be black. All you need do is move the cursor into the grey area
and start to type. Each of these is in fact a small editor with
Emacs--like keybindings\footnote{If you are not familiar with Emacs,
don't worry, you can move around with mouse button 1 and use the
delete key to get rid of things you don't want. Appendix
\ref{emacs-keybindings} gives a brief introduction to editing with
Emacs.}.

In a text window, the mouse may be used to copy and paste regions in
the same fashion as in normal {\bf xterm} windows. The first button, when
held down, can be used to drag out, and thus copy, a region (which
will be highlighted). The second button pastes the copied text into
the buffer at the cursor's location. The third button can be used to
extend the copied region. This is something you should experiment with
if you are not already familiar with this kind of mouse behavior.

In addition, rapid double and triple clicks with the left mouse button
can be used to select the word the cursor is on or the line the cursor
is on. A single click with the first button just moves the text cursor
to the location of the mouse.

A search window can be popped up by typing Control--S, and the
contents of a file can be read into the buffer using Meta--I, which
pops up a window to read the file name. See the Emacs keybindings in
appendix \ref{emacs-keybindings} for more information on operations in
text windows.

\item
\index{Read Only Output}
Blue areas where white text appears, such as the main output window,
are always read--only.

\item
\index{Command Buttons}
\index{Menu Bars}
Command and menu buttons always have a red background and white text.
These always indicate things you can click the mouse on to have some
action performed. Command buttons carry out an action immediately,
while menu buttons pop up a menu of options.  As mentioned above, no
distinction is made here between the buttons on your mouse -- use
whichever you like. \index{Mouse Buttons}

\item
\index{File Selectors}
There are a number of occasions when Echo will need to get a file name
from the user. To do this, it will display a file selector window.
This window has various components. You may enter the file's name in
the text window at the top of the file selector, and then click on
``{\sl OK}''. Often simpler, is to use the mouse to select the file
you want. The scrollable window labeled ``{\sl Directory Contents}''
contains the list of files in the currently scanned directory. Echo
tries to make sure that the default directory is a useful one (using
the {\sc echo\_location} environment variable).

You can select a file in the scrollable list by simply clicking on it
with the mouse. The second mouse button can be used to scroll the
list. If you click on the entry labeled ``{\sl ../}'' the file
selector will read and display the contents of the parent directory.
The ``{\sl Up}'' button can also be used to move up a level.  If you
click on an entry that is a directory, the file selector will read and
display that directory's contents. Directories can be identified by
the trailing ``{\sl /}'' after their name.

The ``{\sl OK}'' button selects the file and causes the file selector
to disappear. The ``{\sl Cancel}'' button dismisses the file selector
and cancels whatever function was called that made it appear
originally. 

\end{itemize}


\subsection{The Control Window}
\index{Control Window}

The control window consists of five pull down menus. These are labeled
``{\sl Control},'' ``{\sl Edit},'' ``{\sl Running},'' ``{\sl
Graphs},'' and ``{\sl Examine}.''

\begin{itemize}

\item
The ``{\sl Control}'' menu contains three options. The first, ``{\sl
Set Verbose Level...}'' \index{Verbose Level} pops up a window that
contains lines that briefly describe types of output that can be
displayed in the main {\bf xterm} window. At the start of each of
these lines is a key letter. In the small grey window at the bottom
you may enter the letters corresponding to the types of output you
want to see in the {\bf xterm}. This output will appear following each
generation, so you must run at least one generation to see anything.

The ``{\sl Show Seed}'' \index{Random Seeds} option of the ``{\sl
Control}'' menu prints the current random seed value into the {\bf
xterm}. This allows you access at any time to the seed that was used
to set up the current run.

The ``{\sl Exit}''\index{Exiting Echo} option does exactly that, it
exits Echo completely. It does not currently ask for confirmation, so
be careful!

\item
The ``{\sl Edit}'' \index{Editing Menu} menu has seven
options. Most of these are very similar.  ``{\sl Worlds''}
\index{Editing Worlds} will pop up
a file selector to let you choose a world to edit. The file selector
can be used to wander through a directory tree and select a file
containing the specifications of a world. Use the mouse to click on
the name of the file you want (or enter it in the top grey window) and
then click the ``{\sl OK''} button to select it. The world you choose
will appear in the World Editor window in the top left corner of the
screen (more on that soon).

Choosing ``{\sl New World}'' \index{Creating New Worlds} will clear the World
Editor window and let you enter the details of a fresh world.

The options ``{\sl Sites},'' \index{Editing Sites} ``{\sl New Site},''
\index{Creating New Sites} ``{\sl Agents},'' \index{Editing Agents}
and ``{\sl New Agent}'' \index{Creating New Agents} all behave in a
similar fashion. They can be used to prepare for editing the
characteristics of sites and agents in the Site Editor (middle left)
and Agent Editor (bottom left) windows.

The final option in the Editing menu is ``{\sl Stack.''}
\index{Editing The Stack} \index{Stack Editing} This allows
you to edit a site in a running world. Since we have not yet begun to
see what happens when a world is actually running, this operation will
be described later.

\item
The ``{\sl Running}'' \index{Running Menu} menu has nine options. The
first, ``{\sl Choose World}'' \index{Choosing A World} is used to tell
Echo which world you intend to run. This option must be used before an
Echo run can begin. It also pops up a file selector so that you can
choose a world.

The next four options, \index{Running Echo} ``{\sl Run
Indefinitely,}'' ``{\sl Run 1 Generation,}'' ``{\sl Run 10
Generations,}'' and ``{\sl Run Until...}''  are all to do with running
a world for a certain time. They should all be self explanatory. The
last will pop up a window so you can enter a stopping generation
number.

The next two options, ``{\sl Pause''} \index{Pausing A Run} and ``{\sl
Continue''} \index{Continuing A Paused Run} can be used to temporarily
halt and restart a run. If you have used one of the above running
options to run until generation 500 and suddenly decide you need to
turn off the output in the main text window, you can use ``{\sl
Pause}'', turn off the output and then ``{\sl Continue}'' to allow the
run to proceed to generation 500. These two options can be used in
this manner any time Echo is running an experiment.

The ``{\sl Replay''} \index{Replaying A Run} option resets the world
and re--initializes the random number generator so that the run can be
re--done exactly. This is very useful when you would like to try some
experiment and need to stop the world at an earlier point.

The ``{\sl Seed''} \index{Random Seeds} option can be used to enter
the random seed for a run.  The seed should be set {\em before} you
choose the world to run. It is important that you perform these two in
this order. If you do not specify a random seed for the run, one will
be chosen for you. The seed chosen for you is guaranteed to be unique,
and the random number generator has been highly scrutinized for
randomness. The file {\bf random.c} contains a blow--by--blow
description of the search for an acceptable generator.

\item
The ``{\sl Graphs}'' \label{graphs-menu} \index{Graphs Menu} menu can
be used to pop up any of five graphs. Initially there are two places
where graphs appear, and two of the five are shown by default. The
species level graph always appears on the left while the other four
are on its right. Of course, since these are normal X windows, you can
move them around and resize them as you wish.  The five graphs are:

\begin{itemize}
\item
The ``{\sl Species Levels}'' \index{Species Levels Graph} graph shows
a legend (after you choose a world) and plots the population level of
all the descendants of the original members of the ``species''
\index{Species}. It is not really correct to call these groups species
(lineage is more accurate), but I will not go into that here.

\item
The ``{\sl Worldwide Population Level}'' \index{Population Level
Graph} graph shows the total number of agents alive in the world.

\item
The ``{\sl World Resource Levels}'' \index{Resource Levels Graph}
graph shows how many of each resource exist (this is often very dull
viewing).

\item
The ``{\sl Schema Level}'' \index{Schema Levels Graph} graph allows you
to track the level of a schema in the population.

\item
The ``{\sl Variant Levels}'' \index{Variant Levels Graph} graph shows
the number of genomes that have ever existed and the number of genomes
that currently exist.

\end{itemize}

\item
The ``{\sl Examine}'' \index{Examine Menu} menu allows you to look at
some properties of the populations. The first option ``{\sl Choose
Schema to Graph...}''  \index{Graphing A Schema} \index{Choosing A
Schema To Graph} allows you to enter a regular expression
\index{Regular Expressions} 
corresponding to a schema you are interested in. The regular
expression is in UNIX--style and is a pattern of resources that might
be found on an agent's genome.  The simple details of these
expressions are explained in section
\ref{schema-tracking}.  More information on regular expressions can be
found in the UNIX manual page for {\bf egrep}.

The next two options, ``{\sl Cluster living population}'' and ``{\sl
Cluster all individuals}'' \index{Cluster Analysis} both perform
cluster analysis. You may choose to have either the current population
clustered or every genome that ever existed clustered. The output will
appear in the {\bf xterm} window, which should have a scrollbar in
case the output is too long. Clustering is explained in section
\ref{cluster-analysis}.

\end{itemize}

\subsection{The World Editor}
\index{World Editor}
\label{world-editor}

The ``{\sl World Editor}'' is used to make changes to the properties
of a world.  These changes do not affect the currently running
world. You can choose the world you wish to alter (or a new world)
from the ``{\sl Edit}'' menu in the ``{\sl Control}'' window. Worlds
have thirteen properties. Eleven of these are edited by entering their
values into the grey horizontal text windows to the right of their
brief descriptions:

\begin{itemize}

\item
Each world has a ``{\sl Name}''. \index{World Name} Usually this is
the same as the ``{\sl File Name}'' \index{World File Name} in which
the world is stored, but this need not be the case. The file name is
the name of the file in the ``{\sl OBJECTS/worlds}'' \index{Echo
Objects} directory pointed to by the {\sc echo\_location} environment
variable. There can only be one world per file, but many worlds
(stored in files with different names) may have the same name.

\item
Each world has some ``{\sl Number Of Resources}.'' \index{Number Of
Resources} \index{Echo Resources} The resources are named a, b,
c... depending on the number that exist. Typically this is a small
number, say three or four.

\item
The ``{\sl Rows}'' \index{Rows} and ``{\sl Columns}'' \index{Columns}
give the size of the two dimensional array of sites. \index{World
Size} \index{World Geography}

\item
The ``{\sl Trading Fraction}'' \index{Trading Fraction} determines how
much of the excess of a resource an agent gives away when it
trades. Each agent trades a particular resource, and when it gets
involved in a trading relationship with another agent, it uses this
fraction to decide how much to give away.  The excess \index{Trading
Excess} is defined to be the amount of the resource in question over
and above what the agent needs for self--replication purposes.

\item
The ``{\sl Interaction Fraction}'' \index{Interaction Fraction}
determines how many agent--agent interactions will take place each
time step. This number is multiplied by the population size to arrive
at a number of interactions. After this many interactions are
performed, the sites produce resources again and the other aspects of
an Echo cycle are executed.

\item
The ``{\sl Self Replication Fraction}'' \index{Self Replication
Fraction} determines how much of its extra resources a parent will
give to a child when self replicating. Self replication involves
making a copy of one's genome when enough resources have accrued in
the reservoirs. Once this happens, there may be extra resources, some
of which might be given to the child to ensure that it is not too weak
at birth.

\item
The ``{\sl Self Replication Threshold}'' \index{Self Replication
Threshold} determines how many copies of its genome an agent must be
capable of making before it actually makes a single one. Thus if this
value is set at 2, the agent must accumulate twice as many of each
resource in its reservoirs as it has in its genome. In this case, when
it does make a copy of itself, it will have an excess of each resource
equal to what it carries in its genome.  This extra can be divided
between the agent and the new child according to the Self Replication
Fraction described above.

\item
The ``{\sl Maintenance Probability}'' \index{Maintenance}
\index{Taxation} determines the frequency with
which sites charge a maintenance fee. This probability is used per
site per agent per Echo cycle.

\item
The ``{\sl Neighborhood}'' \index{Neighborhood} \index{Migration} can
be set to any of ``NONE'', ``EIGHT'' or ``NEWS'' to indicate how
agents can migrate in the world. The first should be clear, it
disables migration. The second means an agent can move to any of the
eight adjacent sites (assuming it is not in a corner or on an edge)
and the third allows only north, south, east or west moves.

\end{itemize}

In addition to these eleven properties, worlds have an array of {\bf
Sites} and a ``{\sl Combat Matrix}.'' \index{Combat Matrix} These can
both be edited by clicking the button at the top of the World Editor,
which will cause a window to pop up.  The sites window should contain
site file names (see below) in an array the size of the Rows and
Columns as specified in the world's properties above. The combat
matrix is a square array of side length the number of resources in the
world. Its actual use is somewhat complicated and will be described
more fully below under Combat.

\subsection{The Site Editor}
\index{Site Editor}
\label{site-editor}

Sites have ten properties, nine of which can be entered directly into
the Site Editor window in the horizontal grey rectangles:

\begin{itemize}

\item
As with worlds (and agents), sites have a ``{\sl Name}'' \index{Site
Name} and a ``{\sl File Name}'' \index{Site File Name} in which they
are stored. The actual location of the site file is in the ``{\sl
OBJECTS/sites}'' \index{Echo Objects} directory.

\item
Each site has a ``{\sl Mutation Probability}.'' \index{Mutation}
Mutation is performed in a genetic algorithm fashion. Each locus on
each genome is mutated with this probability at the end of each
cycle. The allele at a locus may ``mutate'' to the same allele value.

\item
The ``{\sl Crossover Probability}'' \index{Crossover}
\index{Recombination} determines the probability that
crossing over, or recombination, takes place when two agents reproduce
sexually. If recombination does not take place, the agents are left
untouched.

\item
The ``{\sl Random Death Probability}'' \index{Random Death} is the
probability that an agent is killed without cause at the end of a
cycle. This is typically set very low. After each cycle, every agent
is killed for no reason with this probability.

\item
The ``{\sl Production Function}'' \index{Resource Production}
determines how much of each resource the site produces at the end of
each cycle. These values should be specified separated by white
space. There should be as many of them as there are resources. The
same is true for the next three properties.

\item
The ``{\sl Initial Resource Levels}'' \index{Initial Site Resource
Levels} \index{Resource Levels Initially} are the resource levels that
the site is allocated when the world is initially created.

\item
The ``{\sl Maximums}'' \index{Maximum Resource Levels} \index{Resource
Levels Maximally} determine to what level each resource can grow if it
remains ``on the ground'' (i.e. not picked up by an agent) at a site.

\item
The ``{\sl Maintenance}'' \index{Maintenance} \index{Taxation} is the
tax charged by the site. Each agent is charged this tax after every
cycle according to the maintenance probability set for the world. This
probability is used to determine whether each agent individually gets
taxed, not whether the site will charge all agents if the probability
condition is met.

\end{itemize}

The final property of a site is its initial agent list. This can be
accessed by clicking the mouse on the ``{\sl Agents}'' \index{Site
Agent List} \index{Agents List At A Site} button at the top of the
site editor. A window will pop up. Each line in this window is used to
specify some number of agents. The agent names must be agent file
names. Each name may be followed by white space and a decimal number
indicating the number of agents of that type that should be created
contiguously at that location. The agents in this list form the agent
stack at that site.

\subsection{The Agent Editor}
\index{Agent Editor}
\label{agent-editor}

Agents have eleven properties:

\begin{itemize}

\item
As with worlds and sites, agents have both a ``{\sl Name}''
\index{Agent Name} and a
``{\sl File Name}.'' \index{Agent File Name} The file name simply
references a file in the directory ``{\sl OBJECTS/agents}.''
\index{Echo Objects}

\item
Each agent has a ``{\sl Trading Resource}.'' \index{Echo Resources}
\index{Trading Resources}  This is the resource that
the agent, initially, trades. This may be mutated in the course of a
run.

\item
Each agent is provided with some ``{\sl Initial Resources}.''
\index{Agent Resource Levels Initially} \index{Echo Resources}  This
specifies the resource levels in the agent's reservoir \index{Agent
Reservoirs} when it is created. There should be as many numbers here
as there are resources, each separated by white space.

\item
The ``{\sl Uptake Mask}'' \index{Uptake Masks} \index{Agent Uptake
Mask} determines what resources the agent is able to pick up directly
from the ground at the site. This should be a string of ``1'' or ``0''
characters, one for each of the resources. They should not be
separated by white space. A ``1'' indicates that the agent may pick up
this resource, and a ``0'' that it may not. This mask is subject to
mutation.

\item
The next six properties all specify tags \index{Tags} and conditions.
\index{Conditions} These are
used to determine with whom and how the agent interacts in the world.
They are described in detail elsewhere
\cite{holland-92a,holland-92b}. These ``genes'' can all grow and
shrink (even to zero length) under mutation. \index{Mutation}

\end{itemize}

\subsection{The Graphs}
\index{Graphs}

There is not much to say about the graphs. The individual graphs are
briefly described in section \ref{graphs-menu}. There is space
allocated for two of them, side by side at the bottom of the
screen. Only the species graph is ever displayed on the left. However,
since they are fully functional X windows, you can use your window
manager to resize and reposition them as you wish. The exit button on
each graph window simply closes the window. Graphs can be redisplayed
by selecting the graph in question from the Graph menu in the Control
window. The graphs (currently) update every two hundred generations
and there is no way to retrieve data once it moves off the left of the
graph (other than by replaying the world). This will hopefully be
changed sometime.

\subsection{Textual Output}
\index{X Windows}
\index{Text Windows}

Textual output appears in the {\bf xterm} window. This window should have a
scroll bar so you can examine lengthy output. The {\bf --sl} option to
{\bf xterm} can be used to set the number of lines that are saved off
the top of the screen for scrollback purposes. If you invoke Echo with
the supplied {\bf run--echo} script (or by adding the suggested line
to your window manager's startup file), the {\bf xterm} created will
have a scroll bar and will save two thousand lines of previous text.

\section{Creating A World}
\index{Creating A World}

The Echo distribution comes with several worlds, sites and agents in
the {\bf OBJECTS} directory. You should never need to directly edit
the files under this directory, the world, site and agent editors are
designed to read and write these for you.

These editors are described in sections \ref{world-editor},
\ref{site-editor} and \ref{agent-editor}.

To create a world (including its sites and their agents) from scratch,
choose ``{\sl New World}'' from the ``{\sl Edit}'' menu. This will
display a blank world editor. Fill in the details of your new world,
including a name and file name, and then ``{\sl Save}'' it. Notice
that you need to fill in the sites array. Click on the
``{\sl Sites}'' button to display a text window in which you enter
the site names (actually the site file names). This should be an array
that has as many rows and columns as you specify in the world editor.
For an example, take a look at the world {\bf 4x4--insects} in the
Echo distribution. This is a world with four rows and columns. Its
site array specifies the same site 16 times (the site is also called
{\bf 4x4--insects}.

Another way to create a new world is to copy an existing
one. \index{Copying Worlds} This is easily done. Suppose you wish to
make a copy of the {\bf insects} world. Read it into the world editor
and change its name and file name. Then make the other changes you
want and save the new world. The same principle applies to making
copies of sites and agents. \index{Copying Sites} \index{Copying
Agents}

Note that worlds refer to sites (in the ``{\sl Sites}'' text window of
the world editor) and that sites in turn refer to agents (in the
``{\sl Agents}'' text window of the site editor). This is obvious, but
the fact that it implies a connection between the three editors you'll
be using may not be so clear. If you are unclear about how to create
world, sites and agents, the best way to look is to examine the ones
in the Echo distribution.

If you don't own the Echo distribution files, you will not be able to
save your creations in the {\bf OBJECTS} directory that the
distribution came with. \index{Write Permissions} There is a simple
solution to this: simply copy the distribution's {\bf OBJECTS}
directory elsewhere, and change your {\sc echo\_location} environment
variable to indicate where your personal Echo objects are to be
found. For example, if the Echo distribution is located in {\bf
/usr/local/Echo}, you can create your own objects directory in your
home directory with

\begin{shell}
\prompt\ cd
\prompt\ cp --r /usr/local/Echo/OBJECTS Echo--objects
\end{shell}

and then change {\sc echo\_location} to be the {\bf Echo--objects}
directory in your home directory.

Of course, you don't need to copy the distribution's entire
{\bf OBJECTS} directory, you can just create your own. Echo expects
the directory specified in the {\sc echo\_location} variable to
contain three sub--directories, named {\bf worlds}, {\bf sites}, and
{\bf agents}.

\section{Running A World}
\index{Running A World}

\subsection{Choosing A World To Run}
\index{Choosing A World}

The first step in running a world is choosing which world to run.  To
do this, select the ``{\sl Choose World}'' option from the Running
menu in the Control window. This pops up a file selector showing the
available worlds. Once you choose a world, you can use the various
entries in the ``{\sl Running}'' menu to actually run it.

\subsection{Editing a Running World}
\index{Editing Running Worlds}

Although you cannot yet change attributes of sites and worlds while a
run is in progress, you can make changes to the agents at a site. If
you choose ``{\sl Stack}'' \index{Stack} in the Edit menu, a window
will pop up containing the agents present at the site (you will be
asked to enter site coordinates if you have multiple sites). In this
window you can directly edit the genome of any agent. You can search
for a particular string (use Control--s), you can read in a file of
agents from disk (use Meta--i) or you can use the editor to remove or
replicate some number of agents. Once you are done, you can use the
``{\sl Amen}'' command button to make your new site reality. The help
button will pop up a box describing the genome representation.

\subsection{Verbose Output}
\index{Output}
\index{Text Windows}

The ``{\sl Set Verbose Level...}'' \index{Verbose Level} option in the
``{\sl Control}'' menu \index{Control Menu} can be used to make
informative text appear in the X window. This has already been
described above. Interesting output includes that for the letters {\sf
g} (generation number), \index{Generation Number} {\sf s} (species
summary), \index{Species Summary} {\sf u} (details of mutations),
\index{Mutation} {\sf k} (who is killing whom), and {\sf d} (to see
extinct genomes). \index{Extinct Genomes} \index{Dead Agents}

\subsection{Cluster Analysis}
\index{Cluster Analysis}
\label{cluster-analysis}

It is possible to perform a cluster analysis based on the genetic
distance between the genomes. This can be done for the living genomes
or all those that have ever lived. The output will appear in tree
form in the {\bf xterm} window.

The clustering is done by calculating the genetic distance between all
pairs of agents. This distance is defined as the minimum number of
mutations needed to transform the genome of one agent into the genome
of the other. The two agents that are most closely related are grouped
into a ``cluster''. By also defining the distance from an agent to a
cluster and between separate clusters, it is simple to build a tree
showing how closely related the individuals in a population are. This
is done by successively merging the closest clusters (or individuals)
into a larger cluster until only one remains. This process naturally
defines a hierarchy of cluster relatedness which can be displayed as a
tree.  The clustering algorithm runs in $O(n^3 )$, where $n$ is the
number of individuals at the outset, so you may have to be patient if
you have a large population.


\subsection{Schema Tracking}
\index{Schema Tracking}
\label{schema-tracking}

While the world is not running, you can enter a schema to graph by
selecting that option in the ``{\sl Examine}'' \index{Graphing A
Schema} \index{Choosing A Schema To Graph} menu. Genes are separated
on the chromosome by an underscore (\_). The meta--characters
$\hat{\:}$ and
\$ can be used to tie the regular expression \index{Regular
Expressions} to the beginning and end of the genome
respectively. Square brackets ([\,]) can be used to denote a set of
characters any one of which constitutes a match. A star (*) represents
any number of the preceding expression and a plus (+) represents one
or more of the preceeding expression. All this is very standard
regular expression syntax, and this explanation is meant to be brief
at best.

As an example, we could look for agents that had the string ``aa''
somewhere in their mating tag\footnote{Assume a world with four
resources.}. This is matched by the expression \\
\noindent
$\hat{\:}$[abcd]*\_[abcd]*\_[abcd]*aa

which allows anything (including nothing) in the first two genes and
then anything (including nothing) followed by two a's in the third
gene. The range ``[abcd]'' could also have been represented with
``[a--d]''.

To see the level of this schema in the population, select the Schema
Level graph in the Graphs menu.

\newpage
\section*{Appendices}
\appendix

\section{Using FTP}
\index{Using Ftp}
\label{using-ftp}

The following illustrates how the file may be retrieved. Your UNIX
prompt is a percent sign. What the system prints is shown in a {\sl
slanted} font, and what you type, as usual, is in {\sf sans serif}.

\vspace{0.5in}
\noindent
\begin{sl}
\prompt\ {\sf ftp santafe.edu}                                                  \\
Connected to santafe.edu.                                                       \\
220-  * * * * * * * * * * * * * * * * * * * * * * * * * * * * * * * * * * * * * \\
220-                                                                            \\
220-   Anonymous access to the FTP area at SantaFe.edu is available:            \\
220-                                                                            \\
220-   ftp ftp.santafe.edu                                                      \\
220-   Login: anonymous                                                         \\
220-   Password: (Your email address)                                           \\
220-                                                                            \\
220-  * * * * * * * * * * * * * * * * * * * * * * * * * * * * * * * * * * * * * \\
220-                                                                            \\
220 sfi FTP server (Version 2.0WU(10) Mon Apr 12 10:49:51 MDT 1993) ready.      \\
Name (santafe.edu:terry): {\sf anonymous}                                       \\
331 Guest login ok, send your complete e-mail address as password.              \\
Password: {\sf Enter your email address.}                                       \\
230-                                                                            \\
230-  SFI FTP - SFI Anonymous FTP root directory.                               \\
230-                                                                            \\
230-  Directory: $\sim$ftp           Path: $\sim$ftp                            \\
230-                                                                            \\
230-    Welcome to the FTP area at SantaFe.edu...                               \\
230-                                                                            \\
230-    Everything useful is in the pub directory.  Type ``cd pub'' ...         \\
230-                                                                            \\
230-    If you have any questions or problems with this service,                \\
230-    please send email to $<$ftp@santafe.edu$>$.                             \\
230-                                                                            \\
230-Please read the file README.Z                                               \\
230-  it was last modified on Fri Mar 19 15:29:44 1993 - 179 days ago           \\
230 Guest login ok, access restrictions apply.                                  \\
ftp$>$ {\sf cd pub/Users/terry/echo}                                            \\
250 CWD command successful.                                                     \\
ftp$>$ {\sf binary}                                                             \\
200 Type set to I.                                                              \\
ftp$>$ {\sf get Echo--1.0.tar.Z}                                                \\
200 PORT command successful.                                                    \\
150 Opening BINARY mode data connection for Echo-1.0.tar.Z (925969 bytes).      \\
226 Transfer complete.                                                          \\
local: Echo--1.0.tar.Z remote: Echo--1.0.tar.Z                                  \\
925969 bytes received in 6.1e+02 seconds (5.2 Kbytes/s)                         \\
ftp$>$ {\sf quit}                                                               \\
\end{sl}

Now you have retrieved the entire distribution. The distribution
consists of a number of files and directories, which were archived
into a single file that was then compressed. The next job is to
reverse these steps to recover the original Echo files.  To do this:

\begin{shell}
\prompt\ uncompress Echo--1.0.tar.Z \\*
\prompt\ tar xf Echo--1.0.tar \\*
\end{shell}

This should result in the creation of a new directory, called {\bf
Echo--1.0}. Check that that directory has been created. If so, and you
have received no error messages, it is safe to remove the bundled
distribution file with

\begin{shell}
\prompt\ rm Echo-1.0.tar
\end{shell}

\newpage
\section{Environment Variables and X Resources}
\label{setting-up}

This appendix deals with setting up environment variables and X
resources for those who preferred not to have this done automatically
in section \ref{simple-setup}.

You may wish to automatically install just the X resources, or just
the shell's environment variables. Do this with either

\begin{shell}
\prompt\ make x--setup
\end{shell}

or

\begin{shell}
\prompt\ make sh--setup
\end{shell}

In both cases, the shell script that is invoked (either {\bf
echo--x--setup} or {\bf echo--sh-setup}) tries to find an appropriate
file to append some text to. It is fairly conservative and always
makes a backup copy of any file it alters (in a file whose name ends
with {\bf .bak}).

There are several reasons why you may choose not to have either of
these setups done automatically. If you are an experienced UNIX user,
your shell startup and X resource files are probably not something
you'll feel comfortable having someone else's shell script edit
automatically. In this case, you should be able to decide how you wish
to do what follows.

Here are more details about exactly what needs to be done:

\begin{itemize}

\item
The file {\bf Echo.ad} contains X window \index{X Windows} resource
\index{X Resources} specifications.
These set the various colors of the Echo interface, set the sizes and
locations of the pieces of the interface and so on. If you know what
all this means, you should put the contents of this file somewhere
that the X toolkit will find them when Echo starts.

\item
Echo looks for certain environment variables \index{Environment
Variables} that can be used to influence its behavior. None of these
are required, but at least one is highly recommended.

Echo \index{Echo Objects} comes with a collection of pre--defined
worlds, sites and agents which are located in the {\bf OBJECTS}
directory. If you have installed Echo in the directory {\bf
/usr/local/echo} then you should set an environment variable called
{\sc echo\_location} and give it the value

\begin{verse}
{\bf /usr/local/echo/OBJECTS}
\end{verse}

If you do not do this, you will always have to invoke Echo from the
directory where you installed it for it to see the {\bf OBJECTS}
directory. Eventually you may have your own directory of Echo objects
and you can change this variable to point to it.

The way to set the environment variable depends on the shell you are
using. If you are in csh, the simplest thing to do is to place the
line

\begin{shell}
setenv ECHO\_LOCATION /usr/local/echo/OBJECTS
\end{shell}

in your {\bf $\sim$/.login} file. Then, whenever you log in, the variable
will be automatically set for you. To make it affect the current login
session, you can also type the line at the shell's prompt.

If you are not using csh or a variant of it, you should place the
following lines in your {\bf $\sim$/.profile} file (or equivalent),

\begin{shell}
ECHO\_LOCATION=/usr/local/echo/OBJECTS \\*
export ECHO\_LOCATION \\*
\end{shell}

And you can either log out and in again or type those two lines to the
current shell to have them affect this session.

If you aren't sure what shell you are running, type

\begin{shell}
\prompt\ echo \$SHELL
\end{shell}

and if the output ends in the letters {\sl csh}, then use the first
method above. If not, use the second\footnote{If the shell name ends
in {\sl bash}, the lines should go into your {\bf
$\sim$/.bash\_profile} file if you have one and your {\bf
$\sim$/.profile} if not.}.

The other three environment variables are less important, but will
make Echo start in a more attractive fashion. The variables and their
recommended values are given below.  These settings will make Echo
display a world, a site and an agent when it starts up. Otherwise the
three editing areas will be blank.

These variables can be set in exactly the same way that you set the
variable {\sc echo\_location} above. If you use C shell or a variant
of it, the following goes into your {\bf $\sim$/.login}:

\begin{shell}
setenv ECHO\_WORLD insects \\*
setenv ECHO\_SITE insects \\*
setenv ECHO\_AGENT fly \\*
\end{shell}

Otherwise, the following goes into your {\bf $\sim$/.profile}:

\begin{shell}
ECHO\_WORLD=insects \\*
ECHO\_SITE=insects \\*
ECHO\_AGENT=fly \\*
export ECHO\_WORLD ECHO\_SITE ECHO\_AGENT \\*
\end{shell}

\end{itemize}

\newpage
\section{Emacs Keybindings}
\index{Emacs Keybindings}
\label{emacs-keybindings}

In this section an uppercase {\sf C} will be used to represent the use
of the Control key and an uppercase {\sf M} to represent the Meta
key. Thus, {\sf C--x} indicates that you should hold down the Control
key and while doing so, type an {\sf x}. The Meta key is used in an
identical fashion, e.g.  {\sf M--d} tells you to first hold down Meta
and while holding it, press the {\sf d} key.

Table \ref{emacs-keybindings-table} shows the most useful key bindings
present in all text windows.

\begin{table}
\begin{center}
\begin{tabular}{||l|l||l|l||}
\hline
{\sf C--a} & Beginning Of Line.         & {\sf M--b}   & Backward Word. \\
{\sf C--b} & Backward Character.        & {\sf M--d}   & Delete Next Word. \\
{\sf C--d} & Delete Next Character.     & {\sf M--f}   & Forward Word. \\
{\sf C--e} & End Of Line.               & {\sf M--i}   & Insert File. \\
{\sf C--f} & Forward Character.         & {\sf M--v}   & Scroll Backwards. \\
{\sf C--h} & Delete Previous Character. & {\sf M--$<$} & Beginning Of File. \\
{\sf C--k} & Kill To End Of Line.       & {\sf M--$>$} & End Of File. \\
{\sf C--l} & Redraw. & & \\
{\sf C--n} & Next Line. & & \\
{\sf C--p} & Previous Line. & & \\
{\sf C--r} & Search Backwards. & & \\
{\sf C--s} & Search Forwards. & & \\
{\sf C--t} & Transpose Characters. & & \\
{\sf C--v} & Scroll Forward. & & \\
{\sf C--w} & Delete Selected Region. & & \\
{\sf C--y} & Paste Deleted Region. & & \\
\hline
\end{tabular}
\end{center}
\caption{Useful Emacs Keybindings For Text Windows}
\label{emacs-keybindings-table}
\end{table}

\newpage
\section{Echo Widgets}
\index{Echo Widgets}
\label{echo-widgets}

If you are not installing Echo on a SPARC machine, you will need to
obtain and install various widget sets on your machine before you can
make Echo. These are all freely available via {\bf ftp}. If you are
not familiar with this sort of installation, it might be best to
consult your system administrator.

The widget packages all create libraries that Echo must be linked with
when it is compiled. The {\bf Makefile} definition of {\bf LIBS} in
the distribution assumes that these libraries can be found in the {\bf
WIDGETS} directory. In fact, these libraries can be installed anywhere
that is convenient, as long as the {\bf Makefile} is altered to
reflect their location.

The various widget libraries and their {\bf ftp} locations are as
follows:

\begin{itemize}

\item {\bf Athena Plotter Widgets} \\
Machine: {\bf ftp.uni--paderborn.de}, Location: {\bf /unix/tools}.

\item {\bf Free Widget Foundation Widgets} \\
Machine: {\bf ftp.let.rug.nl}, Location: {\bf /ftp/pub/FWF/fwf.tar.Z}.

\item {\bf 3D Athena Widgets} \\
Machine: {\bf ftp.x.org}, Location: {\bf /contrib/Xaw3d}.

\end{itemize}

Each of these widget sets comes with instructions on how to make and
install the libraries that support the various widgets. The FWF and
Xaw3d widgets are available for {\bf ftp} from many locations. Use
{\bf xarchie} to find other sites.

Once these libraries have been created and Echo's {\bf Makefile} knows
where to find them, you should be able to proceed with the Echo
installation itself.

I apologize if you have to go through this procedure. It is the result
of using the X toolkit and the need to find useful (free) widgets. I
hope this will be solved in later Echo versions by using {\bf TCL} --
though this will require that the machine receiving Echo has (or
obtains) the {\bf TCL} libraries...

\newpage
\begin{thebibliography}{99}

\bibitem {holland-92a}
Holland, John H. (1992a).
{\em The Echo Model},
in ``Proposal for a Research Program in Adaptive Computation''.
Santa Fe Institute, July 1992.

\bibitem {holland-92b}
Holland, John H. (1992b).
{\em Adaptation in Natural and Artificial Systems}, 2nd Ed.,
MIT Press, 1992.

\bibitem {holland-93}
Holland, John H. (1993).
{\em ECHOING EMERGENCE : Objectives, Rough Definitions, and
Speculations for Echo-class Models},
To appear in ``Integrative Themes'', George Cowan, David Pines and
David Melzner Eds. Santa Fe Institute Studies in the Sciences of
Complexity, Proc. Vol XIX. Reading, MA: Addison--Wesley 1993.

\end {thebibliography}

\documentstyle [12pt] {article}
% \makeindex

\begin{document}

\newenvironment{shell}{\begin{verse}\begin{sf}}{\end{sf}\end{verse}}
\newcommand{\prompt}{{\bf \%}}

\title {{\bf An Introduction to SFI Echo.}}

\author
{
Terry Jones
\thanks
{
Santa Fe Institute, 1660 Old Pecos Trail, Suite A.
Santa Fe NM 87501.
email: terry@santafe.edu
}
\and
Stephanie Forrest
\thanks
{
Dept. of Computer Science,
University of New Mexico.
Albuquerque NM  87131.
email: forrest@cs.unm.edu
}
}


\pagenumbering{empty}
\maketitle
\newpage

% \pagenumbering{empty}

\pagestyle{plain}
\pagenumbering{roman}
\tableofcontents
\newpage
\pagestyle{headings}
\pagenumbering{arabic}

\section{Introduction}

This report is concerned with an implementation of a family of models
of complex adaptive systems called Echo models. In what follows, you
will find:

\begin{itemize}

\item
An introduction to Echo.

\item
Information on how to obtain, install and run the Echo system.

\item
A description of Echo's graphical interface.

\item
Information on running Echo.

\end{itemize}

\subsection{Echo}

Echo is a model of complex adaptive systems formulated by John Holland
\cite{holland-92a,holland-92b,holland-93}. 
It abstracts away virtually all of the physical details of real
systems and concentrates on a small set of primitive agent--agent and
agent--environment interactions.  The extent to which Echo captures
the essence of real systems is still largely undetermined.  The goal
of Echo is to study how simple interactions among simple agents lead
to emergent high--level phenomena such as the flow of resources in a
system or cooperation and competition in networks of agents (e.g.,
communities, trading networks, or arms races).

An Echo world consists of a lattice of sites. Each is populated by
some number of agents, and there is a measure of locality within each
site.  Sites produce different types of renewable resources; each type
of resource is encoded by a letter (e.g., ``a,'' ``b,'' ``c,'' ``d'').
Different types of agents use different types of resources and can
store these resources internally. Sites charge agents a maintenance
fee or tax. This tax can also be thought of as metabolic cost.

Agents fight, trade and reproduce. Fighting and trading
result in the exchange of resources between agents. There is sexual
and non--sexual reproduction, sexual reproduction
results in offspring whose genomes are a combination of those
of the parents. Each agent's genome encodes various genes
which determine how it will interact with other agents (e.g., which
resource it is willing to trade, what sort of other agents it will
fight or trade with, etc.).  Some of these genes determine phenotypic
traits, or ``tags'' that are visible to other agents.  This
allows the possibility of the evolution of social rules and
potentially of mimicry, a phenomenon frequently observed in
natural ecosystems. The interaction rules rely only on string
matching.

Echo has no explicit fitness function guiding selection and
reproduction.  An agent self--reproduces when it accumulates a
sufficient quantity of each resource to make an exact copy of its
genome.  This cloning is subject to a low rate of mutation.

In preliminary simulations, the Echo system has demonstrated
surprisingly complex behavior (including something resembling a
biological ``arms race'' in which two competing agent types develop
progressively more complex offensive and defensive combat strategies),
ecological dependencies among different species, and sensitivity (in
terms of the number of different phenotypes) to differing levels of
renewable resources.

Ideally, Echo will allow the modeling of a diverse range of complex
adaptive systems without the need for a specialized model for each to
be developed. Typically, the people who know the most about any
particular real--world complex adaptive system are not the people who
can also develop sophisticated models that can be used as tools to
increase understanding. Echo aims to provide a useful modeling tool or
a starting point for the development of a model.

As a cautionary note, one must be a little careful when using the term
``Echo.''  Properly, Echo refers to a large family of models. As
described here, Echo will refer to the implementation developed at the
Santa Fe Institute.

Several versions of the system have been developed by Holland, and
there are significant differences between these. Echo has been
described in
\cite{holland-92a,holland-92b,holland-93}. These descriptions
represent snapshots of ongoing thought about Echo models. The version
implemented here is closest to that described in \cite{holland-92a}.
For further details, refer to the above sources.

\section{Getting Started}

\subsection{A Note On Fonts}
\index{Fonts}

The following conventions are used consistently throughout this
report:

\begin{itemize}

\item
\index{UNIX Commands}
File names and UNIX commands appear in {\bf bold face}. These will
normally be references to files and commands that you might need or
use, not things you will be expected to enter immediately.

\item
\index{UNIX Environment Variables}
Environment variables are shown in {\sc small caps} and are always
completely {\sc uppercase}.

\item
Commands you are expected to type or put in a file are shown in {\sf
sans serif}. This is true even if what you are asked to type is a file
name or an environment variable.

\item
Anything printed by the system appears in a {\sl slanted} font.

\item
Things you will see in the Echo interface are typeset in a {\sl
slanted} font, as for other system output, but, additionally, are
enclosed in ``{\sl double quotes}''. The text so enclosed is exactly
what you can expect to see in Echo's interface. For example, this font
will be used when discussing the various options you'll see in Echo's
popup menus.\index{Pop--up Menus}

\item
\index{UNIX Shell Prompt}
The UNIX shell prompt is a bold percent sign (\prompt) at the start
of a line.

\end{itemize}


\subsection{System Requirements}
\index{System Requirements}

Echo was developed on a SUN SPARC architecture running SUNOS 4.1.3. In
theory it will run on other BSD--based versions of UNIX, but may
require slight modification in some cases. The X Windows \index{X
Windows} system from MIT is required. Development was under X11R5, but
Echo should run without modification under X11R4. If you are using a
SPARC machine, no special widget sets \index{Widgets} are required,
the distribution file contains these. If not, you will need to {\bf
ftp} widgets from several locations -- this is discussed in appendix
\ref{echo-widgets} \index{Echo Widgets}.

\subsection{Porting Echo}
\index{Porting}

Unless a joint research agreement with the Santa Fe Institute exists,
there is currently no guaranteed way to get help porting Echo to a new
architecture. A possible source of help is through electronic mail to
{\bf echo@santafe.edu}. Assistance will probably be contingent on the
project being supported at SFI.

\subsection{Getting Echo Through Anonymous FTP}
\index{Using Ftp}

Echo is available via anonymous ftp from {\bf ftp.santafe.edu} in the
file

\begin{verse}
{\bf /pub/Users/terry/echo/Echo-1.0.tar.Z}
\end{verse}

The entire distribution occupies just over two megabytes of disk
space when uncompressed. You should transfer Echo, uncompress
\index{Uncompress} it and
extract the files using tar\index{Tar}.  If you are not familiar with
the workings of ftp, uncompress or tar, consult appendix
\ref{using-ftp}.


\subsection{Compiling Echo}

Once you have received and unpacked the distribution, you will need to
compile or ``make''\index{Making Echo} it. The {\bf make} command will
take care of the compilation and linking, but first you will need to
make some changes to the file {\bf Makefile}. These should be very
minor, the file contains instructions to help you. You should only
have to change two lines, to indicate where the X libraries and
include files are on your system. If you don't know, you can try to
find them or ask your system manager.

Once you have done this, you can simply type

\begin{shell}
\prompt\ make all
\end{shell}

and make will do the rest. You should see about twenty files get
compiled.  When the make is completed, you should have a file called
{\bf Echo} which is the executable program. This process also creates
you a shell script, {\bf run--echo} to make it simple to run the
executable in the recommended way.

\subsection{Preparing to Run Echo}

\label{simple-setup}

Two additional things need to be done before running Echo.

\begin{itemize}

\item
X resources for Echo need to be installed.

\item
Shell environment variables need to be set.

\end{itemize}

The distribution contains programs that will do this for you if do not
want to do it yourself, or do not know how. If you'd like it taken
care of for you, use

\begin{shell}
\prompt\ make simple--setup
\end{shell}

This will add X resource defaults to your X startup file and set
environment variables in your shell's startup file. If you have used
this option, you should now log out and then log back in, as these
changes will not affect your current login session\footnote{They will
be automatically set for you in all future login sessions.}.

If you'd prefer to do this kind of installation yourself, you should
see the details about what needs to be done in appendix \ref{setting-up}.


\subsection{Running Echo}
\index{Running Echo}
\label{running-echo}

Once you have completed the above, you should be able to run Echo.
Echo writes its textual output to the {\bf xterm} from which it was
invoked.  The X resources \index{X Resources} you just installed
specify the colors, sizes and locations of all the pieces of the Echo
interface.  These pieces are designed to fit around a central {\bf
xterm} window.

There are two ways you can arrange things so that an {\bf xterm} is
created in the right place (and the right color):

\begin{itemize}

\item
The simplest way is to use the shell script {\bf run--echo} that was
made for you when you did the original {\bf make}.  This will do all
the work of setting up an {\bf xterm}, invoking Echo, and making sure
the interface is configured correctly.

\item
Slightly more work, but simpler in the long run is to put Echo into a
popup menu.\index{Pop--up Menus} This is achieved by adding a line to
your window manager's initialization \index{Window Managers}
file. This file's name depends on your window manager. A likely
location is {\bf $\sim$/.twmrc}, or some other file in your home
directory whose name starts with a period ({\bf .}) and ends in the
string ``{\bf wmrc}''.  If you don't know where your window manager
initialization file is or can't find it, ask your system manager. The
file {\bf .window\_manager} in the Echo distribution contains a line
that can probably be usefully inserted in this file\footnote{{\bf
Warning}: it is not as simple as just adding the line to the end of
this file. If you are unsure about what you're doing, get help.}.

\end{itemize}

\subsection{A Sample Echo Run}
\index{Demo Run}
\index{Sample Run}
\index{Running Echo}

This section walks you through a quick Echo demo, without too much
explanation.

Once you have installed the X resources and the shell environment
variables, you are in a position to run Echo. Try it now -- either
from your window manager or with

\begin{shell}
\prompt\ ./run--echo
\end{shell}

You should see a number of windows appear on your screen. These are
explained in the next section. For now notice the main menu bar at the
top of the screen and the {\bf xterm} in the middle.

\begin{itemize}

\item
The first step in running a world is to choose the world you intend to
run. To do this, click on the red menu button at the top of the screen
that says ``{\sl Running}''. Hold the mouse button down and drag the
pointer down to ``{\sl Choose World}'' and let go.\index{Choosing A
World} \index{Pop--up Menus}

\item
A file selector will pop up. This allows you to walk through the
directory structure to find a file containing the specification of a
world you want to run. You should see ``insects'' in the scrollable
region on the right. Click on this word and the line it is on should
change to white on black. Now click on ``{\sl OK}'' to select the
world.

\item
Look at the ``{\sl Species}'' graph.\index{Species Levels Graph} It
now has a legend, showing ``{\sl ant}'', ``{\sl fly}'' and ``{\sl
cat}'' (``cat'' is short for caterpillar). The legend has appeared as
the world you selected contains a site that contains all of these
agent types.

\item
Now select ``{\sl Run 10 Generations}'' from the ``{\sl Running}''
menu.\index{Running Menu} You should see the two graphs update and the
{\bf xterm} will show the text: ``{\sl Run through generation 10.}''

\item
Let's edit the world and remove a few agents.\index{Editing The Stack}
\index{Stack Editing} Click on the ``{\sl Edit}'' menu button and drag
the pointer down to ``{\sl Stack}''. A window will appear showing you
the population of agents present at the site. This is called the
``stack'' at that site. It's a one--dimensional array of agents. Use
the left mouse button to move the small arrowed cursor to the start of
some line. Now hit Control--k twice on your keyboard. You should see
the line in the stack disappear -- and with it some unfortunate
agent. Control--k is an Emacs keybinding that kills the text until the
end of the line. See appendix
\ref{emacs-keybindings} for more information on the keybindings if
these are not familiar to you.

Now that you've removed the agent (or agents if you were feeling
enthusiastic), hit the ``{\sl Amen}'' button at the top of the stack
editor to make it so.

\item
Before restarting the world, choose ``{\sl Set Verbose Level...}''
from the ``{\sl Control}'' menu at the top of the
screen.\index{Verbose Level} Here you can type letters to see certain
types of output in the {\bf xterm} window.  In the small gray window
at the bottom of the panel that popped up, type the following
incantation: ``gsukbx''. You can see what output each of these letters
will produce by reading the text above the small text window. To type
in the text window you'll need to move the mouse into it. Now click on
the ``{\sl OK}'' button to dismiss the window.

\item
Choose ``{\sl Run 1 Generation}'' from the ``{\sl Running}''
menu.\index{Running Menu} You should see various output appear in the
{\bf xterm}. This will likely include information on the agents that
were taxed during the generation and those that went bankrupt. There
is a small chance you'll see a mutation or an agent killed in
combat. The end of the output will show a population summary.

\item
Now we'll run to generation 100. Choose ``{\sl Run Until...}'' from
the ``{\sl Running}'' menu.\index{Running Menu} In the small window
that pops up, enter 100 and then click on ``{\sl OK}''. You should see
the graphs rapidly update to generation 100. You'll also see a lot of
text flashing by in the {\bf xterm} window. Often it is good to turn
off all such output when you are about to run for a large number of
generations.

\item
Select ``{\sl Variant Levels}''\index{Variant Levels Graph} from the
``{\sl Graphs}'' menu. The graph will appear where the ``{\sl
Worldwide Population Level}'' graph was (actually it's just on top of
it). The top line (in red) shows the number of different genomes that
have ever existed in the world. The bottom (blue) line shows the
number that are currently alive. You can compare this number with the
worldwide population level, and it will probably be quite a lot
smaller. Obviously, some genomes have multiple copies -- presumably
because they are doing something right.

\item
Finally, choose ``{\sl Cluster Living Population}''\index{Cluster
Analysis} from the ``{\sl Examine}'' menu.\index{Examine Menu} You
will see a tree displayed in the {\bf xterm} window that indicates how
closely related the agents in the world are.  You will probably need
to resize (or scroll) your {\bf xterm} window to see the whole
tree. This tree will be explained in more detail later.

\item
To exit Echo, choose ``{\sl Exit}''\index{Exiting Echo} from the
``{\sl Control}'' menu.  Of course you can continue to play if you
like.

\end{itemize}

The next section gives more details about the kinds of things you just
saw and did.


\section{The Echo Interface}
\index{Echo Interface}

After starting Echo, you should see seven windows on your
screen. Starting in the center at the top and moving
counter--clockwise, these are the Control window, the World Editor,
the Site Editor, the Agent Editor, a graph showing species levels, a
graph showing the worldwide population level, and an {\bf xterm}
window.

\subsection{The Overall Look And Feel}
\index{Echo Look and Feel}

There are some high level features of the interface that have been
designed to make the look and feel consistent. These are:

\begin{itemize}

\item
\index{Mouse Buttons}
In general, the interface makes no distinction between the buttons on
your mouse (assuming you have more than a single button). You can use
any button you like to push command buttons, or use popup menus.
\index{Pop--up Menus} The only exception to this rule is described
in the next point.

\item
\index{Text Windows}
Some of the areas of the screen are a steely grey color. These are
text windows. This color is used consistently to indicate an area of
the screen where you may type characters. Text in these windows will
always be black. All you need do is move the cursor into the grey area
and start to type. Each of these is in fact a small editor with
Emacs--like keybindings\footnote{If you are not familiar with Emacs,
don't worry, you can move around with mouse button 1 and use the
delete key to get rid of things you don't want. Appendix
\ref{emacs-keybindings} gives a brief introduction to editing with
Emacs.}.

In a text window, the mouse may be used to copy and paste regions in
the same fashion as in normal {\bf xterm} windows. The first button, when
held down, can be used to drag out, and thus copy, a region (which
will be highlighted). The second button pastes the copied text into
the buffer at the cursor's location. The third button can be used to
extend the copied region. This is something you should experiment with
if you are not already familiar with this kind of mouse behavior.

In addition, rapid double and triple clicks with the left mouse button
can be used to select the word the cursor is on or the line the cursor
is on. A single click with the first button just moves the text cursor
to the location of the mouse.

A search window can be popped up by typing Control--S, and the
contents of a file can be read into the buffer using Meta--I, which
pops up a window to read the file name. See the Emacs keybindings in
appendix \ref{emacs-keybindings} for more information on operations in
text windows.

\item
\index{Read Only Output}
Blue areas where white text appears, such as the main output window,
are always read--only.

\item
\index{Command Buttons}
\index{Menu Bars}
Command and menu buttons always have a red background and white text.
These always indicate things you can click the mouse on to have some
action performed. Command buttons carry out an action immediately,
while menu buttons pop up a menu of options.  As mentioned above, no
distinction is made here between the buttons on your mouse -- use
whichever you like. \index{Mouse Buttons}

\item
\index{File Selectors}
There are a number of occasions when Echo will need to get a file name
from the user. To do this, it will display a file selector window.
This window has various components. You may enter the file's name in
the text window at the top of the file selector, and then click on
``{\sl OK}''. Often simpler, is to use the mouse to select the file
you want. The scrollable window labeled ``{\sl Directory Contents}''
contains the list of files in the currently scanned directory. Echo
tries to make sure that the default directory is a useful one (using
the {\sc echo\_location} environment variable).

You can select a file in the scrollable list by simply clicking on it
with the mouse. The second mouse button can be used to scroll the
list. If you click on the entry labeled ``{\sl ../}'' the file
selector will read and display the contents of the parent directory.
The ``{\sl Up}'' button can also be used to move up a level.  If you
click on an entry that is a directory, the file selector will read and
display that directory's contents. Directories can be identified by
the trailing ``{\sl /}'' after their name.

The ``{\sl OK}'' button selects the file and causes the file selector
to disappear. The ``{\sl Cancel}'' button dismisses the file selector
and cancels whatever function was called that made it appear
originally. 

\end{itemize}


\subsection{The Control Window}
\index{Control Window}

The control window consists of five pull down menus. These are labeled
``{\sl Control},'' ``{\sl Edit},'' ``{\sl Running},'' ``{\sl
Graphs},'' and ``{\sl Examine}.''

\begin{itemize}

\item
The ``{\sl Control}'' menu contains three options. The first, ``{\sl
Set Verbose Level...}'' \index{Verbose Level} pops up a window that
contains lines that briefly describe types of output that can be
displayed in the main {\bf xterm} window. At the start of each of
these lines is a key letter. In the small grey window at the bottom
you may enter the letters corresponding to the types of output you
want to see in the {\bf xterm}. This output will appear following each
generation, so you must run at least one generation to see anything.

The ``{\sl Show Seed}'' \index{Random Seeds} option of the ``{\sl
Control}'' menu prints the current random seed value into the {\bf
xterm}. This allows you access at any time to the seed that was used
to set up the current run.

The ``{\sl Exit}''\index{Exiting Echo} option does exactly that, it
exits Echo completely. It does not currently ask for confirmation, so
be careful!

\item
The ``{\sl Edit}'' \index{Editing Menu} menu has seven
options. Most of these are very similar.  ``{\sl Worlds''}
\index{Editing Worlds} will pop up
a file selector to let you choose a world to edit. The file selector
can be used to wander through a directory tree and select a file
containing the specifications of a world. Use the mouse to click on
the name of the file you want (or enter it in the top grey window) and
then click the ``{\sl OK''} button to select it. The world you choose
will appear in the World Editor window in the top left corner of the
screen (more on that soon).

Choosing ``{\sl New World}'' \index{Creating New Worlds} will clear the World
Editor window and let you enter the details of a fresh world.

The options ``{\sl Sites},'' \index{Editing Sites} ``{\sl New Site},''
\index{Creating New Sites} ``{\sl Agents},'' \index{Editing Agents}
and ``{\sl New Agent}'' \index{Creating New Agents} all behave in a
similar fashion. They can be used to prepare for editing the
characteristics of sites and agents in the Site Editor (middle left)
and Agent Editor (bottom left) windows.

The final option in the Editing menu is ``{\sl Stack.''}
\index{Editing The Stack} \index{Stack Editing} This allows
you to edit a site in a running world. Since we have not yet begun to
see what happens when a world is actually running, this operation will
be described later.

\item
The ``{\sl Running}'' \index{Running Menu} menu has nine options. The
first, ``{\sl Choose World}'' \index{Choosing A World} is used to tell
Echo which world you intend to run. This option must be used before an
Echo run can begin. It also pops up a file selector so that you can
choose a world.

The next four options, \index{Running Echo} ``{\sl Run
Indefinitely,}'' ``{\sl Run 1 Generation,}'' ``{\sl Run 10
Generations,}'' and ``{\sl Run Until...}''  are all to do with running
a world for a certain time. They should all be self explanatory. The
last will pop up a window so you can enter a stopping generation
number.

The next two options, ``{\sl Pause''} \index{Pausing A Run} and ``{\sl
Continue''} \index{Continuing A Paused Run} can be used to temporarily
halt and restart a run. If you have used one of the above running
options to run until generation 500 and suddenly decide you need to
turn off the output in the main text window, you can use ``{\sl
Pause}'', turn off the output and then ``{\sl Continue}'' to allow the
run to proceed to generation 500. These two options can be used in
this manner any time Echo is running an experiment.

The ``{\sl Replay''} \index{Replaying A Run} option resets the world
and re--initializes the random number generator so that the run can be
re--done exactly. This is very useful when you would like to try some
experiment and need to stop the world at an earlier point.

The ``{\sl Seed''} \index{Random Seeds} option can be used to enter
the random seed for a run.  The seed should be set {\em before} you
choose the world to run. It is important that you perform these two in
this order. If you do not specify a random seed for the run, one will
be chosen for you. The seed chosen for you is guaranteed to be unique,
and the random number generator has been highly scrutinized for
randomness. The file {\bf random.c} contains a blow--by--blow
description of the search for an acceptable generator.

\item
The ``{\sl Graphs}'' \label{graphs-menu} \index{Graphs Menu} menu can
be used to pop up any of five graphs. Initially there are two places
where graphs appear, and two of the five are shown by default. The
species level graph always appears on the left while the other four
are on its right. Of course, since these are normal X windows, you can
move them around and resize them as you wish.  The five graphs are:

\begin{itemize}
\item
The ``{\sl Species Levels}'' \index{Species Levels Graph} graph shows
a legend (after you choose a world) and plots the population level of
all the descendants of the original members of the ``species''
\index{Species}. It is not really correct to call these groups species
(lineage is more accurate), but I will not go into that here.

\item
The ``{\sl Worldwide Population Level}'' \index{Population Level
Graph} graph shows the total number of agents alive in the world.

\item
The ``{\sl World Resource Levels}'' \index{Resource Levels Graph}
graph shows how many of each resource exist (this is often very dull
viewing).

\item
The ``{\sl Schema Level}'' \index{Schema Levels Graph} graph allows you
to track the level of a schema in the population.

\item
The ``{\sl Variant Levels}'' \index{Variant Levels Graph} graph shows
the number of genomes that have ever existed and the number of genomes
that currently exist.

\end{itemize}

\item
The ``{\sl Examine}'' \index{Examine Menu} menu allows you to look at
some properties of the populations. The first option ``{\sl Choose
Schema to Graph...}''  \index{Graphing A Schema} \index{Choosing A
Schema To Graph} allows you to enter a regular expression
\index{Regular Expressions} 
corresponding to a schema you are interested in. The regular
expression is in UNIX--style and is a pattern of resources that might
be found on an agent's genome.  The simple details of these
expressions are explained in section
\ref{schema-tracking}.  More information on regular expressions can be
found in the UNIX manual page for {\bf egrep}.

The next two options, ``{\sl Cluster living population}'' and ``{\sl
Cluster all individuals}'' \index{Cluster Analysis} both perform
cluster analysis. You may choose to have either the current population
clustered or every genome that ever existed clustered. The output will
appear in the {\bf xterm} window, which should have a scrollbar in
case the output is too long. Clustering is explained in section
\ref{cluster-analysis}.

\end{itemize}

\subsection{The World Editor}
\index{World Editor}
\label{world-editor}

The ``{\sl World Editor}'' is used to make changes to the properties
of a world.  These changes do not affect the currently running
world. You can choose the world you wish to alter (or a new world)
from the ``{\sl Edit}'' menu in the ``{\sl Control}'' window. Worlds
have thirteen properties. Eleven of these are edited by entering their
values into the grey horizontal text windows to the right of their
brief descriptions:

\begin{itemize}

\item
Each world has a ``{\sl Name}''. \index{World Name} Usually this is
the same as the ``{\sl File Name}'' \index{World File Name} in which
the world is stored, but this need not be the case. The file name is
the name of the file in the ``{\sl OBJECTS/worlds}'' \index{Echo
Objects} directory pointed to by the {\sc echo\_location} environment
variable. There can only be one world per file, but many worlds
(stored in files with different names) may have the same name.

\item
Each world has some ``{\sl Number Of Resources}.'' \index{Number Of
Resources} \index{Echo Resources} The resources are named a, b,
c... depending on the number that exist. Typically this is a small
number, say three or four.

\item
The ``{\sl Rows}'' \index{Rows} and ``{\sl Columns}'' \index{Columns}
give the size of the two dimensional array of sites. \index{World
Size} \index{World Geography}

\item
The ``{\sl Trading Fraction}'' \index{Trading Fraction} determines how
much of the excess of a resource an agent gives away when it
trades. Each agent trades a particular resource, and when it gets
involved in a trading relationship with another agent, it uses this
fraction to decide how much to give away.  The excess \index{Trading
Excess} is defined to be the amount of the resource in question over
and above what the agent needs for self--replication purposes.

\item
The ``{\sl Interaction Fraction}'' \index{Interaction Fraction}
determines how many agent--agent interactions will take place each
time step. This number is multiplied by the population size to arrive
at a number of interactions. After this many interactions are
performed, the sites produce resources again and the other aspects of
an Echo cycle are executed.

\item
The ``{\sl Self Replication Fraction}'' \index{Self Replication
Fraction} determines how much of its extra resources a parent will
give to a child when self replicating. Self replication involves
making a copy of one's genome when enough resources have accrued in
the reservoirs. Once this happens, there may be extra resources, some
of which might be given to the child to ensure that it is not too weak
at birth.

\item
The ``{\sl Self Replication Threshold}'' \index{Self Replication
Threshold} determines how many copies of its genome an agent must be
capable of making before it actually makes a single one. Thus if this
value is set at 2, the agent must accumulate twice as many of each
resource in its reservoirs as it has in its genome. In this case, when
it does make a copy of itself, it will have an excess of each resource
equal to what it carries in its genome.  This extra can be divided
between the agent and the new child according to the Self Replication
Fraction described above.

\item
The ``{\sl Maintenance Probability}'' \index{Maintenance}
\index{Taxation} determines the frequency with
which sites charge a maintenance fee. This probability is used per
site per agent per Echo cycle.

\item
The ``{\sl Neighborhood}'' \index{Neighborhood} \index{Migration} can
be set to any of ``NONE'', ``EIGHT'' or ``NEWS'' to indicate how
agents can migrate in the world. The first should be clear, it
disables migration. The second means an agent can move to any of the
eight adjacent sites (assuming it is not in a corner or on an edge)
and the third allows only north, south, east or west moves.

\end{itemize}

In addition to these eleven properties, worlds have an array of {\bf
Sites} and a ``{\sl Combat Matrix}.'' \index{Combat Matrix} These can
both be edited by clicking the button at the top of the World Editor,
which will cause a window to pop up.  The sites window should contain
site file names (see below) in an array the size of the Rows and
Columns as specified in the world's properties above. The combat
matrix is a square array of side length the number of resources in the
world. Its actual use is somewhat complicated and will be described
more fully below under Combat.

\subsection{The Site Editor}
\index{Site Editor}
\label{site-editor}

Sites have ten properties, nine of which can be entered directly into
the Site Editor window in the horizontal grey rectangles:

\begin{itemize}

\item
As with worlds (and agents), sites have a ``{\sl Name}'' \index{Site
Name} and a ``{\sl File Name}'' \index{Site File Name} in which they
are stored. The actual location of the site file is in the ``{\sl
OBJECTS/sites}'' \index{Echo Objects} directory.

\item
Each site has a ``{\sl Mutation Probability}.'' \index{Mutation}
Mutation is performed in a genetic algorithm fashion. Each locus on
each genome is mutated with this probability at the end of each
cycle. The allele at a locus may ``mutate'' to the same allele value.

\item
The ``{\sl Crossover Probability}'' \index{Crossover}
\index{Recombination} determines the probability that
crossing over, or recombination, takes place when two agents reproduce
sexually. If recombination does not take place, the agents are left
untouched.

\item
The ``{\sl Random Death Probability}'' \index{Random Death} is the
probability that an agent is killed without cause at the end of a
cycle. This is typically set very low. After each cycle, every agent
is killed for no reason with this probability.

\item
The ``{\sl Production Function}'' \index{Resource Production}
determines how much of each resource the site produces at the end of
each cycle. These values should be specified separated by white
space. There should be as many of them as there are resources. The
same is true for the next three properties.

\item
The ``{\sl Initial Resource Levels}'' \index{Initial Site Resource
Levels} \index{Resource Levels Initially} are the resource levels that
the site is allocated when the world is initially created.

\item
The ``{\sl Maximums}'' \index{Maximum Resource Levels} \index{Resource
Levels Maximally} determine to what level each resource can grow if it
remains ``on the ground'' (i.e. not picked up by an agent) at a site.

\item
The ``{\sl Maintenance}'' \index{Maintenance} \index{Taxation} is the
tax charged by the site. Each agent is charged this tax after every
cycle according to the maintenance probability set for the world. This
probability is used to determine whether each agent individually gets
taxed, not whether the site will charge all agents if the probability
condition is met.

\end{itemize}

The final property of a site is its initial agent list. This can be
accessed by clicking the mouse on the ``{\sl Agents}'' \index{Site
Agent List} \index{Agents List At A Site} button at the top of the
site editor. A window will pop up. Each line in this window is used to
specify some number of agents. The agent names must be agent file
names. Each name may be followed by white space and a decimal number
indicating the number of agents of that type that should be created
contiguously at that location. The agents in this list form the agent
stack at that site.

\subsection{The Agent Editor}
\index{Agent Editor}
\label{agent-editor}

Agents have eleven properties:

\begin{itemize}

\item
As with worlds and sites, agents have both a ``{\sl Name}''
\index{Agent Name} and a
``{\sl File Name}.'' \index{Agent File Name} The file name simply
references a file in the directory ``{\sl OBJECTS/agents}.''
\index{Echo Objects}

\item
Each agent has a ``{\sl Trading Resource}.'' \index{Echo Resources}
\index{Trading Resources}  This is the resource that
the agent, initially, trades. This may be mutated in the course of a
run.

\item
Each agent is provided with some ``{\sl Initial Resources}.''
\index{Agent Resource Levels Initially} \index{Echo Resources}  This
specifies the resource levels in the agent's reservoir \index{Agent
Reservoirs} when it is created. There should be as many numbers here
as there are resources, each separated by white space.

\item
The ``{\sl Uptake Mask}'' \index{Uptake Masks} \index{Agent Uptake
Mask} determines what resources the agent is able to pick up directly
from the ground at the site. This should be a string of ``1'' or ``0''
characters, one for each of the resources. They should not be
separated by white space. A ``1'' indicates that the agent may pick up
this resource, and a ``0'' that it may not. This mask is subject to
mutation.

\item
The next six properties all specify tags \index{Tags} and conditions.
\index{Conditions} These are
used to determine with whom and how the agent interacts in the world.
They are described in detail elsewhere
\cite{holland-92a,holland-92b}. These ``genes'' can all grow and
shrink (even to zero length) under mutation. \index{Mutation}

\end{itemize}

\subsection{The Graphs}
\index{Graphs}

There is not much to say about the graphs. The individual graphs are
briefly described in section \ref{graphs-menu}. There is space
allocated for two of them, side by side at the bottom of the
screen. Only the species graph is ever displayed on the left. However,
since they are fully functional X windows, you can use your window
manager to resize and reposition them as you wish. The exit button on
each graph window simply closes the window. Graphs can be redisplayed
by selecting the graph in question from the Graph menu in the Control
window. The graphs (currently) update every two hundred generations
and there is no way to retrieve data once it moves off the left of the
graph (other than by replaying the world). This will hopefully be
changed sometime.

\subsection{Textual Output}
\index{X Windows}
\index{Text Windows}

Textual output appears in the {\bf xterm} window. This window should have a
scroll bar so you can examine lengthy output. The {\bf --sl} option to
{\bf xterm} can be used to set the number of lines that are saved off
the top of the screen for scrollback purposes. If you invoke Echo with
the supplied {\bf run--echo} script (or by adding the suggested line
to your window manager's startup file), the {\bf xterm} created will
have a scroll bar and will save two thousand lines of previous text.

\section{Creating A World}
\index{Creating A World}

The Echo distribution comes with several worlds, sites and agents in
the {\bf OBJECTS} directory. You should never need to directly edit
the files under this directory, the world, site and agent editors are
designed to read and write these for you.

These editors are described in sections \ref{world-editor},
\ref{site-editor} and \ref{agent-editor}.

To create a world (including its sites and their agents) from scratch,
choose ``{\sl New World}'' from the ``{\sl Edit}'' menu. This will
display a blank world editor. Fill in the details of your new world,
including a name and file name, and then ``{\sl Save}'' it. Notice
that you need to fill in the sites array. Click on the
``{\sl Sites}'' button to display a text window in which you enter
the site names (actually the site file names). This should be an array
that has as many rows and columns as you specify in the world editor.
For an example, take a look at the world {\bf 4x4--insects} in the
Echo distribution. This is a world with four rows and columns. Its
site array specifies the same site 16 times (the site is also called
{\bf 4x4--insects}.

Another way to create a new world is to copy an existing
one. \index{Copying Worlds} This is easily done. Suppose you wish to
make a copy of the {\bf insects} world. Read it into the world editor
and change its name and file name. Then make the other changes you
want and save the new world. The same principle applies to making
copies of sites and agents. \index{Copying Sites} \index{Copying
Agents}

Note that worlds refer to sites (in the ``{\sl Sites}'' text window of
the world editor) and that sites in turn refer to agents (in the
``{\sl Agents}'' text window of the site editor). This is obvious, but
the fact that it implies a connection between the three editors you'll
be using may not be so clear. If you are unclear about how to create
world, sites and agents, the best way to look is to examine the ones
in the Echo distribution.

If you don't own the Echo distribution files, you will not be able to
save your creations in the {\bf OBJECTS} directory that the
distribution came with. \index{Write Permissions} There is a simple
solution to this: simply copy the distribution's {\bf OBJECTS}
directory elsewhere, and change your {\sc echo\_location} environment
variable to indicate where your personal Echo objects are to be
found. For example, if the Echo distribution is located in {\bf
/usr/local/Echo}, you can create your own objects directory in your
home directory with

\begin{shell}
\prompt\ cd
\prompt\ cp --r /usr/local/Echo/OBJECTS Echo--objects
\end{shell}

and then change {\sc echo\_location} to be the {\bf Echo--objects}
directory in your home directory.

Of course, you don't need to copy the distribution's entire
{\bf OBJECTS} directory, you can just create your own. Echo expects
the directory specified in the {\sc echo\_location} variable to
contain three sub--directories, named {\bf worlds}, {\bf sites}, and
{\bf agents}.

\section{Running A World}
\index{Running A World}

\subsection{Choosing A World To Run}
\index{Choosing A World}

The first step in running a world is choosing which world to run.  To
do this, select the ``{\sl Choose World}'' option from the Running
menu in the Control window. This pops up a file selector showing the
available worlds. Once you choose a world, you can use the various
entries in the ``{\sl Running}'' menu to actually run it.

\subsection{Editing a Running World}
\index{Editing Running Worlds}

Although you cannot yet change attributes of sites and worlds while a
run is in progress, you can make changes to the agents at a site. If
you choose ``{\sl Stack}'' \index{Stack} in the Edit menu, a window
will pop up containing the agents present at the site (you will be
asked to enter site coordinates if you have multiple sites). In this
window you can directly edit the genome of any agent. You can search
for a particular string (use Control--s), you can read in a file of
agents from disk (use Meta--i) or you can use the editor to remove or
replicate some number of agents. Once you are done, you can use the
``{\sl Amen}'' command button to make your new site reality. The help
button will pop up a box describing the genome representation.

\subsection{Verbose Output}
\index{Output}
\index{Text Windows}

The ``{\sl Set Verbose Level...}'' \index{Verbose Level} option in the
``{\sl Control}'' menu \index{Control Menu} can be used to make
informative text appear in the X window. This has already been
described above. Interesting output includes that for the letters {\sf
g} (generation number), \index{Generation Number} {\sf s} (species
summary), \index{Species Summary} {\sf u} (details of mutations),
\index{Mutation} {\sf k} (who is killing whom), and {\sf d} (to see
extinct genomes). \index{Extinct Genomes} \index{Dead Agents}

\subsection{Cluster Analysis}
\index{Cluster Analysis}
\label{cluster-analysis}

It is possible to perform a cluster analysis based on the genetic
distance between the genomes. This can be done for the living genomes
or all those that have ever lived. The output will appear in tree
form in the {\bf xterm} window.

The clustering is done by calculating the genetic distance between all
pairs of agents. This distance is defined as the minimum number of
mutations needed to transform the genome of one agent into the genome
of the other. The two agents that are most closely related are grouped
into a ``cluster''. By also defining the distance from an agent to a
cluster and between separate clusters, it is simple to build a tree
showing how closely related the individuals in a population are. This
is done by successively merging the closest clusters (or individuals)
into a larger cluster until only one remains. This process naturally
defines a hierarchy of cluster relatedness which can be displayed as a
tree.  The clustering algorithm runs in $O(n^3 )$, where $n$ is the
number of individuals at the outset, so you may have to be patient if
you have a large population.


\subsection{Schema Tracking}
\index{Schema Tracking}
\label{schema-tracking}

While the world is not running, you can enter a schema to graph by
selecting that option in the ``{\sl Examine}'' \index{Graphing A
Schema} \index{Choosing A Schema To Graph} menu. Genes are separated
on the chromosome by an underscore (\_). The meta--characters
$\hat{\:}$ and
\$ can be used to tie the regular expression \index{Regular
Expressions} to the beginning and end of the genome
respectively. Square brackets ([\,]) can be used to denote a set of
characters any one of which constitutes a match. A star (*) represents
any number of the preceding expression and a plus (+) represents one
or more of the preceeding expression. All this is very standard
regular expression syntax, and this explanation is meant to be brief
at best.

As an example, we could look for agents that had the string ``aa''
somewhere in their mating tag\footnote{Assume a world with four
resources.}. This is matched by the expression \\
\noindent
$\hat{\:}$[abcd]*\_[abcd]*\_[abcd]*aa

which allows anything (including nothing) in the first two genes and
then anything (including nothing) followed by two a's in the third
gene. The range ``[abcd]'' could also have been represented with
``[a--d]''.

To see the level of this schema in the population, select the Schema
Level graph in the Graphs menu.

\newpage
\section*{Appendices}
\appendix

\section{Using FTP}
\index{Using Ftp}
\label{using-ftp}

The following illustrates how the file may be retrieved. Your UNIX
prompt is a percent sign. What the system prints is shown in a {\sl
slanted} font, and what you type, as usual, is in {\sf sans serif}.

\vspace{0.5in}
\noindent
\begin{sl}
\prompt\ {\sf ftp santafe.edu}                                                  \\
Connected to santafe.edu.                                                       \\
220-  * * * * * * * * * * * * * * * * * * * * * * * * * * * * * * * * * * * * * \\
220-                                                                            \\
220-   Anonymous access to the FTP area at SantaFe.edu is available:            \\
220-                                                                            \\
220-   ftp ftp.santafe.edu                                                      \\
220-   Login: anonymous                                                         \\
220-   Password: (Your email address)                                           \\
220-                                                                            \\
220-  * * * * * * * * * * * * * * * * * * * * * * * * * * * * * * * * * * * * * \\
220-                                                                            \\
220 sfi FTP server (Version 2.0WU(10) Mon Apr 12 10:49:51 MDT 1993) ready.      \\
Name (santafe.edu:terry): {\sf anonymous}                                       \\
331 Guest login ok, send your complete e-mail address as password.              \\
Password: {\sf Enter your email address.}                                       \\
230-                                                                            \\
230-  SFI FTP - SFI Anonymous FTP root directory.                               \\
230-                                                                            \\
230-  Directory: $\sim$ftp           Path: $\sim$ftp                            \\
230-                                                                            \\
230-    Welcome to the FTP area at SantaFe.edu...                               \\
230-                                                                            \\
230-    Everything useful is in the pub directory.  Type ``cd pub'' ...         \\
230-                                                                            \\
230-    If you have any questions or problems with this service,                \\
230-    please send email to $<$ftp@santafe.edu$>$.                             \\
230-                                                                            \\
230-Please read the file README.Z                                               \\
230-  it was last modified on Fri Mar 19 15:29:44 1993 - 179 days ago           \\
230 Guest login ok, access restrictions apply.                                  \\
ftp$>$ {\sf cd pub/Users/terry/echo}                                            \\
250 CWD command successful.                                                     \\
ftp$>$ {\sf binary}                                                             \\
200 Type set to I.                                                              \\
ftp$>$ {\sf get Echo--1.0.tar.Z}                                                \\
200 PORT command successful.                                                    \\
150 Opening BINARY mode data connection for Echo-1.0.tar.Z (925969 bytes).      \\
226 Transfer complete.                                                          \\
local: Echo--1.0.tar.Z remote: Echo--1.0.tar.Z                                  \\
925969 bytes received in 6.1e+02 seconds (5.2 Kbytes/s)                         \\
ftp$>$ {\sf quit}                                                               \\
\end{sl}

Now you have retrieved the entire distribution. The distribution
consists of a number of files and directories, which were archived
into a single file that was then compressed. The next job is to
reverse these steps to recover the original Echo files.  To do this:

\begin{shell}
\prompt\ uncompress Echo--1.0.tar.Z \\*
\prompt\ tar xf Echo--1.0.tar \\*
\end{shell}

This should result in the creation of a new directory, called {\bf
Echo--1.0}. Check that that directory has been created. If so, and you
have received no error messages, it is safe to remove the bundled
distribution file with

\begin{shell}
\prompt\ rm Echo-1.0.tar
\end{shell}

\newpage
\section{Environment Variables and X Resources}
\label{setting-up}

This appendix deals with setting up environment variables and X
resources for those who preferred not to have this done automatically
in section \ref{simple-setup}.

You may wish to automatically install just the X resources, or just
the shell's environment variables. Do this with either

\begin{shell}
\prompt\ make x--setup
\end{shell}

or

\begin{shell}
\prompt\ make sh--setup
\end{shell}

In both cases, the shell script that is invoked (either {\bf
echo--x--setup} or {\bf echo--sh-setup}) tries to find an appropriate
file to append some text to. It is fairly conservative and always
makes a backup copy of any file it alters (in a file whose name ends
with {\bf .bak}).

There are several reasons why you may choose not to have either of
these setups done automatically. If you are an experienced UNIX user,
your shell startup and X resource files are probably not something
you'll feel comfortable having someone else's shell script edit
automatically. In this case, you should be able to decide how you wish
to do what follows.

Here are more details about exactly what needs to be done:

\begin{itemize}

\item
The file {\bf Echo.ad} contains X window \index{X Windows} resource
\index{X Resources} specifications.
These set the various colors of the Echo interface, set the sizes and
locations of the pieces of the interface and so on. If you know what
all this means, you should put the contents of this file somewhere
that the X toolkit will find them when Echo starts.

\item
Echo looks for certain environment variables \index{Environment
Variables} that can be used to influence its behavior. None of these
are required, but at least one is highly recommended.

Echo \index{Echo Objects} comes with a collection of pre--defined
worlds, sites and agents which are located in the {\bf OBJECTS}
directory. If you have installed Echo in the directory {\bf
/usr/local/echo} then you should set an environment variable called
{\sc echo\_location} and give it the value

\begin{verse}
{\bf /usr/local/echo/OBJECTS}
\end{verse}

If you do not do this, you will always have to invoke Echo from the
directory where you installed it for it to see the {\bf OBJECTS}
directory. Eventually you may have your own directory of Echo objects
and you can change this variable to point to it.

The way to set the environment variable depends on the shell you are
using. If you are in csh, the simplest thing to do is to place the
line

\begin{shell}
setenv ECHO\_LOCATION /usr/local/echo/OBJECTS
\end{shell}

in your {\bf $\sim$/.login} file. Then, whenever you log in, the variable
will be automatically set for you. To make it affect the current login
session, you can also type the line at the shell's prompt.

If you are not using csh or a variant of it, you should place the
following lines in your {\bf $\sim$/.profile} file (or equivalent),

\begin{shell}
ECHO\_LOCATION=/usr/local/echo/OBJECTS \\*
export ECHO\_LOCATION \\*
\end{shell}

And you can either log out and in again or type those two lines to the
current shell to have them affect this session.

If you aren't sure what shell you are running, type

\begin{shell}
\prompt\ echo \$SHELL
\end{shell}

and if the output ends in the letters {\sl csh}, then use the first
method above. If not, use the second\footnote{If the shell name ends
in {\sl bash}, the lines should go into your {\bf
$\sim$/.bash\_profile} file if you have one and your {\bf
$\sim$/.profile} if not.}.

The other three environment variables are less important, but will
make Echo start in a more attractive fashion. The variables and their
recommended values are given below.  These settings will make Echo
display a world, a site and an agent when it starts up. Otherwise the
three editing areas will be blank.

These variables can be set in exactly the same way that you set the
variable {\sc echo\_location} above. If you use C shell or a variant
of it, the following goes into your {\bf $\sim$/.login}:

\begin{shell}
setenv ECHO\_WORLD insects \\*
setenv ECHO\_SITE insects \\*
setenv ECHO\_AGENT fly \\*
\end{shell}

Otherwise, the following goes into your {\bf $\sim$/.profile}:

\begin{shell}
ECHO\_WORLD=insects \\*
ECHO\_SITE=insects \\*
ECHO\_AGENT=fly \\*
export ECHO\_WORLD ECHO\_SITE ECHO\_AGENT \\*
\end{shell}

\end{itemize}

\newpage
\section{Emacs Keybindings}
\index{Emacs Keybindings}
\label{emacs-keybindings}

In this section an uppercase {\sf C} will be used to represent the use
of the Control key and an uppercase {\sf M} to represent the Meta
key. Thus, {\sf C--x} indicates that you should hold down the Control
key and while doing so, type an {\sf x}. The Meta key is used in an
identical fashion, e.g.  {\sf M--d} tells you to first hold down Meta
and while holding it, press the {\sf d} key.

Table \ref{emacs-keybindings-table} shows the most useful key bindings
present in all text windows.

\begin{table}
\begin{center}
\begin{tabular}{||l|l||l|l||}
\hline
{\sf C--a} & Beginning Of Line.         & {\sf M--b}   & Backward Word. \\
{\sf C--b} & Backward Character.        & {\sf M--d}   & Delete Next Word. \\
{\sf C--d} & Delete Next Character.     & {\sf M--f}   & Forward Word. \\
{\sf C--e} & End Of Line.               & {\sf M--i}   & Insert File. \\
{\sf C--f} & Forward Character.         & {\sf M--v}   & Scroll Backwards. \\
{\sf C--h} & Delete Previous Character. & {\sf M--$<$} & Beginning Of File. \\
{\sf C--k} & Kill To End Of Line.       & {\sf M--$>$} & End Of File. \\
{\sf C--l} & Redraw. & & \\
{\sf C--n} & Next Line. & & \\
{\sf C--p} & Previous Line. & & \\
{\sf C--r} & Search Backwards. & & \\
{\sf C--s} & Search Forwards. & & \\
{\sf C--t} & Transpose Characters. & & \\
{\sf C--v} & Scroll Forward. & & \\
{\sf C--w} & Delete Selected Region. & & \\
{\sf C--y} & Paste Deleted Region. & & \\
\hline
\end{tabular}
\end{center}
\caption{Useful Emacs Keybindings For Text Windows}
\label{emacs-keybindings-table}
\end{table}

\newpage
\section{Echo Widgets}
\index{Echo Widgets}
\label{echo-widgets}

If you are not installing Echo on a SPARC machine, you will need to
obtain and install various widget sets on your machine before you can
make Echo. These are all freely available via {\bf ftp}. If you are
not familiar with this sort of installation, it might be best to
consult your system administrator.

The widget packages all create libraries that Echo must be linked with
when it is compiled. The {\bf Makefile} definition of {\bf LIBS} in
the distribution assumes that these libraries can be found in the {\bf
WIDGETS} directory. In fact, these libraries can be installed anywhere
that is convenient, as long as the {\bf Makefile} is altered to
reflect their location.

The various widget libraries and their {\bf ftp} locations are as
follows:

\begin{itemize}

\item {\bf Athena Plotter Widgets} \\
Machine: {\bf ftp.uni--paderborn.de}, Location: {\bf /unix/tools}.

\item {\bf Free Widget Foundation Widgets} \\
Machine: {\bf ftp.let.rug.nl}, Location: {\bf /ftp/pub/FWF/fwf.tar.Z}.

\item {\bf 3D Athena Widgets} \\
Machine: {\bf ftp.x.org}, Location: {\bf /contrib/Xaw3d}.

\end{itemize}

Each of these widget sets comes with instructions on how to make and
install the libraries that support the various widgets. The FWF and
Xaw3d widgets are available for {\bf ftp} from many locations. Use
{\bf xarchie} to find other sites.

Once these libraries have been created and Echo's {\bf Makefile} knows
where to find them, you should be able to proceed with the Echo
installation itself.

I apologize if you have to go through this procedure. It is the result
of using the X toolkit and the need to find useful (free) widgets. I
hope this will be solved in later Echo versions by using {\bf TCL} --
though this will require that the machine receiving Echo has (or
obtains) the {\bf TCL} libraries...

\newpage
\begin{thebibliography}{99}

\bibitem {holland-92a}
Holland, John H. (1992a).
{\em The Echo Model},
in ``Proposal for a Research Program in Adaptive Computation''.
Santa Fe Institute, July 1992.

\bibitem {holland-92b}
Holland, John H. (1992b).
{\em Adaptation in Natural and Artificial Systems}, 2nd Ed.,
MIT Press, 1992.

\bibitem {holland-93}
Holland, John H. (1993).
{\em ECHOING EMERGENCE : Objectives, Rough Definitions, and
Speculations for Echo-class Models},
To appear in ``Integrative Themes'', George Cowan, David Pines and
David Melzner Eds. Santa Fe Institute Studies in the Sciences of
Complexity, Proc. Vol XIX. Reading, MA: Addison--Wesley 1993.

\end {thebibliography}

\input{how-to.index}

\end{document}


\end{document}


\end{document}


\end{document}
