%The self-reproduction threshold is used to determine
%when enough resources have been collected. For example, if the genome
%contains three ``a'' resources and the self-reproduction threshold is
%two, then the reservoir must contain at least six ``a''
%resources before replicating.  A similar requirement must be satisfied
%for each of the resource types.


%\noindent*{Self-reproduction}
%\label{dynamics:self-reproduction}

%An agent may self-reproduce when it acquires sufficient resources.
%In self-reproduction, an agent makes a copy of its genome using the
%resources it has stored in its reservoir. The self-reproduction
%threshold is used to determine when enough resources have been
%collected. For example, if the genome contains three ``a'' resources
%and the self-reproduction threshold is two, then the reservoir must
%contain at least siz ``a'' resources before self-reproduction can take
%place. A similar requirement must be satisfied for each of the
%resource types. 

\section{Deleted Stuff}

\subsubsection{The Structure of an Echo World}


%% STEPH: do you think it's better to put all our parameter settings 
%%        in a table at the end of this section (or at the start of the
%%        one on results), rather than having them scattered through
%%        this section? I do I think.

%%  TERRY: I THINK THEY SHOULD ALL BE GATHERED TOGETHER IN A TABLE.

\noindent{Worlds}

The design of an Echo world requires the choice of a geography of
sites. In the current implementation, this geography must form a
rectangular array, though the general class of Echo models makes no
such restriction. In this paper, we shall only be concerned with
simple Echo models in which there is a single site. The choice of
geography usually involves a choice of the properties of the various
sites that will be created.

Another high-level decision in the construction of a world is the
choice of the number of resource types that will exist in the world.

There are a number of other global properties of a world that need to
be set:

\begin{itemize}
\item
The {\em interaction fraction\/} determines the number of agent-agent
interactions that will occur at a site in each Echo cycle. This is
determined by multiplying this fraction by the population at a
site. This many pairs of agents will be selected to interact. A value
of 0.02 was used in the results of this paper.

\item
The {\em self-reproduction fraction\/} and {\em self-reproduction
threshold\/} determine when agents in the world will be allowed to
self-reproduce. Self-reproduction is described in
section~\ref{dynamics:self-reproduction}. The first of these
world-wide parameters determines the division of excess resources
between parent and offspring after self-reproduction. The second
determines the level of resource accumulation necessary for
self-reproduction to occur. This is the number of copies of itself
that an agent must be able to make before it actually does make a
single copy. If this value is set to one, the parent and child will
initially have no resources, making them especially vulnerable to
taxation. These parameters were set to 0.5 and 2 in the Echo world
used in this paper.

\item
The {\em trading fraction\/} determines what proportion of an agent's
excess trading resource it will exchange in a trading interaction with
another agent. Trade is described in section~\ref{dynamics:trade}. A
value of 0.5 was used in this paper.

\item
The {\em taxation probability\/} is the probability that an agent gets
taxed in an Echo cycle. Taxation is described in
section~\ref{dynamics:taxation}. A value of 0.1 was used in this paper.

\item
The {\em neighborhood\/} determines the possible directions of
migration between sites. The three possibilities, given a rectangular
array of sites, are none (i.e. no migration), the four compass points
or the compass points plus the diagonals. Of course in the Echo model
described here, there is only a single site, and thus there can be no
migration between sites regardless of how this parameter is set.
\end{itemize}

An Echo world also requires the definition of a {\em combat matrix},
but a full discussion of that is beyond the scope of this paper.
%% STEPH: is it????
This matrix is used in the calculation of combat outcome
probabilities. For details, see~\cite{Holland92}. Combat is described
in section~\ref{dynamics:combat}.


\noindent{Sites}

An Echo site may initially contain an arbitrary number of agents.
These are arranged in a one-dimensional array. The probability that a
pair of agents will be chosen to interact falls off exponentially with
increasing distance between agents in this array. The user must decide
which agents initially reside at each site, and in what order they
should appear in the array.

There are three probabilities that must be set for each site:

\begin{itemize}
\item
The {\em mutation probability\/} is similar to that used in a genetic
algorithm.
%% STEPH: citation needed here????
Each locus of the genome of each agent at the site undergoes mutation
with the given probability at the end of each Echo cycle. In the Echo
model of this paper, this was set to 0.0005.

\item
The {\em crossover probability\/} is also used in a way that is
similar to the crossover probability of a genetic algorithm. It
determines the probability that two agents that are to undergo sexual
reproduction will be subject to crossover. In the Echo
model of this paper, this was set to 0.7.

\item
The {\em random death probability\/} determines the probability that
each agent at the site will be killed for no reason each Echo cycle.
This is described in section~\ref{dynamics:ZAP} and was set to 0.0001
in our experiments.
\end{itemize}

In addition to these probabilities, every site must also specify four
resource level vectors. Each of these consists of a vector of
non-negative integers whose length is the same as the number of
resource types in the world.  These vectors are used as follows:

\begin{itemize}
\item
The {\em production\/} vector contains the number of resources of each
resource type that will be produced by the site in each Echo cycle.

\item
The {\em initial\/} vector contains the number of resources of each
resource type that will be present at the site when it is first
created.

\item
The {\em maximum\/} vector contains the maximum level of each resource
type that can accumulate at the site.

\item
The {\em maintenance\/} vector contains the number of each resource
type that will be charged to an agent when it is taxed, as described
in section~\ref{dynamics:taxation}.
\end{itemize}

%\noindent*{Uptake of Resources}

%In every cycle, all the agents at a site compete for all the resources
%their uptake masks allow them to collect. The site produces resources
%according to its production vector, and these are distributed as
%equally as possible amongst the agents at the site that are
%genetically able to collect them.

%\noindent*{Return of Resources}

%When an agent at a site dies for some reason, its resources are
%returned to the environment and become immediately available to other
%agents at that site.

%\noindent*{Taxation}
%\label{dynamics:taxation}

%In each cycle, each site exacts a resource tax from each agent with a
%given (worldwide) probability. This can be thought of as economic
%taxation, or as the cost required to live at the site. Biologically,
%this can be thought of as metabolic cost.

%\noindent*{Random Death}
%\label{dynamics:ZAP}

%In each cycle, every agent in the world is killed for no apparent
%reason with some low probability. This can be interpreted as bad luck
%or as a mechanism that prevents agents from getting too old. If they
%are not killed some other way (through combat or taxation), they will
%eventually be randomly deleted.


STUFF ABOUT SPECIES ABUNDANCE:

%A number of attempts have been made to fit observed species abundance
%data to distributions. These include distributions based on harmonic
%numbers \cite{Corbet42}, a logarithmic distribution due largely to
%Fisher \cite{FisherCorbetWilliams43}, a class of truncated lognormal
%distributions \cite{Preston48,Preston62a,Preston62a}, the Poisson
%lognormal \cite{Bulmer74}, various distributions arising from
%MacArthur's ``broken stick'' model \cite{MacArthur57,MacArthur60} and
%a sequential broken stick model of \cite{Sugihara80} that, with
%seemingly reasonable assumptions, gives rise to Preston's
%``canonical'' lognormal distribution.

% \footnote{This is
% a generalization, not a re-phrasing, of MacArthur's broken stick model
% \cite{MacArthur57} which posits a distribution in addition to the
% perspective.}. Aspects of this combinatorial viewpoint have been
% well-studied in mathematics. For instance, the number of partitions
% into non-empty sets is given by the Stirling numbers of the second
% kind.

% These questions can
% naturally be extended to collections of samples and to partitions of
% samples.  In other words, biologists have been interested in the {\em
% distributions} of the sizes of such partitionings, the extent to which
% certain distributions are ubiquitous, and the reasons behind this.

% In the remainder of this section we give a brief
% outline of the answers that have been proposed.

% Preston's canonical lognormal distribution is doubly important since
% the conditions under which a lognormal curve is a member of Preston's
% canonical family give rise to the species-area constant of
% $z \simeq \frac{1}{4}$ in the species-area relationship $S = cA^z$
% \cite{Preston62a,Sugihara80,May75}. 

%This is based on sequential
%division of niche space, and can be described in terms of MacArthur's
%combinatorial broken stick model, with the crucial difference in
%resultant distribution arising from breaking the stick {\em
%sequentially\/} rather than simultaneously.

%The first thing we should do when considering species abundance in
%Echo is to be wary when talking of species. Within biology, there has
%been much discussion about what exactly a species is, how species are
%created and how they can be delineated
%\cite{Mayr82,Mayr88,Eldredge89}. For the purposes of this paper, it is
%not necessary to propose and defend a definition of species in
%Echo. This can be done, indeed it is not too hard to delimit species
%in Echo using Mayr's {\em Biological Species Concept\/} \cite{Mayr42}
%or Paterson's {\em Specific Mate Recognition System\/}
%\cite{Paterson85}. 

%We are interested in various ways of
%looking at Echo populations that may prove in accord with work on
%species abundance. This can later be used to help decide what in Echo
%could reasonably be considered a species, if it proves necessary to
%say what a species in Echo is. We propose to let processes observed in
%real biological systems help to draw conclusions about the nature of
%Echo, rather than to make a priori definitions and attempt to impose
%them on biology. 

% The figure is not based on any grouping of Echo individuals into
% ``species.'' In fact, it may not be unreasonable to consider
% individual Echo agents as species, as has been suggested by James
% Brown (private communication).

