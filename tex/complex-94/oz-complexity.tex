% \documentstyle[epsf]{iospress}
\documentstyle[epsf,fullpage,12pt]{article}

\begin{document}
\title{\bf Modeling \\ Complex Adaptive Systems \\
       with Echo \\ }

\author
{
{\bf Stephanie Forrest} \\
Department of Computer Science \\
University of New Mexico \\
Albuquerque NM 87131, USA. \\
forrest@cs.unm.edu \\
\and
{\bf Terry Jones} \\
Santa Fe Institute \\
1399 Hyde Park Road \\
Santa Fe NM 87501, USA. \\
terry@santafe.edu \\
}

\maketitle

\begin{abstract}

Complex adaptive systems (CAS) consist of many interacting and
adapting components.  Echo is a computational CAS model in which
evolving agents are situated in a resource-limited environment.
Different views of the notion of species within Echo are compared to
biological experiments on relative species abundance, specifically to
Preston's ``canonical'' lognormal distribution.

\end{abstract}

\noindent
\hspace*{1.5in} \hrulefill \hspace*{1.5in} \\
\noindent
To appear in: {\em Complex Systems: Mechanism of Adaptation},
R.J. Stonier and X.H. Yu (Eds), Amsterdam: IOS Press, September 1994,
pp. 3--21. \\
\hspace*{1.5in} \hrulefill \hspace*{1.5in}

\section{Introduction}

Many interesting systems are difficult to describe or control using
traditional methods.  These include
natural ecological systems, immune systems, economies and other social
systems.  One source of difficulty arises from nonlinear interactions
among system components.  Nonlinearities can lead to unanticipated
emergent behaviors, a phenomenon that has been well documented and
studied in physical, chemical, biological, and social systems as well
as in some forms of computation \cite{Forrest91b}.  Nonlinear systems
with interesting emergent behavior are often referred to as {\em
complex systems}.  A second form of complexity arises when the
primitive components of the system can change their specification, or
evolve, over time.  Systems with this additional property are
sometimes called {\em complex adaptive systems}.  Here, we will use
the term ``complex adaptive system'' (CAS) to refer to a system with
the following properties:
\begin{itemize}
\item A collection of primitive components, called ``agents.''
\item Interactions among agents and between agents and their environment.
\item Unanticipated global properties often result from the interactions.
\item Agents adapt their behavior to other agents and environmental 
constraints.
\item As a consequence, system behavior evolves over time.
\end{itemize}

% There are also all of the usual problems of
% verifying that complex simulation code is correct.  

Building models of CAS is difficult for several reasons.  First,
useful and predictive mathematical analyses rarely exist.  This is due
to both nonlinearities and the changing behavior of the primitive
elements of the system.  Second, detailed simulations are problematic
because it is virtually impossible to get all of the details correct.
Consider, for example, the vertebrate immune system which in some
cases has been estimated to express over $10^7$ different receptors at
a time.  Modeling the physical chemistry of just one receptor/ligand
binding event, even at an abstract level, requires enormous amounts of
computation, and it is therefore infeasible to model the expressed
repertoire of receptors precisely.  This problem exists for all large
complicated systems, but because nonlinear systems can be highly
dependent on seemingly small details, even a trivial inaccuracy in the
model could lead to wildly erroneous results.  One approach to this
dilemma is to strip away as much detail as possible, retaining only
the essential interactions.  The goal is then to develop models whose
behavior is robust with respect to the details of the interactions
(e.g., avoiding parameter tweaking to coax a system to produced
desired behaviors), and which produces the broad categories of
behaviors in which we are interested.  An implication of this approach
is that such models will rarely, if ever, be able to make precise
quantitative predictions.  Adaptation is central in CAS, and this is a
third reason that modeling CAS is difficult.  The underlying rules of
the system are changing over time which means that different agents
behave according to different rules at different times.

Because of these difficulties, a class of models, variously called
``artificial worlds,'' ``particle-based,'' and ``agent-based,'' have
been a popular approach to studying CAS.  This style of modeling is
quite different from the differential-equation style of models used
most frequently to model nonlinear dynamical systems.  In agent-based
models, each ``actor'' and each interaction among actors (i.e., not
just each type of interaction) is represented (simulated) explicitly.
Individuals are capable of quite different kinds of behaviors (the
agents in the system are heterogeneous).  Agent-based models are
discrete in most dimensions, typically time, state, and update rules.
Thus, the standard approximations for infinite-sized systems
and the techniques developed for studying asymptotic behavior of
continuous nonlinear dynamical systems often do not directly apply.
As a result, these systems tend to be more difficult to analyze.

Agent-based CAS models have several apparent drawbacks.  These include
the mapping problem, the problem of asking the right question, scaling
issues, and nonlinear interactions (already discussed).  Because CAS
models tend to strip away many details, it is often impossible to say
what any component of one of these models corresponds to in the real
world.  Continuing with the immune system example, many theoretical
immunologists use string matching to model receptor/ligand binding
\cite{PerelsonAndOster79}.  Patterns of bits (or other symbols) are
used to represent both molecular shape and electrostatic charge.
Consequently, it is difficult to say what one bit in the model
corresponds to in the immune system.  Since different alphabets and
different matching rules can have very different properties, the
challenge is to select an alphabet and matching rule that has general
properties similar to the real system without worrying too much what
each bit really stands for \cite{Smith94a}.  Most theories of modeling
are based on the premise that a correspondence can be established
between the modeled system and the primitive components of its model.
As a consequence of this mapping problem, it is not always clear what
scientific questions are being addressed by CAS models.  In more
conventional simulation-based modeling, models are used to make
quantitative predictions based on certain predicated inputs, for
example, to determine optimal parameter values.  Agent-based models of
CAS are rarely able to make this kind of quantitative prediction, and
as a result the focus is on identifying broad categories of behavior
and critical parameters (but not necessarily the exact critical
parameter values).  A third problem faced by agent-based models is one
of scale.  Because they are simulations, agent-based models typically
operate on vastly different time scales of evolution and with much
smaller population sizes than those of the systems they model.  Also,
we tend to be intolerant of high failure rates such as those often
observed in nature.  For example, consider the selection algorithms
typically used in genetic algorithms.  Selection pressure is
maintained at an artificially high rate and often scaled to maintain
increased pressure near the end of a run.  Evolution thus occurs
orders of magnitude more quickly than in natural systems, and as a
result, we may lose some of the richness of the natural evolutionary
process.

We have studied several different CAS models over the past fifteen
years.  Genetic algorithms \cite{Holland92} focus on the evolutionary
component of CAS.  They are reasonably well understood and mature, but
ignore several important features, including resource allocation,
heterogeneity, and endogenous fitness.  Classifier systems
\cite{HollandEtAl86,Forrest91a}
apply genetic algorithms to a cognitive modeling framework.
Similarly, Echo extends genetic algorithms to an ecological setting,
adding the concepts of geography (location), competition for
resources, and interactions among individuals (coevolution).  Echo is
intended to capture important generic properties of ecological
systems, and not necessarily to model any particular ecology in
detail.  What can we hope to learn with a model that by design does
not correspond to any real system?  We can study patterns of behavior,
e.g., how resources flow through different kinds of ecologies, how
cooperation among agents can arise through evolution, and arms races
\cite{Holland94}.  We can also use such a model to identify 
parameters or collections or parameters that are critical, i.e., to
perform sensitivity analysis.  As with any simulation tool, it is much
easier to run hypothetical what-if experiments than to conduct
experiments on a real system.  If a model like Echo were successful
and correct, it would enable users to build deep intuitions about how
different aspects of an ecological system affect one another,
important dependencies, and an appreciation of how evolution interacts
with the ongoing dynamics of an ecology.  This is perhaps the most
important contribution that models like Echo can make.  The original
idea of Echo, including motivation, design decisions, and overall
structure were introduced in \cite{Holland92,Holland94}.  Our goal in
this paper is to describe more fully one specific Echo model (Echo
really refers to a class of models) and to show how one might study
the extent to which Echo does or does not capture important properties
of ecological systems.  Towards this end, we report preliminary
results on the relative abundance of species, an important feature of
any ecological system.  This feature raises some fundamental
questions, such as how to define precisely the concept of ``species''
in Echo, which we also discuss.


\section{Echo}

% Ecology == communities of agents interacting and evolving ?

Echo was designed to capture the essential features of ecological
systems in an agent-based model.  All of the entities and interactions
in Echo are highly abstract, and it is not yet known whether Echo can
be used to model real-world phenomena effectively.  Many CAS can be
viewed as ecologies (e.g., \cite{Huberman91}), but our focus in this
paper is on the analogy with natural ecologies.  Echo resembles some
other CAS models.  These include Swarm \cite{Langton94}, Sugarscape
\cite{Epstein94}, and the Evolutionary Reinforcement Learning (ERL) model
\cite{AckleyAndLittman92}.  Unlike Swarm, Echo makes specific
commitments about agent types and interactions; it differs from
Sugarscape, both in specific details, and in its focus on ecological
principles; ERL provides two levels of learning (there is only one in
Echo) but is not intended as a general ecological model.

Echo extends classical genetic algorithms in several important ways:
(1) fitness is endogenous, (2) individuals (called agents) have both a
genome and a local state that persists through time, and (3) genomes
are highly structured.  In Echo, an agent replicates (makes a copy of
itself, possibly with mutation) when it has acquired enough
``resources'' to copy its genome.  The local state of an agent is
exactly the amount of these resources it has stored.  Agents acquire
resources through interactions with other agents (combat or trade) or
from the environment.  This mechanism for ``endogenous'' reproduction
comes much closer to the way fitness is assessed in natural settings
than conventional ``fitness functions'' in genetic algorithms.

Along with these extensions to the evolutionary component, Echo
specifies certain structural features of the environment in which
agents evolve.  Specifically, there is a two-dimensional grid of
``sites'' and each agent is located at a site, although it is possible
for agents to move between sites.  There are usually many agents at
one site, and there is a notion of neighborhood within a site.  Each
site may produce renewable resources.  These resources are represented
by different letters of the alphabet, and genomes are constructed from
the same letters.  Resources can exist in three places: as part of an
agent's genome, as part of an agent's local state, or free in the
environment.  There are three forms of interactions among agents:
trade, combat, and mating.  In trade, resources stored internally (the
local state) are exchanged; in combat, all resources (both genetic and
stored) are transferred from loser to winner; in mating, genetic
material is exchanged through crossover, thus creating hybrids.
Mating, together with mutation during the replication process,
provides the mechanism for new types of agents to evolve, as shown in
Figure~\ref{fig:agent-evolution}.  Resource constraints provide the
pressure for agents to diversify and occupy new niches.

\begin{figure}[htb]
\begin{center}
\leavevmode
\epsfysize=3in
\epsfbox{figures/agent-evolution.ps}
\caption{The ways in which an Echo agent can undergo genetic modification.
\label{fig:agent-evolution}}
\end{center}
\end{figure}


In each Echo run there is a fixed number of resource types which is
determined by the user of the system. These may be representative of
resources in a real-world system, or may correspond to a more abstract
notion of something that is required to ensure survival.  For example,
the environment can be designed to require that agents possess a
certain resource, which some agents may only obtain through trade. In
this situation, the resource need not be thought of as corresponding
to a physical entity, but as something that requires a certain type of
agent-agent interaction for agent survival. The number of resources in
an Echo world is typically small. These are denoted by lower-case
letters: {\em a}, {\em b}, {\em c\/} and so on. In the Echo world used
in this paper, there are four resources and one site.

The following sections describe Echo in more detail. Much of this is
devoted to describing agents and the interactions that can occur, both
between pairs of agents and between an agent and its environment.

\subsection{Echo Structure}

Our implementation of Echo divides Echo into a structural hierarchy.
Each run of Echo involves a {\em world\/} that contains a fixed number
of {\em sites\/}.  Each site may contain an arbitrary number of {\em
agents}, including zero.  Each world specifies certain system-wide
parameters, including the number of sites, the number of resource
types, the taxation rate, parameters controlling replication, and the
probability of random death.  See \cite{JonesForrest93} for details of
these parameters.  Each site specifies its own mutation, crossover,
and random death probabilities, as well as some parameters controlling
the details of how resources are managed at the site (e.g., the
maximum amount of a resource that can accumulate at the site).

Each of these components is designed by the user of the system,
typically as an abstraction of some aspect of a real-world CAS.  In each
case, the use of Echo requires decisions about the structure of these
objects and the ways they will behave when the result is set in
motion.  This paper refers briefly to the elements of worlds and
sites.  A full description of these elements can be found in any of
\cite{JonesForrest93,Holland92,Holland94}.  Section~\ref{agents}
describes the structure and properties of agents.

\subsection{The Echo Cycle}

The sequence of events in an Echo cycle consists of the following:
\begin{enumerate}
\item
Interactions between agents are performed at each site.  These include
trade, mating, and combat.  The number of interactions is controlled
by a ``world'' parameter.

\item
Agents collect resources from the site if any are available.  The site
produces resources according to its ``site'' parameters, and these are
distributed as equally as possible among the agents at the site that
are genetically able to collect them.

\item
Each agent at each site is taxed (probabilistically).
Each site exacts a resource tax from each agent with a given
(worldwide) probability. If an agent does not possess the resources to
pay the tax, it is deleted and its resources are returned to the
environment.  Tax in Echo can be thought of as economic taxation, or
as the cost required to live at the site.  Biologically, this can be
thought of as metabolic cost.

\item
Agents are killed at random with some small probability.  This can be
interpreted as bad luck or as a mechanism that prevents agents from
living forever. If they are not killed some other way (through combat
or taxation), they will eventually be randomly deleted.  

\item
The sites produce resources. Different sites may produce different
amounts of each resource.  For example, one site may produce ten {\em
a}'s and ten {\em b}'s on each time step, whereas another may produce
five {\em b}'s and twenty {\em c}'s.  The thought is that agents will
replicate frequently if they are located at sites whose resources
match their genomes, if the site is not too crowded.  When an agent at
a site dies, its resources are returned to the environment and become
immediately available to other agents at that site.

\item
Agents that have not received resources this cycle migrate.  If an
agent does not acquire any resources during an Echo cycle (either
through picking them up or through combat or trade), it will migrate
to a neighboring site. The neighboring site is selected at random from
among those permitted by the geography of the world. This is
not the same as the local movement within a site that occurs as the
result of the agent-agent interactions that are described in
section~\ref{agent-agent}.

\item
Agents that can replicate do so (asexual reproduction).  An agent may
replicate when it acquires sufficient resources.  In replication, an
agent makes a copy of its genome using the resources it has stored in
its reservoir.  A parameter controls how many resources are required
to be stored beyond those needed to make an exact copy. The
replication process is noisy: random mutations may 
result in genetic differences between parent and child.

\end{enumerate}
This cycle is iterated many times during the course of a ``run.''


\subsection{Agents}
\label{agents}

Figure~\ref{fig:example-agent} illustrates an example Echo agent.
Agents have a genome which is roughly analogous to a
single chromosome in a haploid species. The chromosome has $r + 7$
genes, where $r$ is the number of resources in the world.  Each of
these genes can be altered by the mutation operator. Six of
these, the {\em tags\/} and {\em conditions\/} are composed of
variable-length strings of resources (i.e. of the lower-case letters
that represent resources). The mutation operator can alter the allele
value at any locus, and can also cause a tag or condition to grow or
shrink in length.

\begin{figure}[htb]
\begin{center}
\leavevmode
\epsfysize=3.5in
\epsfbox{figures/example-agent.ps}
\caption{The structure of an Echo agent. Tags are visible to the
outside world. Conditions and other properties are not.
\label{fig:example-agent}}
\end{center}
\end{figure}

Tags are genes that produce some easily observable feature of the
phenotype. Conditions are genes that do not produce observable
phenotypic effects, and their result cannot be detected by other
agents. Thus an agent will interact with another on the basis of its
own conditions and the other's tags. This allows, for example, the
possibility of agents that appear dangerous but are in fact usually
unwilling to fight. It also allows for the evolution of intransitive
combat relationships. For example, an agent $A$ might always attack an
agent $B$, and $B$ always attack $C$, but it does not follow that $A$
will attack $C$. This has obvious parallels in real-world systems
(e.g., in food webs). The importance of this kind of relationship
among agents in CAS has often been stressed
\cite{Holland92,Holland93b}.

The six tag and condition genes possessed by every agent are the {\em
offense tag\/}, {\em defense tag\/}, {\em mating tag\/}, {\em combat
condition\/}, {\em trade condition\/} and {\em mating condition\/}.
These genes are used to determine what sort of interaction will take
place between a pair of agents, and what the outcome will be. The use
of these genes is described below. It should be noted that the current
implementation conforms to a very large extent with the description
given in~\cite{Holland92}, but not with that in~\cite{Holland94}.

The $r$ genes correspond to the agent's {\em uptake mask}, which
determines its ability to collect each resource type directly from the
environment. If an agent does not have a '1' allele for the uptake
gene corresponding to a certain resource, it will not be able to
collect that resource if it encounters some amount of it at a
site. Consequently, if the agent requires this resource (for example
because the site at which it is located charges a tax that includes
it, or because the agent needs it to replicate), it will either have
to fight or trade for it. The designer of an Echo world can create
trading webs among agents by requiring them to trade in various ways
to ensure survival. Of course there is nothing in Echo to guarantee
that such webs will not soon be greatly altered through mutation, or
that they will survive at all.
The final gene is the {\em trading resource\/} which is the resource
type that the agent will provide to another agent if trading takes
place.
Each agent also has a {\em reservoir\/} in which it keeps some amount
of each resource type. Resources from the reservoir are used to pay
taxes, to produce offspring and for trade.  The reservoir corresponds
exactly to the local state of the agent.

Agents at a site are arranged in a one-dimensional array. The
probability that a pair of agents will be chosen to interact falls off
exponentially with increasing distance between agents in this
array. The user must decide which agents initially reside at each
site, and in what order they should appear in the array.


\subsection{Agent-Agent Interactions}
\label{agent-agent}

There are three main forms of agent-agent interaction: combat, trading
and reproduction. All of these interactions take place between agents
that are located at the same site and all involve the transfer of
resources between agents

\subsubsection{Combat}

Combat is an idealization of any interaction that might occur between
real-world entities that is antagonistic. It does not necessarily
imply that the agents are actually fighting, though of course this is
not precluded. If two agents in a real-world system are behaving in a
competitive fashion, this would be modeled in Echo by designing the
agents in such a way that they would engage in combat. When combat
occurs, one agent is always killed, and its resources are given to the
survivor. In a more recent version of Echo \cite{Holland94}, the
interaction need not be so extreme and results in a transfer of
resources (possibly in both directions, and possibly in a very uneven
fashion) between the agents.

When two agents encounter each other, the system first checks to see
if either would attack the other. An agent $A$ will attack an agent
$B$ if its combat condition is a prefix of $B$'s offense tag.  If
attacked, an agent is given a chance to flee (which it does with a
probability equivalent to the probability of it losing in the combat
encounter). The calculation of the probability of victory in combat is
somewhat complicated and is not described fully here. It is based on
matching $A$'s offense tag with $B$'s defense tag and vice versa. The
resource characters that comprise these genes are used as an index
into a {\em combat matrix}, with special provisions for zero length genes
and for genes of unequal length.

As a result of this computation, each agent receives some number of
points. If $A_p$ and $B_p$ are the points awarded to $A$ and $B$, 
then $A$ will win the combat with a probability of $A_p / (A_p + B_p)$.
The resources that comprise the loser (both its genome and the
contents of its reservoir) are given to the winner and the loser is
removed from the population.

\subsubsection{Trade}

If two agents are chosen to interact and they do not engage in combat,
they are given the opportunity to trade and mate. Unlike combat,
trading and mating must be by mutual agreement. Agents $A$ and $B$
will trade if $A$'s trading condition is a prefix of $B$'s offense tag
and vice versa. Notice that the offense tag is used here as well as in
determining whether combat will occur.

When trade takes place, each agent contributes its excess trading
resource.  Excess is defined to be the amount of resource that an
agent possesses above that which is required to replicate its genome,
plus some reserves (system parameters control how much reserve an
agent retains).  Thus an agent provides some fraction of the resource
that it does not need for the next self-reproduction. This may be
zero, in which case an agent does not provide anything in the
trade. This behavior is analogous to a form of deception or
bluffing. An agent cannot know in advance if another agent will supply
a positive quantity of a resource, or what that resource may be. This
may seem an odd form of trade, but agents can ``learn'' to recognize
each other based on their trading tags. Agents whose tags tend to
involve them in disadvantageous trades will tend to reproduce less
quickly and tend to have smaller probabilities of being able to meet
taxation demands.

\subsubsection{Sexual Reproduction}

Agents that interact and do not engage in combat may produce offspring
through recombination. As in many genetic algorithms, the
offspring replace the parents in the population. Sexual reproduction
occurs between two agents $A$ and $B$ if $A$ finds $B$ acceptable and
vice versa. $A$ will find $B$ acceptable if either 1) $A$'s mating
condition is a non-zero prefix of $B$'s mating tag or 2) both $A$'s
mating condition and $B$'s mating tag are zero length. The restriction
to non-zero prefixes is designed to stop agents with zero-length
mating conditions from rapid proliferation. Such an agent finds all
other agents desirable (including copies of itself). To prevent this,
an agent with a zero length mating condition will only find an agent
with a zero length mating tag acceptable. This is a slight departure
from the description of mating given in \cite{Holland92}.
Figure~\ref{fig:agent-agent} shows a simplified view of the two-way
matching process used to determine whether mating will occur.

\begin{figure}
\begin{center}
\leavevmode
\epsfysize=3in
\epsfbox{figures/agent-agent.ps}
\caption{A simplified view of the two-way tag and condition matching
that is used by agents to determine whether mating will occur.
\label{fig:agent-agent}}
\end{center}
\end{figure}

When sexual reproduction does occur, a form of two-point crossover is
employed. This is complicated by the fact that agent genomes are
variable length. Thus one can choose a crossover point in one agent
and find that the same crossover point does not exist in the other
agent.  Without going into detail, two genes are selected to contain
crossover points.  Then the actual crossover points are chosen in each
gene in each agent, and the crossover is performed. The operation
conserves resources (i.e. resources are not created or destroyed) but
the ratio of genetic material from each parent in each of the children
will typically not be 50:50.

%
% STEPH: you didn't respond to the comment below, so I made an
%        executive decision (a.k.a. power play) and axed this...
%
%\subsection{Agent-Environment Interactions}
%
%% STEPH: Perhaps this section can just go now. I don't think it would
%%        be a great loss.
%There are three interactions that involve a single agent and the
%environment: uptake of resources, return of resources at death, and
%payment of tax. Each of these has already been discussed.

\subsection{Agent Movement}

There are two forms of agent movement in Echo: within a single site
and between sites. Intra-site movement is the result of an
agent-agent interaction.  In each of these, one agent is first
selected.  A second agent is then selected in the vicinity of the
first. The first agent is moved next to the second in the
one-dimensional array of agents at the site. If the first agent would
attack the second, the second may run away by moving a small distance
away in the array. In both cases, distances are likely to be small,
with the probability of a large distance being used falling off
exponentially.

Inter-site movement occurs if an agent does not acquire any resources
during an Echo cycle (either through picking them up, combat or
trade). In this case it will migrate to a neighboring site, selected
at random from among those permitted by the geography of the world.

\section{Experimental Results: Species Abundance and Echo}

In this section we present preliminary results comparing
Echo populations with previous work on relative species
abundance.  Our overall goal is to confirm or disconfirm the
hypothesis that Echo exhibits many of the same broad classes of
behaviors as natural ecological systems.  Because Echo emphasizes
evolution, a natural starting point in the confirmation process is to
ask whether or not evolution in Echo produces distributions of agents
that are similar to or different from those observed in natural
systems.  Although we are still in the early stages of this
investigation, our results to date are encouraging.

As we discussed earlier, it can be quite difficult to say what the
individual components of a CAS model like Echo actually correspond to
in the modeled system.  To address the question of species abundance,
for example, we need to define exactly what we mean by a species.  The
concept of species is not directly built into Echo, and there are a
number of ways in which species could be defined.  The simplest of
these is to simply interpret individual Echo agents as species.  A
second interpretation, perhaps more appealing, is to group genetically
related agents together in species.  In the following we consider
both of these interpretations.

\subsection{Introduction to Species Abundance}

% STEPH: Do you like this fishing boat thing? I'm not particularly
% attached to it. I just put it in because i find it a bit hard
% to describe what it is we're looking at. John's example with
% common and rare words might be better. i like this enough to leave
% it in, but wouldn't mind at all if you wanted to cut it.
%
% Terry: I think it is fine.  We can revisit later.

Suppose we took the catch from a laden fishing boat returning to
harbor and sorted the fish according to species. What would the
distribution of fish into species look like? The answer, of course,
will depend on many factors---weather, bait (if any), the depth at
which the fish were caught, the water temperature at that depth, the
size of the catch, and myriad others. Experiments of this nature have
been performed many times by biologists, with samples of many sizes
drawn from taxa including birds, snakes, fish, snails, lepidoptera,
phytoplankton, arthropods, mammals and many others.  A general
perspective on such experiments is to consider the ways in which the
$n$ individuals that are sampled can be partitioned to represent a
(typically unknown) number of $m$ species.  From a biological
perspective, the interesting questions are: Does the distribution
into species follow a pattern that can be characterized
mathematically? And if so, are there biological theories that can
account for this pattern?  In many cases it is possible to fit
mathematical models of distribution to observed patterns and to give
plausible biological explanations for why these patterns should arise.
See, for example,
\cite{Corbet42,FisherCorbetWilliams43,Preston48,MacArthur57,MacArthur60,Preston62a,Preston62b,Sugihara80}.

A commonly observed phenomenon, is that the vast majority of species
in a sample are made up of relatively few individuals. The conditions
under which distributions of this kind are seen include early
successional communities, environments perturbed by toxins or
pollutants, and in appropriately sized samples
\cite{MacArthur60,May86}. Relatively stable ``climax'' communities
consisting of many species typically do not exhibit this qualitative
pattern.

In examples where this general pattern is seen, Preston's canonical
lognormal distribution has often proved the most accurate model,
e.g. \cite{Williams64}. Preston \cite{Preston48} took the counts for
the various species in observed data and grouped them into a series of
``octaves.'' This was simply a (base 2) logarithmic grouping of
the species counts. His octaves were labeled ``$<1$'', ``1--2'',
``2--4'' and so on. Octaves were plotted on the $x$-axis and the
counts of the species in each octave, a frequency of frequencies, was
plotted on the $y$-axis. If a species count fell within octave
boundaries, it counted 1 for that octave. If a count fell on the
boundary between octaves, (as any count that is a power of 2 will),
one-half was counted for the neighboring octaves.

Preston plotted these ``species curves'' for a number of experiments,
and found that their general shape was well approximated by a Gaussian
(normal) distribution of the form
$$ y = y_0 e^{-{(aR)}^2} $$
where $y$ is the number of species falling into the $R^{th}$ octave left
or right of the modal octave, $y_0$ is the value of the mode of the
distribution and $a$ is a constant, related to the logarithmic standard
deviation, to be determined from the data \cite{Preston48}.

Because it is not possible to observe less than a single individual
from a species in a sample, these distributions were truncated on the
left at what Preston called the ``veil line.''  As the distribution of
octave counts is reasonably approximated by a normal
distribution, the original species counts were postulated to come from
a lognormal distribution. In particular, Preston found that the value
of $a$ that was calculated for the experiments he examined tended to
be in the vicinity of $0.2$. This gave rise to the ``canonical''
lognormal distribution of \cite{Preston62a,Preston62b}.  In the
canonical distribution the general lognormal distribution is reduced
to a family of lognormal distributions dependent on a single
independent variable. This relationship makes it possible to form good
predictions of species relative abundance given only the number of
individuals or the number of species \cite{Preston62a,Sugihara80}.

There are a number of conditions under which Preston's canonical
distribution might be expected to arise, mentioned above.  Alternative
explanations for the occurrence of this distribution have also been
advanced \cite{Hogg90a}.  These range from arguments that such
distributions are an artifact of the Central Limit Theorem, to simple
statistical arguments.  When these do account for the lognormal
distribution, they fail to account for the fact that a wide range of
experimental data is is not only lognormal, but also is close to
Preston's canonical family of lognormal distributions. Sugihara
\cite{Sugihara80} discusses these attempts and presents a biologically
plausible alternative that generates the canonical distributions.

\subsection{Species Abundance in Echo}

In this section we consider different groupings of Echo agents, any
one of which could be potentially considered a species in Echo.  This
section does not provide details of the various Echo worlds that have
been observed to produce the effects described. To a large extent,
this is not necessary as these can be seen in a wide range of Echo
worlds in populations that are of reasonable size (roughly several
hundred agents) that have been allowed to evolve for a reasonable
number of iterations (at least two hundred cycles).  In all of the
figures of this section, the populations are taken from Echo worlds
that were stopped after 1000 generations.  The parameter settings that
have been held constant throughout the experiments reported in this
section are summarized in Table~\ref{tab:echo-params}.  Details on the
precise meaning of these parameters are provided in
\cite{JonesForrest93}.

\begin{table}
\begin{center}
\begin{tabular}{|l|r|} \hline
Parameter  & Value  \\
\hline
Number of Resources & 4   \\
Trading Fraction    & 0.5 \\
Interaction Fraction & 0.02 \\
Self Replication Fraction & 0.5 \\
Self Replication Threshold & 2 \\
Taxation Probability & 0.1 \\
Number of Sites & 1 \\
\hline
Mutation Probability & 0.02 \\
Crossover Probability & 0.7 \\
Random Death Probability & 0.0001 \\
\hline
\end{tabular}
\end{center}
\caption[hey]{The world and site parameters that were held constant
throughout this section. Those above the line are the worldwide
parameters. These parameters are described in \cite{JonesForrest93}.
\label{tab:echo-params}}
\end{table}

These effects were also observed in earlier versions of the program in
which several properties of the model were slightly different.  In
fact, Echo agents at one point managed to find and exploit a hole
(bug) in the function that calculated the points agents receive in
combat. When exploited, this typically results in an agent becoming
relatively powerful and that agent and its offspring will tend to
quickly dominate the world. Nevertheless these agents and those that
found ways to survive, produced graphs of ranked genome abundance that
were similar to those of the corrected program.  All of this suggests
that species abundance patterns in Echo are very robust.  

The simplest way to study relative abundance in Echo is to sort the
genomes by their abundance, and to plot these by rank on the $x$-axis
and by number of individuals on the $y$. This was the method used by
MacArthur \cite{MacArthur57,MacArthur60} and the data shown in
Figure~\ref{fig:ranked-genome-frequencies} are similar to his
graphs. This figure was produced by simply examining the number of
copies of individual genomes in the population after 1000 generations
of an Echo run.

\begin{figure}
\begin{center}
\leavevmode
\epsfysize=3in
\epsfbox{figures/ranked-genome-frequencies.ps}
\caption{An example of the abundance of Echo genomes in a population
after 1000 cycles. Abundances are ranked from commonest (left) to
rarest (right), with the actual abundance given on the $y$-axis.
The final population contained 603 different genomes.
\label{fig:ranked-genome-frequencies}}
\end{center}
\end{figure}

Taking the population data from the same Echo run and organizing it
into octaves using the method described by Preston \cite{Preston48}
results in Figure~\ref{fig:ungrouped-prestonized}.  This figure bears
a strong resemblance to those of Preston, especially those in which
the veil line is close to the mode of the distribution.  It is clear
that the character of genome abundances in Echo populations tends to
follow the general patterns found in some biological systems. The
question is how close is the correspondence.

\begin{figure}
\begin{center}
\leavevmode
\epsfysize=3in
\epsfbox{figures/ungrouped-prestonized.ps}
\caption[hey]{The population data from Figure
\ref{fig:ranked-genome-frequencies} organized into octaves according
to the method of Preston \cite{Preston48}.
\label{fig:ungrouped-prestonized}}
\end{center}
\end{figure}

There are two important aspects of this correspondence: (1) how Echo
agents are grouped into ``species, '' and (2) how the result is
sampled.  In Echo, there is no {\em a priori} grouping, one has to be
defined. We have already seen the simplest case, in which each genome
is considered a group, and that this gives rise to graphs that
resemble those of biological systems.  We have considered several
possible strategies for grouping, including clustering based on
genetic distance (e.g., see Figure~\ref{fig:clustering}), clustering
based on functional properties (agents that act alike are grouped
together), and clusterings based on evolutionary history (agents that
evolved together are grouped together).  Here, we examine groupings
based on genetic distances, and we use a simplistic method of deciding
where to impose ``species'' boundaries between clusters.

The second dimension is of great importance in both biological systems
and in Echo.  Sample size (and location) can completely determine
whether distributions such as those shown will appear. This has been
mentioned in virtually every work cited in this section.  It may be
the case that a very large sample does not exhibit certain properties,
but if that sample is divided into a set of smaller samples at random,
then each of the smaller samples will show the highly skewed
distribution. The locality from which the sample is drawn will also
have a great affect since most species show considerable variation in
relative density over their entire range of habitats.  Thus even if
all species contained exactly the same number of individuals, this
variation could produce a skewed distribution if the sample size were
small relative to the total number of individuals.

In Figures~\ref{fig:ranked-genome-frequencies} and
\ref{fig:ungrouped-prestonized} there is no grouping and no sampling.
As a result, all curves for Echo derived in the method of Preston will
have a mode of one, since every single individual is present in the
data and there is great variation at the level of individual genomes.
Such sampling is rare (but not unheard of) in biological systems.

\begin{figure}
\begin{center}
\leavevmode
\epsfysize=3in
\epsfbox{figures/clustering.ps}
\caption{A fragment of a cluster analysis of Echo agents based on
genetic distance.
\label{fig:clustering}}
\end{center}
\end{figure}


Using the minimal number of mutations required to transform one agent
into another as a distance metric, we used a hierarchical clustering
algorithm to cluster the genomes of populations. At each iteration,
the clustering algorithm locates the two clusters at minimum distance
and combines them.  By imposing a maximum on this distance, the
algorithm can be restricted from proceeding all the way to a single
giant cluster. We then consider each of the clusters that has been
formed to represent a species.  When the limit is reached, any agents
that have not been included in a cluster will be considered
singletons---the sole representatives of a species.
Table~\ref{tab:cluster-params} shows the number of species that are
produced from three bounded clusterings of three example Echo runs.
\begin{table}
\begin{center}
\begin{tabular}{|c|c|c|c|c|c|} \hline
Cluster  & Resource & Total   & Non-singleton & Non-singleton & Singleton \\
limit    & level    & species & agents        & species       & species   \\
\hline
         & 100      &     187 &           128 &            43 &       144 \\
\cline{2-6}
10       & 200      &     294 &           214 &            72 &       222 \\
\cline{2-6}
         & 300      &     462 &           235 &            95 &       367 \\
\hline
         & 100      &      77 &           225 &            30 &        47 \\
\cline{2-6}
15       & 200      &     149 &           331 &            44 &       105 \\
\cline{2-6}
         & 300      &     303 &           410 &           111 &       192 \\
\hline
         & 100      &      19 &           265 &            12 &         7 \\
\cline{2-6}
20       & 200      &      50 &           412 &            26 &        24 \\
\cline{2-6}
         & 300      &     140 &           514 &            52 &        88 \\
\hline
\end{tabular}
\end{center}
\caption{The number of species resulting from different bounding
conditions on genetic clustering of Echo agents. The experiments all
consider the same world with differing resource levels provided by the
site. The sizes of the final populations in the three experiments
were: 1191, 2388 and 3509.
\label{tab:cluster-params}}
\end{table}


Figure~\ref{fig:preston.300.20} plots an example of the data in
Table~\ref{tab:cluster-params}.  The curve was obtained from the
experiment in which the site produced 300 units of each resource in
every Echo cycle. Here the clustering algorithm was prevented from
combining clusters with an average distance of greater than 20.
\begin{figure}
\begin{center}
\leavevmode
\epsfysize=3in
\epsfbox{figures/preston.300.20.ps}
\caption{The species curve resulting from genetic clustering of 3509
Echo agents. Clustering was restricted to distance 20 or less.
\label{fig:preston.300.20}}
\end{center}
\end{figure}
This can be compared to Figure~\ref{fig:preston.300.10} which shows
exactly the same experiment (i.e. started with the same random seed)
but with the clustering limit set to 10. There are several differences
between the graphs that are not difficult to account for. The first
has a larger number of octaves expressed, which is a direct result of
grouping agents into fewer categories (140 species as opposed 462, as
shown by Table~\ref{tab:cluster-params}). On average, categories will
tend to be larger and thus more octaves will be represented.
\begin{figure}
\begin{center}
\leavevmode
\epsfysize=3in
\epsfbox{figures/preston.300.10.ps}
\caption{The species curve resulting from genetic clustering of 3509
Echo agents. Clustering was restricted to distance 10 or less.
\label{fig:preston.300.10}}
\end{center}
\end{figure}
The heights of the modes of the two figures also differ considerably.
This is to be expected since the higher clustering distance limit will
gather more singletons into clusters before halting. This results in
far fewer species falling into the lower octaves. The first figure,
with the higher clustering limit, more closely resembles the figures
found in \cite{Preston48}. The clustering method, in all the cases
examined, reduces the height of the mode of the species curve 
significantly.

We tried a simple sampling method (results not shown), which does
not appear to produce any change in distributions.  In it, each agent
in the population is sampled with some fixed probability.  However, we
expect that sampling based on Echo's geography will produce marked
changes.

\section{Conclusion}

Our preliminary work on species abundance is encouraging, and
there are several directions in which we plan to extend it.  These
include deciding how to limit the clustering algorithm based on
population size; examining other methods for grouping, in particular
clustering based on agent behavior and evolutionary history;
investigating sampling methods; and finally, fitting Echo data
obtained from different choices of grouping and sampling to that of
the various models of relative species abundance.
These directions are not independent. The extent to which Echo data
will fit existing work on species abundance will, as described above,
depend on how species are delineated in the model and on how
populations are examined. Given the tendency for this qualitative
behavior to be present in several different versions of Echo, it seems
likely that there will be no single correct answer.  Rather, we expect
to identify some perspectives on Echo that are most appropriate for
modeling biological ecologies.

Examining species abundance is our first formal step in the validation
of Echo.  Informally, a number of interesting phenomena have also been
reported, such as the evolution of ``arms'' races.  This suggests that
Echo is quite a rich system.  Our approach to validating Echo as an
ecological model is to perform a series of small experiments, each of
which is designed to explore one aspect of Echo's behavior.  If the
system performs realistically on this set of experiments, we will have
much more confidence in Echo's relevance to real-world systems.
We believe that such a validation will increase the confidence with
which the model can be applied.  

It will be a long time before models like Echo can be used to provide
quantitative answers to many questions regarding CAS.  A more realistic
goal is that these systems might be used to explore the range of
possible outcomes of particular decisions and to suggest where to look
in real systems for relevant features.  The hope is that by using such
models, people can develop deep intuitions about sensitivities and
other properties of their particular worlds.  High-level knowledge of
this kind could be very valuable.  In many CAS, a small increment in
intuition would translate into large gains.  For example, even a very
small increment in our intuitions about the likely behavior of some
aspect of the economy or environment could be used to great effect.
We view Echo as an early step in the
building of CAS models.  The process of validating such models is a
daunting task.  We hope that by examining carefully the model's
behavior we will learn lessons that are also valuable to the
development of future models with similar aims.


%``That is, the goal is to develop robust models whose behavior doesn't
%depend on the details of the interactions, and produces broad
%categories of behaviors.''  We did this in the experiments.
%
%\section{Misc}
%
%What can we hope to learn with highly abstract idealized models like Echo?
%1. Patterns of behavior, e.g., flow of resources, cooperation among agents,
%arms races, communities.
%2. Critical regions of parameter space. (John's ``levers'').
%3. Interactions between local situations and global effects.
%4. Effects of exogenous and endogenous changes.
%Note: unlikely to get real quantitative predictions, but
%can do some kind of sensitivity analysis.  For example, if you
%do X, event Y will happen with Probability P.
%5. Flight simulator
%
%
%Need to discuss large scale of systems earlier.
%
%Need to introduce an example, possibly follow immune system example
%all the way through.
%
%Discuss Lee Segel's ``deep theorem'' idea.
%
%Talk about some of the stuff in Doyne's Rosetta Stone paper.


\section{Acknowledgments}

This work was supported by the Santa Fe Institute (Project 2050) and
by the National Science Foundation (grant IRI-9157644). John Holland
conceived of and designed the original Echo model on which our work is
based, and he suggested many of the ideas in this paper.  James Brown
and Timothy Keitt, both of the Biology department at the University of
New Mexico, have both provided many useful comments on Echo.  Brown
suggested the examination of Preston's work on species abundance. We
also thank Josh Epstein, Rob Axtell and all those involved in Project
2050 at the Santa Fe Institute, the Brookings Institute and the World
Resources Institute for their comments.

\bibliographystyle{unsrt}
\bibliography{references}

\end{document}

