% \documentstyle [nooutdent,fullpage,11pt] {article}
\documentstyle [11pt] {article}

\begin {document}
\pagestyle {empty}

\title {{\bf Thesis Ideas For ECHO}}

\author
{Terry Jones }

\maketitle

\begin{abstract}
There are several aspects of evolution that I find interesting and
would be interested in studying within {\bf ECHO}. Most of them could not
be done within {\bf ECHO} as it currently stands. What follows is a brief
outline of each with a discussion of what sorts of things would need
to be added to the model to accomodate the experiment.
\end{abstract}

\section{Arms Races}
This is a very interesting subject for me. Evolution seems to be a
huge mass of intertwined arms races. I don't know who coined the term,
but I first encountered it, I think, in "The Selfish Gene". John
Holland has reported that he observed an arms race in an early run of
his version of {\bf ECHO}, in which agent's tags became increasingly
long.

It seems to me that an arms race would not need to involve tags
becoming longer necessarily. An arms race is a chase through
chromosome space and there is no need for the direction taken to be
unidirectional. Perhaps the unidirectional ones are the simplest to
identify as arms races though. Within {\bf ECHO} it would be enough
for an agent to mutate a chromsome to make trading or comabat matching
revert to an earlier stage in the history of the run. What I mean by
this is that combat initiation, say, is determined by walking down the
different agent's tags and conditions doing an allele by allele
comparison. If an agent that would otherwise have been attacked
mutates a single locus (as opposed to increasing the chromosome's
length), it is also escalating the arms race.

With more thought, it seems that all evolutionary steps can be
interpreted as being some sort of move in an arms race game. This is
especially true when thinking of agent--agent interactions, but is
also true under some milder interpretation when thinking about
agent--environment interactions (if the environment is changing).

{\bf ECHO} could perhaps be used to study arms races in some detail. I
am not sure what could be learnt or what the explicit goal of such an
undertaking would be. I am not familiar with any arms race literature,
if there is such a thing. This would probably be a case of examining
the existing theories on the subject, picking some aspect that was the
subject of interest or controversy (or whatever) and seeing what
happened in {\bf ECHO}.

It is not clear how to encourage {\bf ECHO} to exhibit behavior that
is more easily interpreted as arms race like (by this I mean in the
John Holland example style). It is not at all clear that one would
want to do this anyway. More probably, it would be better to make a
definition of an arms race something similar to the above (a chase
through chromosome space) and develop better tools for watching such
things in the current version of {\bf ECHO}.

\section{Parasitism}

\section{Speciation And Punctuated Equilibrium}

This is a very interesting area which is of great interest to
biologists. Gould and Elderidge(?) proposed that evolution does not
always move in small steps and that major events, for example
speciation, occur rather as fast jumps. There is a large body of
literature on the subject. Some of it, the more popular stuff, I have
read. Both Gould and Dawkins have written quite a bit on the subject.

An essential question that would need to be answered before
undertaking any experiments here is: in {\bf ECHO}, what is a species?
The answer to this question is not clear. The simplest is probably
that a species is defined by equivalence classes of agents that can
mate. This does not consider what the agent is named (fly, ant etc.)
but hopefully agents with such suggestive names could not mate
initially. Note that this definition of species is not a direct
mapping onto the biological definition -- in fact there is, as I
remember, no clear cut biological definition.

Two problems come immediately to mind. One is that the chromosomes in
John's typical agents are so short that, for example, a standard fly
is only two mutations from being able to mate with a standard
caterpillar. This is not something that I have observed happen (in any
case I do not do any mating), but it would be bound to arise. The
obvious solution is to make the chromosomes longer to make this sort
of thing rarer. This is not to say that it should be made impossible,
just highly unlikely I suppose.

The problem with the obvious solution is that if mating tags and
conditions are increased in length, then an agent whose tag or
condition mutates by one position will probably not be able to mate
with any other agent, including copies of itself. This agent may not 
die as a result. Is this bad? It is at least unrealistic in a ``mate
or die'' world, but perhaps not in a world with self--replication
(like {\bf ECHO}). The length of the chromosomes (and the similarity
between the agent's mating tag and condition) would deteremine when it
was likely that this agent would find a mate.

Another point is that, as Gould presents it, speciation happens
quickly in geographically isolated populations that somehow get cut
off and in which environmental conditions are different in some
important way from those that the rest of the species are used to
dealing with. I would probably need to make some provision for some
sort of events like this to occur within {\bf ECHO}. Possibly this
could be as simple as turning off and then on migration between two
sites.

\section{Ants, Flies and Caterpillars}


\end{document}
