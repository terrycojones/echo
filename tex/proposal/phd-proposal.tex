\documentstyle [12pt] {article}

%
% relevant pages from first notebook: 99, 101.
%

\begin{document}
% \pagestyle{empty}

\title {{\bf Ph.D. Thesis Proposal}}

\author
{Terry Jones}

\maketitle

\section{Introduction}

The thesis will concern itself with a family of models of complex
adaptive systems (CAS) called the Echo models. These models
originated with John Holland, who also invented Genetic Algorithms
(GA) and Classifier Systems (CS). In what follows I shall often
refer to ``Echo'' as though it were singular. This is somewhat due to
laziness and somewhat a result of the fact that I have developed a
single Echo model from the family described by Holland. Where the
distinction is important, I shall make it.

There is no real definition of a complex adaptive system, but
everything that deserves the name has at least the following
characteristics:

\begin{itemize}

\item
The system can be thought of as a ``world'' with an ever--changing
environment. The world might be, for example, an economic market, the
body or a computer network. The world has some geography and the
environment will, in general, be different at different locations in
the world.

\item
The system is never in equilibrium. Of course, when viewed from a
great enough distance, every system will appear dead and when viewed
from close enough will appear completely chaotic. This objection should
not be taken too seriously. The intent of the first sentence is that
the system be viewed from a natural distance, something at about the
agent level. Thus, for example, in an ecosystem, agents might be
particular fauna and one could examine the system at the level of
species or at the level of individual agents. At these levels, complex
adaptive systems are always in flux.

\item
A large number (usually at least in the hundreds) of agents. The
agents are at once independent and interdependent. That is to say they
make individual decisions about how to act, but that these decisions
may be influenced by what other agents are doing or have done and the
state of the environment.

An agent might be, corresponding to the example worlds above, a
stock--broking firm or an operator on the trading floor, an antibody in
the body's immune system or a virus loose on the internet. Agents will
typically come in many shapes and sizes and will adapt to the
conditions in which they find themselves. They may ``reproduce'' in
some fashion -- perhaps through mere duplication or some form of
sexual reproduction.

\item 
Holland (Holland 93) mentions a further and ``less obvious'' feature
of such systems, that agents ``employ internal models to direct their
behavior''. The reason this is to be considered less obvious is that,
in looking at a simple agent, it is often not clear what an internal
model for that agent could possibly be. For instance, what internal
model does a given antibody in the body's immune system possess?

I shall not address this question in any detail here, or in the rest
of this proposal. Briefly, I think the answer is that at this level
one must look at the ways that such an agent behaves. The kinds of
things it does can be thought of as actions that are triggered by
conditions in the world. In this sense an agent has a model of the
world, although it is not explicit. To me this requirement (the agents
in a CAS have an internal models) is not necessary, as it is difficult
to attribute an internal model to such simple agents, and so we should
not try to impose it. Of course if it feels ``right'' to attribute
internal models to an agent, there is no reason not to do so and it
certainly would tend to increase the complexity of the system.

\end{itemize}

Echo is an attempt to capture these commonalities in a general model.
Ideally, this will allow the modeling of a diverse range of such
systems without the need for a specialized model for each to be
developed. Typically, the people who know the most about any
particular real--world complex system are not the people who can also
develop sophisticated models that can be used as tools to increase
understanding. The hope is that Echo will ameliorate this situation.

An Echo world consists of some number of sites, arranged in some
geography. Simple geographies include lines, rings and
two--dimensional arrays. The simplest possible world, which is used
exclusively in this thesis, contains a single site. One may well ask
why this decision was taken. The answer is that even the complexity
arising within a single Echo site will take an entire thesis to
describe. There is little point in examining the behavior of a
collection of interacting objects when we have enough trouble
describing the behavior of even one of these objects.

Any number of agents can exist at a single Echo site. The number will
depend on many factors, such as the richness of the site, the behaviors
of the agents at the site, site ``taxes'', and various probabilities
that are associated with the world and with the site.

Agents are composed of ``resources''. The world has a fixed number of
resource types. These are usually denoted by lower case letters
beginning with {\em a}. To say that agents are composed of resources
means that their genomes consist of strings of the resources. One
should feel free to draw analogies with DNA.

Sites produce resources, agents collect resources, fight for
resources, trade resources and return resources to the world when they
die (unless they are killed in combat, in which case the loser's
resources are taken by the winner). All interactions in the world
involve flows of resources.

There are three basic interactions in Echo worlds.

\begin{itemize}

\item Combat

\item Trade

\item Reproduction

\end{itemize}

\section{Speciation and Echo}

\section{Thesis Outline}

The thesis will contain seven chapters, the main one being the fourth,
on ``Hypothesis and Results''. Below is an outline of the contents of
each. 

\begin{enumerate}

\item Introduction

The introduction will discuss the connection between computer science
and biology -- how they interact and the benefits that they derive
from each other. The Echo model will be briefly introduced. The major
contribution of the thesis will be outlined. The purpose of this sort
of modeling will be explained and grounded firmly in the real world.
There will be some discussion of complex systems in general -- what
they are, how we try to understand them, etc. I will describe some
other approaches to understanding complex systems, including through
modeling.

\item Echo Models

This section will describe the Echo model in detail. Currently there
are only a few written descriptions of Echo (Holland 92a, 92b, 93),
and of these, only one could be considered widely available (Holland
92).

Echo actually refers to a family of models. Here I will make clear
what the model of Echo used in the thesis was and how it differs from
the models Holland has used. All differences and similarities will be
presented, and should enable future researchers to implement an Echo
model that closely matches the one for which the results are
described.

\item Species and Speciation

This chapter will present some discussion on real world speciation. It
will outline current problems and theories. The concept of speciation
needs to be defined, and needs to be defined with respect to Echo
models as well. The various theories about how speciation occurs in
the real world will be described.

\item Hypothesis and Results

The basic hypothesis that I wish to test in Echo is the following:

{\em That preferential mate selection based on evolving phenotypic and
genotypic traits can lead to pre--zygotic isolation between
sub--populations, and hence to species.}

A rough explanation of this is as follows. Suppose that mate selection
for sexual reproduction is based upon something observable about the
opposite sex. This, of course, is not a wild supposition. If there is
both variety in preference and in the characteristic, then it is
highly likely that two individuals of the same sex will feel attracted
to members of the opposite sex in differing amounts. If the
characteristics that determine both appearance and the strength of
attractions are allowed to evolve, one can imagine reaching a
situation in which the population is divided into sub--populations, or
equivalence classes, which are defined by mating choices. Under the
standard biological definition of species, each of these
sub--populations is a species.

Pre--zygotic refers to the fact that the species boundary is something
that is imposed {\em before} sperm meets egg. In this case, no sperm
never reaches an egg as the individuals do not choose to have sex. On
the other hand, post--zygotic isolation between species occurs when
sex occurs but offspring either do not result or are sterile. The
best known example of post--zygotic isolation is that of the donkey.

The questions that I am addressing in my research with Echo is the
following. Can this sort of speciation occur? If it happens, will the
isolation persist (as it does in the real world)? If not, what is
required to make speciation occur? This is also a very important
problem. The most widely accepted theories of speciation postulate
that some geographical isolation between sub--populations is needed
for speciation to occur. This is known as {\em Allopatric Speciation},
and there are two main variants of it. {\em Dichopatric Speciation} --
in which two large populations somehow become separated (this is also
known as the {\em dumbell} model), and {\em Peripatric Speciation} --
in which a small founder population is established outside the range
of the original population (Mayr 1992).

If speciation due to sexual preferences and the evolution of ``mating
tags'' does not occur in Echo, does it occur when sub--populations
experience geographical separation? How long does the separation need
to last? If the sub--populations are reunited, will the species
boundary persist?

\item Connection With Traditional Genetic Algorithms

Recently there has been quite a bit of interest in speciation within the
genetic algorithm (GA) community. Most of this work is not explicitly
concerned with speciation. The aim of this sort of thing is not to
model the real world (and hence the term ``speciation'' is not often
encountered), but to make GAs perform better by increasing the
diversity of the solutions they discover.

Some people have done this explicitly (Todd et. al. 1991)

\item Related Work

\item Conclusion

\end{enumerate}

\section{Time Frame}

I have been developing an Echo model since July 1992. John Holland has
provided a lot of input and it is expected that he will serve on my
committee in some capacity, probably as an external co--chairman.

As far as code development goes, there is not a lot left to do. Some
specialized tools need to be written to help with the analysis of
species, but the biggest hurdle has been safely cleared. I plan to
complete the thesis by the end of August 1993. The immediate goal is
to begin experimentation on the species formation question and to
continue broad reading on the subject.

\subsubsection*{References}

Holland,~John~H. (1992a). \\
{\em The Echo Model},
in ``Proposal for a Research Program in Adaptive Computation''.
Santa Fe Institute, July 1992. \\

Holland,~John~H. (1992b). \\
{\em Adaptation in Natural and Artificial Systems}, 2nd Ed.,
MIT Press, 1992. \\

Holland,~John~H. (1993). \\
{\em ECHOING EMERGENCE : Objectives, Rough Definitions, and
Speculations for Echo-class Models},
To appear in ``Integrative Themes'', George Cowan, David Pines and
David Melzner Eds. Santa Fe Institute Studies in the Sciences of
Complexity, Proc. Vol XIX. Reading, MA: Addison--Wesley 1993. \\

Mayr,~E. (1992). \\
{\em Towards a New Philosophy of Biology}. \\

Todd,~Peter~M. and Miller,~Geoffrey~F (1991). \\
{\em On the Sympatric Origin of Species: Mercurial Mating in the
Quicksilver Model} in ``Proceedings of an International Conference on
Genetic Algorithms'', Morgan--Kaufmann.

\end{document}
