\section{Ecological Considerations}
\label{species-abundance}

Does the model described in the previous sections behave similarly to
natural ecological systems?  Comparing Echo with properties of ecological 
systems might help us answer this question; some such properites are listed 
below:
\begin{itemize}
\item Stable patterns of interactions among variants, or trophic networks,
such as the ant-fly-catepillar triangle described in [REFERENCE], or
[CITE YALE GROUP].
\item Different levels of resources, i.e., what happens when the
amount of resources pumped into the system at each time step is
increased or decreased?  
\item Relative species abundance (trajectories, duration,
distributions).
\item Community assembly patterns.
\item Cataclysmic events such as meteors, droughts, etc.
\item Isolation effects---what happens when a number of sites are
allowed to evolve independently (with no migration) and then
the sites are connected (i.e., migration is enabled).
\item Flows of resources---are there resource sources and sinks?  How
quickly do resources flow through the system?  With what are rates of
flows correlated?
\item Transitions from single-cellular to multi-cellular organization
(this last question requires certain extensions to the Echo model
described in Section \ref{echo-overview}.
\end{itemize}
We have examined several of these questions informally by observing
running Echo models.  In these studies, we start Echo either with a
population of arbitrarily chosen [randomly constructed] agents or 
(more typically) with a fixed collection of agents (e.g., that implement a 
trophic triangle).

[a paragraph here on the Yale work.]

In the remainder of this paper, we consider one of these properties in
detail---species abundance patterns.  The experiments described in the
following two sections show how species abundance patterns can be used
to test whether or not Echo exhibits ecologically plausible behavior.
We use two techniques to test the hypothesis that diversity patterns
in Echo resemble those of natural systems: quantitative and
qualitative expectations and a neutral model.  To make a falsifiable
comparison with biological systems, we consider two ecological
patterns, each of which has similar quantitative properties in many
different ecosystems---the Preston curve and the species-area scaling
relation.  This allows us to compare Echo diversity patterns with
empirical patterns of species abundance.  In the case of the Preston
curve, careful quantitative comparisons are problematic for
several reasons, so we report more detailed results for the
species-area scaling relation.  Because evolution of agent genotypes
is the mechanism of principle interest in Echo, we define a neutral
model that removes specific selective pressures and compare
its behavior with that of the original Echo model.

%In the introduction, we discussed difficulties in modeling CAS, and
%relative merits of Echo as a model of CAS.

% \subsection{Species Diversity}
 
Species diversity in ecological communities is often considered the
single most interesting ecological variable
\cite{Hutchinson59,May75,May86,Brown81}.  One reason for this interest
is the overwhelming number of species found on Earth; approximately 30
million species live today, and a great many more have lived and
suffered extinction \cite{Brown94,Raup91}.  Another reason is that
species diversity is believed to play a role in the stability of
ecosystems to perturbations or invasions.  Ultimately, the science of
ecology seeks means for predicting species diversity, by identifying
those mechanisms which regulate it, for example \cite{Brown81}:
\begin{quote}
  One of the greatest remaining challenges in biology is to explain
  the diversity of living things.  What determines the number and
  kinds of animals, plants and microbes that live together in one
  place?  What causes variation in diversity from one place to
  another?  What accounts for changes in the abundance and identity of
  species over time?  How do individual species contribute to the
  diversity and stability of the natural world? (p. 877)
\end{quote}
 
A related property of ecological communities is the relative
abundances of species.  Consider a sample of vegetation from a
semi-arid desert or tropical forest.  What is the distribution of
individuals into species?  Many factors will contribute to this
distribution.  These include ecological factors (e.g., the presence or
absence of various predators or mutualists, and relative population
sizes and distances), environmental factors (e.g., temperature,
availability of precipitation, and soil texture and quality), and
experimental factors (e.g., size of sample, scale of observer, and
location of the sample).  Ecologists have performed such sampling
countless times from many biota.  One purpose of such
sampling is to describe the way $N_{tot}$ individuals present are
partitioned into $S_{tot}$ species, and in particular, to determine
whether the distribution of species fits a mathematically distinct
pattern.  If it does, then it is relevant to ask what biological
processes could produce this pattern.  Such analyses help to identify
processes that regulate species diversity.

\subsection{The Preston Distribution}
\label{preston-distribution}

When sampling experiments are performed on natural ecosystems, a
common result is that most species sampled have few representatives;
that is, most species are rare, but a few are common
\cite{Preston62a,May75,Brown81,Brown94}.  Preston's canonical
log-normal distribution is the most widely accepted formalization of
the relative commonness and rarity of species.  When species counts
are plotted on linear axes, the distribution is unimodal and
right-skewed with an extremely long tail.  Because it is impossible to
observe less than a single individual representing a species, these
distributions are truncated on the left at what Preston called the
``veil line.''  When plotted on a log-transformed $x$-axis, the
distribution above the veil line closely resembles a normal
distribution.  Preston observed this and postulated that species
counts follow log-normal distributions.

To construct these curves, Preston counted abundances of sampled
species and grouped them into a series of ``octaves''
\cite{Preston48}.  An octave is simply a logarithmic (base 2) grouping
of species abundances.  His octaves were labeled ``$<1$'', ``1--2'',
``2--4'', and so forth.  Preston plotted octaves on the $x$-axis and
the number of species per octave, a frequency of frequencies, on the
$y$-axis. If a species abundance fell within octave boundaries, it
counted as one species in that octave. If an abundance fell on the
boundary between octaves, (as will any value which is a power of 2),
one-half was counted in the two neighboring octaves.

Preston constructed these curves for a number of biota,
and found their general shape was well approximated by a Gaussian
(normal) distribution of the form $$ y = y_0 e^{-{(aR)}^2} $$ where
$y$ is the number of species falling into the $R^{th}$ octave left or
right of the modal octave, $y_0$ is the value of the mode of the
distribution and $a$ is a constant, related to the logarithmic
standard deviation ($\sigma$) such that $a = (2\sigma)^{-\frac{1}{2}}$
\cite{Preston48,MacArthurAndWilson67}.  In particular, Preston found
that the value of $a$ tended to be in the vicinity of $0.2$.  This
observation gave rise to the ``canonical'' log-normal distribution 
\cite{Preston62a,Preston62b}.  In the canonical distribution, the
general log-normal distribution is reduced to a family of log-normal
distributions with similar variance. This relationship makes it
possible to predict relative species abundances given only the
number of individuals or the number of species
\cite{Preston62a,Sugihara80}. 

[INSERT HERE pp. 13-18 (edited down) and data from Forrest/Jones Echo paper
---possibly in the results section (PTH prefers that stuff goes in results)]

Several considerations lead us to question whether the log-normal
distribution is an important consequence of identifiable ecological
processes or whether it is simply the expected outcome of a random
process with certain properties.  It is well-known that
log-normal distributions often arise when random processes are the
outcomes of multiplicatively compounded subprocesses
\cite{AitchisonBrown57,May75,May86,Hogg90a,MontrollSchlesinger82,Sugihara80}.

%Could the constraint on observed variance also result from
%non-biological causes?  [STEPH/pETER: what are we really trying to say
%here?]  
% In Section \ref{neutral-model}, we address this question by
% using a neutral model to study species diversity.  But, it didn't
% tell us much

Randomly compounded subprocesses can account for log-normal
distributions in general, but they fail to account for Preston's
``canonical'' log-normal, in which distributions always have similar
variance.  A number of steps used to construct Preston curves may
limit species abundance distributions to the canonical family.  For
example, the procedure of splitting a count between octaves forces an
inflection of the curve about a mode of one.  In the case of
singletons (species with only one representative), half of the counts
are assigned to the octave labelled ``$<1$'', and half of the counts
are assigned to the ``1--2'' octave.  Species with 2, 3, or 4
representatives will also fall into the ``1--2'' octave, so this
octave will generally have a greater value than the ``$<1$'' octave.
Thus, the observation of an inflection near the veil line is caused by
the binning procedure used in constructing the histogram.  If an
alternative procedure were used, we might not observe an inflection,
and perhaps not infer a unimodal distribution.  Other researchers have
indicated that fitting a specific nonlinear function to discretized
data \cite{Pielou77}, as well as sampling and histogram binning procedures
\cite{May75,May86} complicate the task of testing formally for a
canonical log-normal distribution.  All of these complications make it
difficult to explain what mechanisms cause the Preston distribution.

%[MOVE THIS SENTENCE.  IT'S REDUNDANT HERE.]  Do ecological processes
%or chance alone produce the canonical log-normal distribution?  This
%is one question we will address using a neutral model to study species
%diversity.  [PETER: how does the neutral model address this question?]

Whether or not a meaningful statistical test can be performed,
describing the relative abundaces of species is an ecologically
meaningful activity, and serves as an important first check of how
Echo compares with empirical distributions of abundance.  Our data
show a good qualitative resemblance, under a wide range of Echo
operating conditions, as shown in Figure \ref{fig:preston-curves} and
\cite{ForrestAndJones94}  [PTH says this statement should be moved into the
results section].  However, we believe that a more meaningful
test can be performed by studying a related phenomenon---the relation of
species richness ($S_{tot}$) to size of habitat (or sample).  This
is a better way to compare Echo with empirical systems for two
reasons: the species-area function is linear (and therefore much
easier to identify than a Preston distribution), and the slope of the 
line should match the value observed in nature.

\subsection{The Species-Area Scaling Relation}

As more individuals are collected, more species will be found.
Preston observed that doubling the number of individuals sampled will
``unveil'' another octave in the species curve by moving all species
counts one bin to the right \cite{Preston48,Preston62a}.  One way to
collect more individuals is to increase the area of the sample.  This
can be achieved either by increasing sample size, or by
sampling from similar habitats of increasing area.  Conveniently,
islands in an archipelago provide ecologists with natural sampling
units of varied size, all with similar climate and terrain. 

When species richness is tallied for islands of increasing area, the
following scaling relation holds:  $S_{tot} = cA^z$, where $S_{tot}$
is the total number of species, $A$ is the area of the habitat (e.g.,
an island or mountaintop), and $c$ and $z$ are regression constants.
For empirical ecological communities with a canonical log-normal
distribution of abundances, $z \approx \frac{1}{4}$.  This relation and the
value for $z$ can be predicted from the Preston distribution
\cite{Preston62a,May75,Pielou77} and is well supported empirically
\cite{Preston62a,MacArthurAndWilson67,DurrettAndLevin95}.

The species-area scaling relation is an important ecological
pattern.  The theory of island biogeography theory is predicated
on species-area relations \cite{MacArthurAndWilson67}.  In
conservation biology, this relation has been used to predict the
effects of different reserve sizes on species diversity.  Although the
species-area relation can be derived from Preston's canonical
log-normal distribution, a satisfactory explanation of the processes
that regulate this relation has not been advanced
\cite{DurrettAndLevin95}.  Exceptions to the species-area relation can be 
found (see, e.g., \cite{Preston80}), which help identify mechanisms
which shape the species-area curve.  For instance, a value for $z$
larger than $\frac{1}{4}$ is typically found when sampling at the continental
scale.  This is taken to reflect the effect of greater habitat
heterogeneity represented by vast regions of complex terrain
\cite{MacArthurAndWilson67}.

%[STEPH: do we need the rest of this 
%paragraph?] Trivially, the value for $z$ must range between 
%0 and 1.  The case of $z=0$ implies no turnover in species with
%increasing area, while $z=1$ implies that doubling area will double
%$S$, or that there is no redundancy of species with increasing area
%\cite{May75}.

The Preston curve and species-area scaling relation are two relatively
robust ecological phenomena.  The next section describes experiments
that ask whether Echo exhibits these phenomena, both for the original
and the neutral versions of Echo.  The species-area curve is our
primary focus for formal quantitative comparision, for the reasons
just discussed, but we include some data on Preston distributions for
comparison.  An earlier study \cite{ForrestAndJones94} examined more
carefully whether Echo populations resemble Preston distributions but
did not use a neutral model.


