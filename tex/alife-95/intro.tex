\section{Introduction}
\label{intro}

Many interesting systems are difficult to analyze and control using
traditional methods.  These include natural ecosystems, immune
systems, cognitive systems, economies and other social organizations,
and arguably, modern computers.  One source of difficulty arises from
nonlinear interactions among system components.  Nonlinearities can
lead to unanticipated emergent behaviors, a phenomenon that has been
well documented and studied in physical, chemical, biological, and
social systems [CITE TAYLOR, ALife II?], as well as in some forms of
computation \cite{Forrest91b}.  Nonlinear systems with interesting
emergent behavior are often referred to as {\em complex systems}.  A
second form of complexity arises when the primitive components of the
system can change their specification, or evolve, over time.  Systems
with this additional property are sometimes called {\em complex
adaptive systems\/}
\cite{Holland95a}.  Here, we will use the term {\em complex adaptive
system\/} (CAS) to refer to a system with the following properties:
\begin{enumerate}
\item A collection of primitive components, called ``agents.''
\item Interactions among agents and between agents and their
  environment.
\item Unanticipated global behaviors that result from the
  interactions.
\item Agents adapt their behavior to other agents and environmental
  constraints.
\item System components and behaviors evolve as a consequence (4).
\end{enumerate}

% There are also all the usual problems of
% verifying that complex simulation code is correct.  

It can be quite difficult to model systems with these properties
analytically.  Useful and predictive mathematical treatments are
difficult, due primarily to the following: nonlinearities (as
discussed above), discreteness, spatial inhomogeneities, and the
changing behavior of the primitive elements of the system.
Discreteness arises, for example, in time (as in the case of
generations in population genetics), state spaces, and internal
variable values, while spatial heterogeneities arise through resource
gradients, nonuniform operating conditions (e.g., different mutation
rates in different parts of the body [cite Kepler and Perelson], or
even random drift.  Both discreteness and spatial heterogeneity can
have a significant effect on system behavior.  For example, Durrett
and Levin \cite{DurrettAndLevin93a} present an elegant description of
how both both properties affect the predictive power of conventional
methods, including ordinary differential equations and
reaction-diffusion systems, on a classical problem in ecology.  As a
consequence, many of the standard approximations for infinite-sized
systems and techniques developed for studying asymptotic behavior of
continuous nonlinear dynamical systems cannot be directly applied to
discrete or spatially heterogeneous systems.  Finally, adaptation is
central in CAS\@.  Because the primitive components of the system can
change over time, different agents may behave according to different
rules at different times.  Individual variants are important and can
determine the overall system trajectory, which precludes modeleing
only their aggregate behavior.  Although the underlying evolutionary
mechanisms themselves can in principle be modeled, as suggested in
\cite{Farmer90}, this is a difficult undertaking, and most such
efforts to date have been notably unsuccessful in describing the
behavior of any particular evolutionary system.
[...this is a difficult undertaking and this level of detail is
usually omitted when modeling complicated systems.  Further, outcomes
of stochastic processes or ``frozen accidents'' can have irreversible
effects on system dynamics \cite{GellMann94,GellMann95}.  Omit rest
of paragraph...PTH] 
For example, analytical models in population genetics typically
describe two-locus systems at equilibrium, often assuming infinite
population sizes, while the cases in which we are interested involve
relatively small populations with richer interactions (i.e., longer
genomes).

An alternative to analytical models of CAS is simulation.  Detailed
simulations are also problematic, because it is often impossible to get
all of the details correct.  Consider, for example, the vertebrate
immune system which has been estimated to express over $10^7$
different receptors simultaneously [STEPH: REFERENCE].  Modeling the
physical chemistry of just one receptor/ligand binding event, even at
an abstract level, requires enormous amounts of computation, and it is
therefore infeasible to model the expressed repertoire of receptors
precisely.  This problem exists for many large complicated systems,
but because nonlinear systems can be highly dependent on seemingly
small details, even a trivial inaccuracy in the model could lead to
wildly erroneous results.  One approach to this problem is to strip
away as much detail as possible, retaining only the essential
interactions.  The goal is then to develop models whose behavior is
robust with respect to the details of the interactions (e.g., avoiding
parameter tweaking to coax a system to produce desired behaviors), and
which produces the broad categories of behaviors in which we are
interested.  An implication of this approach is that such models will
rarely, if ever, be able to make precise quantitative predictions.
The model described in this paper, called Echo, is an example of this
approach, as are many artificial-life models.

Echo is a mechanistic model in the sense that it encodes (as a
computational artifact) a theory about which mechanisms are most
relevant in ecosystems.  In Echo, certain primitive components and
interactions are built in, and when the model is ``run,'' i.e.,
simulated, these mechanisms give rise to various macro-level
properties.  The goal is that the relevant behaviors will arise
spontaneously as a consequence of the primitive mechanisms.  This is
quite a different kind of explanation than simply predicting what will
happen next without representing the underlying mechanisms explicitly.
Of course, prediction is important for models like Echo, but it is
quite a different kind of prediction than that typically associated
with simulations.  In this paper, we use the word ``model'' to
describe the design decisions we have made about which components and
interactions are included and which are not.  This might be confusing
because the model is described in terms of computational structures,
like ``agent,'' ``stack,'' and ``rules.''  Our model is really a
high-level simulation of generic ecological behavior.

%This style of modeling is quite different from the
%differential-equation style of models used most frequently to model
%nonlinear dynamical systems.  In agent-based models, each ``actor''
%and each interaction among actors (i.e., not just each type of
%interaction) is represented (simulated) explicitly.  Individuals are
%capable of quite different kinds of behaviors (the agents in the
%system are heterogeneous).  Agent-based models are discrete in most
%dimensions, typically time, state, and update rules.  As a result,
%these computational models are difficult to analyze.

Echo is related to several earlier CAS models.
Genetic algorithms \cite{Holland92} focus on the evolutionary
component of CAS\@.  They are reasonably well understood and mature,
but ignore several important features, including resource allocation,
heterogeneity, and endogenous fitness.  Classifier systems
\cite{HollandEtAl86} apply genetic algorithms to a cognitive modeling 
framework.  Similarly, Echo extends genetic algorithms to an
ecological setting, adding the concepts of geography (location),
competition for resources, and interactions among individuals
(coevolution).  Echo is intended to capture important generic
properties of ecological systems, and not necessarily to model any
particular ecology in detail.  Echo's contribution to ecological
modeling lies in the fact that evolution is built in as a fundamental
component of the system.  Most existing ecological models 
\cite{May74,Caswell89,DeAngelisAndGross92} do not have such an evolutionary 
component.  Thus, it is difficult to address issues such as flows of
information and resources in many-species assemblages, patterns
of speciation and extinction in ecological communities, and effects
of spatial heterogeneity on population dynamics.

Echo also resembles more recent CAS implementations.  These include Swarm
\cite{Langton94}, Sugarscape \cite{Epstein94}, and the 
Evolutionary Reinforcement Learning (ERL) model
\cite{AckleyAndLittman92}.  Echo describes a
family of models, but it is not a generic modeling platform like
Swarm.  Swarm supports a wide range of agent types and interaction
rules, but Echo makes specific commitments about the form of system
components (agents, resources, interactions).  Echo does resemble
Sugarscape in several respects, but it differs in specific details
(e.g., the complexity of each individual agent) and in its focus on
ecological principles.  ERL provides two levels of learning (there is
only one in Echo) and is not intended as a general ecological model.
[GET ACKLEY TO HELP HERE.]  

%Echo extends classical genetic algorithms
%in several important ways: (1) fitness is endogenous, (2) individuals,
%called agents, have both a genome and a local state (a reservoir of
%stored resources) that persists through time, and (3) genomes have
%variable length.

The original conception of Echo, including motivation, design
decisions, and overall structure were introduced in
\cite{Holland92,Holland94,Holland95a}.  In these works, a family of
progressively more elaborate Echo models is outlined.  Our goal in
this paper is to describe more fully one specific Echo model and to
show how one might study the extent to which Echo does or does not
capture important properties of ecological systems.  
Sections \ref{echo-overview}, \ref{echo-cycle}, and \ref{echo-interactions} 
describe the Echo model we have implemented.  An important aspect of
our study is the idea of a neutral model.  
Neutral models are useful for testing what mechanisms are necessary to
produce an observed pattern \cite{NiteckiAndHoffman87}.  We propose a 
neutral model that allows us to assess the effect of evolutionary pressures on
Echo's behavior, and we compare its behavior with that of our original
Echo implementation.  Sections \ref{species-abundance} and
\ref{results} report experiments (with and without the neutral model)
on the relative abundance of species in Echo, a characteristic feature
of ecological systems.  This raises some fundamental questions, such
as how to define species in Echo and what scientific purpose is served
by such models, which we also discuss.
