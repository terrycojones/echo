\section{Interactions in Echo}
\label{echo-interactions}
\label{agent-agent}

There are three main forms of agent-agent interaction: combat, trading
and mating. All of these interactions take place between agents
that are located at the same site and all involve the transfer of
resources between agents

\subsection{Combat}
\label{dynamics:combat}

Combat is an idealization of any antagonistic interaction between
real-world entities. It does not imply that the agents are actually
fighting, although this is not precluded. If two agents in a
real-world system are behaving in a competitive fashion, this would be
modeled in Echo by designing the agents in such a way that they would
engage in combat. When combat occurs, one agent is always killed, and
its resources are given to the survivor.\footnote{In more recent formulations
of Echo \cite{Holland94,Holland95a}, the interaction need not be so
extreme and results in a transfer of resources (possibly in both
directions, and possibly in a very uneven fashion) between the agents.}

When two agents meet, it is first determined whether or not either
agent will attack the other.  An agent $A$ will attack an agent $B$ if
its combat condition is a prefix of $B$'s offense tag.  [TERRY: is
this right?  SHouldn't it be B's defense tag?]  an agent is given a
chance to flee (which it does with a probability equivalent to the
probability of it losing in the combat encounter).  The calculation of
the probability of victory in combat is based on matching $A$'s
offense tag with $B$'s defense tag and vice versa. The resource
characters that comprise these genes are used as an index into a {\em
combat matrix}, with special provisions for zero-length genes and for
genes of unequal length.  This is somewhat complicated and is not
described fully here.

As a result of this computation, each agent receives some number of
points. If $\alpha$ and $\beta$ are the points awarded to $A$ and $B$,
then $A$ will win the combat with a probability of $\alpha / (\alpha +
\beta)$.  The resources that comprise the loser (both its genome and
the contents of its reservoir) are given to the winner and the loser
is removed from the population.

\subsection{Trade}
\label{dynamics:trade}

If two agents are chosen to interact and they do not engage in combat,
they are given the opportunity to trade and mate. Unlike combat,
trading and mating must be by mutual agreement. Agents $A$ and $B$
will trade if $A$'s trading condition is a prefix of $B$'s offense tag
and vice versa. Notice that the offense tag is used here as well as in
determining whether combat will occur.

When trade takes place, each agent contributes its excess trading
resource.  Excess is defined to be the amount of resource that an
agent possesses above that which is required to replicate its genome,
plus some reserves (world-level parameters control how much reserve an
agent retains).  Thus an agent provides some fraction of the resource
that it does not need for the next self-reproduction. This may be
zero, in which case an agent does not provide anything in the
trade. This behavior is analogous to a form of deception or bluffing.
An agent cannot know in advance if another agent will supply a
positive quantity of a resource, or what that resource might be. This
might seem to be an odd form of trade, but agents can learn to
recognize each other based on their trading tags. Agents whose tags
tend to involve them in disadvantageous trades will tend to reproduce
less quickly and tend to have smaller probabilities of being able to
meet taxation demands.

\subsection{Recombination}

Agents that interact and do not engage in combat may exchange genetic
information through recombination.  As in many genetic algorithms, the
new agents replace their parents in the population.  Recombination
occurs between two agents $A$ and $B$ if $A$ finds $B$ acceptable and
vice versa. $A$ will find $B$ acceptable if (1) $A$'s mating
condition is a non-zero prefix of $B$'s mating tag or (2) both $A$'s
mating condition and $B$'s mating tag are zero length. The restriction
to non-zero prefixes is designed to stop agents with zero-length
mating conditions from rapid proliferation. Such an agent finds all
other agents desirable (including copies of itself). To prevent this,
an agent with a zero-length mating condition will only find an agent
with a zero-length mating tag acceptable. This is a slight departure
from the description of mating given in
\cite{Holland92}.  Figure~\ref{fig:agent-agent} shows a simplified
view of the two-way matching process used to determine whether mating
will occur.

\begin{figure}[htb]
\begin{center}
\leavevmode
\psfig{figure=figures/agent-agent.ps}
\caption{A simplified view of the two-way tag and condition matching
that is used by agents to determine whether mating will occur.
\label{fig:agent-agent}}
\end{center}
\end{figure}

When sexual reproduction does occur, a form of two-point crossover is
employed. This is complicated by the fact that genomes are variable
length; one can choose a crossover point in one agent but find that
the same crossover point does not exist in the other.  Briefly, a
recombination proceeds by (1) selecting two genes to contain crossover
points, (2) choosing crossover points in each gene in each agent, and
(3) crossing over in the manner of two-point crossover. The operation
conserves resources (i.e., resources are not created or destroyed) but
the ratio of genetic material from each parent in each of the children
will typically not be 50:50.

