\section{An Overview of Echo}
\label{echo-overview}

Echo was designed to capture the essential features of ecological
systems.  All of the entities and interactions in Echo are highly
abstract, and it is not yet known whether Echo can be used to model
real-world phenomena directly.  Many CAS can be viewed as ecologies
(e.g., \cite{Huberman91},TIERRA?), but our focus in this paper is on
the analogy with natural ecologies.  

In Echo, an agent replicates (makes a copy of itself, possibly with
mutation) when it has acquired enough resources to copy its
genome.  The local state of an agent is exactly the amount of these
resources it has stored and its location in the world.  Agents acquire
resources through interactions with other agents (combat or trade) or
from the environment.  This mechanism for endogenous reproduction
is much closer to the way fitness is assessed in natural settings than
conventional fitness functions in genetic algorithms.

Along with these extensions to the evolutionary component, Echo
specifies certain structural features of the environment in which
agents evolve.  Specifically, there is a two-dimensional grid of
``sites'' and each agent is located at a site, although it is possible
for agents to move between sites.  There are usually many agents at
one site, and there is a linear neighborhood within a
site.\footnote{This definition of locality within a site may be
changed in future Echo implementations.}  Each site can produce
renewable resources.  Resources are represented by different letters
of the alphabet, and genomes are constructed from the same letters.
The number of resources in an Echo world is typically small. These are
denoted by lower-case letters: {\em a}, {\em b}, {\em c\/} and so
forth.  Resources can exist as part of an agent's genome, as part of
an agent's local state (in its reservoir), or free in the environment.

There are three forms of interactions among agents: trade, combat, and
mating.  In trade, resources stored internally are exchanged; in
combat, all resources (both genetic and stored) are transferred from
loser to winner; in mating, genetic material is exchanged through
crossover, thus creating hybrids and providing a primitive form of
sexual reproduction.  Mating, together with mutation during the
replication process, provides the mechanism for new types of agents to
evolve, as shown in Figure~\ref{fig:agent-evolution}.  Resource
constraints provide the pressure for agents to diversify and occupy
new niches.

\begin{figure}[htbp]
\begin{center}
\leavevmode
\psfig{figure=figures/agent-evolution.ps}
\caption{Operations introducing genetic change in Echo agents.
\label{fig:agent-evolution}}
\end{center}
\end{figure}

In each Echo run there is a fixed number of resource types which is
determined by the user of the system. These can be representative of
resources in a real-world system, or can correspond to a more abstract
notion of something that is required to ensure survival.  For example,
the environment can be designed to require that agents possess a
certain resource, which some agents can only obtain through trade. In
this situation, the resource need not be thought of as corresponding
to a physical entity, but as something that requires a certain type of
agent-agent interaction for agent survival. 

The following sections describe Echo in more detail. Much of this is
devoted to describing agents and the interactions that can occur, both
between pairs of agents and between an agent and its environment.
