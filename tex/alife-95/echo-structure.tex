\section{Echo Structure}
\label{echo-structure}

[EVERYONE: this section is pretty boring---how hard to break out the
``structure part'' and include some details about the implementation?]

This section describes the details of the structural hierarchy of
Echo.  Each run of Echo involves a {\em world\/} that contains a fixed
number of {\em sites}.  Each site may contain an arbitrary number of
{\em agents}, including zero.  These components are designed by the
user, typically as an abstraction of some aspect of a real-world
CAS\@.  In the Echo world used in this paper, there are four
resources.  The use of Echo requires decisions about the structure of
all these objects and the ways they will behave when the result is set
in motion.

\subsection{Echo Worlds}
The design of an Echo world requires the choice of a geography of
sites. In the current implementation, this geography must form a
rectangular array, although the general class of Echo models makes no
such restriction. Various properties of these sites must be
specified by the user, including the following:

\begin{itemize}
\item The number of resource types that will exist in the world.

\item The {\em interaction fraction\/} determines the number of
  agent-agent interactions that will occur at a site in each Echo
  cycle. This is determined by multiplying the interaction fraction by the
  population at a site. 

\item The {\em trading fraction\/} determines what proportion of an
  agent's excess trading resource it will exchange in a trading
  interaction with another agent. Trade is described in
  section~\ref{dynamics:trade}. 

\item The {\em taxation probability\/} is the probability that an
  agent is taxed in an Echo cycle (see Section~\ref{dynamics:taxation}
  for details). 

\item The {\em neighborhood\/} determines the possible directions of
  migration between sites. The three possibilities, given a
  rectangular array of sites, are none (i.e., no migration), the four
  compass points (used in the results described in this paper) or the
  compass points plus the diagonals.
\end{itemize}

An Echo world also requires the definition of a {\em combat matrix}
(see Section~\ref{dynamics:combat} for an overview and
\cite{Holland92} for the detailed description).  This matrix is used
in the calculation of combat outcome probabilities.  
Table~\ref{tab:simulation-parameters} shows typical values for
these paramters.

\subsection{Echo Sites}

An Echo site initially contains an arbitrary number of agents.
These are arranged in a one-dimensional array. The probability that a
pair of agents will be chosen to interact falls off exponentially with
increasing distance between agents in this array. The user must decide
which agents initially reside at each site, and in what order they
should appear in the array.  There are three parameters that must
be specified for each site:
\begin{itemize}
\item The {\em mutation probability\/} determines the
probability that the genome of
an agent is mutated during self-reproduction.  This is a per-gene
probability (see Section \ref{agents}).  Mutation is
handled slightly differently for different genes (see Section
\ref{self-reproduction}).
  
\item The {\em crossover probability\/} is used similarly to a genetic
algorithm.  It determines the probability that two agents that mate
will be subject to crossover.  If crossover is not performed, the
agents are unchanged.

\item At the end of an Echo cycle, there is a (typically small)
probability of each agent being killed.  This is called the {\em
random death probability}.
\end{itemize}

In addition to these parameters, every site must specify four
resource-level vectors. Each of these consists of a vector of
non-negative integers whose length is the same as the number of
resource types in the world.  These vectors are used as follows:
\begin{itemize}
\item The {\em production\/} vector contains the number of resources
  of each resource type that will be produced by the site in each Echo
  cycle.

\item The {\em initial\/} vector contains the number of resources of
  each resource type that will be present at the site when it is first
  created.

\item The {\em maximum\/} vector contains the maximum level of each
  resource type that can accumulate at the site.

\item The {\em maintenance\/} vector contains the number of each
  resource type that will be charged to an agent when it is taxed, as
  described in section~\ref{dynamics:taxation}.
\end{itemize}

\subsubsection{Taxation}
\label{dynamics:taxation}

In each cycle, sites exact a resource tax from their agents with a
given (worldwide) probability. This can be thought of as economic
taxation, or as the cost required to live at the site.  Biologically,
taxation can be thought of as metabolic cost. Agents that are unable
to pay the tax from their current store of resources in the reservoir
are killed, and their resources are returned to the environment.

\subsection{Echo Agents}
\label{agents}

\begin{figure}[htbp]
\begin{center}
\leavevmode
\psfig{figure=figures/example-agent.ps}
\caption{The structure of an Echo agent. Tags are visible to the
outside world. Conditions and other properties are not.
\label{fig:example-agent}}
\end{center}
\end{figure}

Agents have a genome that is roughly analogous to a single chromosome
in a haploid species, as shown in Figure~\ref{fig:example-agent}. The
chromosome has $r + 7$ genes, where $r$ is the number of resources in
the world.  Six of these, the {\em tags\/} and {\em conditions\/} are
composed of variable-length strings of resources (i.e., of the
lower-case letters that represent resources).

Tags are genes that produce some externally visible feature of the
phenotype, e.g., color.  Conditions are genes that cannot be directly
observed by other agents.  They encode internal preferences,
behavioral rules, etc.  Thus, one agent will interact with another on
the basis of its own internal conditions (rules for interaction) and
the other's external tags (appearance).  This allows the possibility
of quite complex interactions between agents, including mimicry,
bluffing, other forms of deception, and some intransitivities.  For
example, in mimicry, an agent can appear dangerous but actually be
unwilling to fight.  An example of intransitive combat relationships
would be the following: an agent $A$ might always attack an agent $B$,
and $B$ always attack $C$, but it does not follow that $A$ will attack
$C$. This has obvious parallels in real-world systems (e.g., in food
webs).  See \cite{Holland92,Holland93b,Holland95a} for discussions of
the significance of these kinds of interactions.

The six (external) tag and (internal) condition genes possessed by
every agent are the {\em offense tag}, {\em defense tag}, {\em mating
tag}, {\em combat condition}, {\em trade condition\/} and {\em mating
condition}.  These genes are used to determine what sort of
interaction will take place between a pair of agents, and what the
outcome will be.  The use of these genes is described below. It should
be noted that the current implementation conforms largely to the early
description in~\cite{Holland92}, but not to those
in~\cite{Holland94,Holland95a}.

The $r$ genes correspond to the agent's {\em uptake mask}, which
determines its ability to collect each resource type directly from the
environment. If an agent does not have a ``1'' allele for the uptake
gene corresponding to a certain resource, it will not be able to
collect that resource directly from the environment.  Consequently, if
the agent requires this resource (because the site at which it is
located charges a tax that includes it, or because the agent needs it
to replicate), it will either have to fight or trade for it. The
designer of an Echo world can create trading webs among agents by
requiring them to trade in various ways to ensure survival.  These
webs, often unstable, may be quickly altered through mutation,
crossover and migration.

The final gene is the {\em trading resource\/} which is the resource
type that the agent will provide to another agent if trading takes
place.  An agent may trade with another agent even if it does not
possess any of its trading resource to exchange. In this case, the
other agent in the interaction receives nothing.  Agents also have a
{\em reservoir\/} in which to store resources.  Resources from the
reservoir are used to pay taxes, to produce offspring, and for trade.

%Agents at a site are arranged in a one-dimensional array.  The
%probability that a pair of agents will be chosen to interact falls off
%exponentially with increasing distance between agents in this array.
%The user must decide which agents initially reside at each site, and
%in what order they should appear in the array.

\subsubsection{Replicating Agents}
\label{self-reproduction}

An agent is replicated when it acquires sufficient resources.  The
agent's genome is copied using the resources it has stored in its
reservoir to construct the daughter agent.  There are two
parameters that control the replication process:

\begin{itemize}

\item The {\em self-reproduction threshold\/} is used to determine
when enough resources have been collected.  As an example, if the
genome contains three {\em a}s and the self-reproduction threshold is
two, then the reservoir must contain at least six {\em a}s before
self-reproduction can take place. A similar requirement must be
satisfied for each resource type.

\item The {\em self-reproduction fraction\/} determines the division
  of excess resources between the original agent and the daughter
  agent after replication.  The fraction indicates how the proportion
  of extra resources (i.e., those not needed to make the clone) should
  be divided between the two.  A setting of one will result in
  offspring with no internal reserve of resources.  At the other
  extreme, a setting of zero allocates all extra resources to the
  offspring, leaving the original vulnerable.
\end{itemize}

During replication, there may be spontaneous mutations; their
frequency is controlled by the {\em mutation probability}.  Mutations
in different parts of the genome are treated differently:
\begin{itemize}
\item If a mutation occurs in the trading-resource gene, the agent's
  trading resource is set to a randomly selected resource (including
  the current resource).

\item Mutation in an uptake mask bit (see Section~\ref{agents}), simply
  flips the bit from one to zero or vice-versa.

\item If a mutation occurs in one of the tags or conditions, it occurs
  in the following way.  If there are $r$ resource types and the gene
  is of length $n$, a number is selected uniformly at random from the
  interval $[1..n+1]$. If this number lies in the interval $[1..n-1]$
  the corresponding gene position is set to a randomly chosen value
  (resource type).  This case is analogous to point mutation in
  conventional genetic algorithms.  If the number is $n+1$, a symbol
  is added to the end of the gene and assigned a random value
  (an insertion).  Finally, if the number is $n$, a number is selected
  uniformly at random from the interval $[1..r+1]$.  If this second
  number is $r+1$, the last symbol is deleted from the gene.
  Otherwise, the last symbol in the gene is set to the corresponding
  resource (a point mutation).
                                % Steph - you put the following
                                % cryptic comment here. What did you
                                % mean?
                                % [TERRY: ??].
  Consequently, gene length can change through additions or deletions
  at the last position in the gene.  These possibilities are
  illustrated in Figure~\ref{fig:agent-evolution}.
\end{itemize}
[EVERYONE: do we get the overall proababilities we want?  Couldn't this
be simplified?]

%\subsubsection{Agent Movement}

%There are two forms of agent movement in Echo: within a single site
%and between sites. Intra-site movement is the result of an agent-agent
%interaction.  In each of these, one agent is first selected.  A second
%agent is then selected in the vicinity of the first. The first agent
%is moved next to the second in the one-dimensional array of agents at
%the site. If the first agent would attack the second, the second may
%run away by moving a small distance away in the array. In both cases,
%distances are likely to be small, as the probability of large
%distances decreases exponentially.

%Inter-site movement occurs if an agent does not acquire any resources
%during an Echo cycle (either through picking them up, combat or
%trade). In this case it will migrate to a neighboring site, selected
%at random from among those permitted by the geography of the world.

%\subsubsection{Agent Death}

%In each cycle, a (typically) very small number of agents are killed
%randomly. This can be interpreted either as bad luck or as a mechanism
%that prevents agents from living indefinitely.  Agents that are not
%killed some other way (through combat or taxation), will eventually be
%randomly killed.  When an agent at a site dies, for whatever reason,
%its resources are returned to the environment, thereby becoming
%available to other agents at that site.

%\section{Resources}

%In every cycle, all the agents at a site compete for all the resources
%their uptake masks allow them to collect. The site produces resources
%according to its production vector, and these are distributed as
%equally as possible amongst the agents at the site that are
%genetically able to collect them.
