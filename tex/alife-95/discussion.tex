\section{Discussion}

Our results show that the original and neutral versions of Echo behave
differently with respect to species richness and population size
in a site, as well as the scaling exponent in the species-area 
relation.  This suggests that the dynamics of the two models differ.
%However, it is not clear which is a better model of ecological and
%evolutionary dynamics.  
Both models produce distributions of abundance which resemble Preston
distributions, though neither produces a canonical distribution.
Both models yield scaling exponents far greater than the expected
value of $\frac{1}{4}$.  We attribute this result to a combination of factors,
discussed below, some of which have interesting implications for the
Echo model.

Our choice of simulation parameters was largely arbitrary.
A different choice of parameter settings could yield different results
than observed here, and some combination of parameters might produce
the expected value.  However, our experience with Echo has shown that
the general patterns reported here appear to be relatively insensitive to
parameter settings.  We have obtained similar results in experiments
with lower mutation rates, increased interaction fractions, and
different initial populations (data not shown).

A second factor affecting our results is the definition of
species in Echo.  In this study, we used the provisional definition
of unique genotypes as species.  A previous study
\cite{ForrestAndJones94} defined species based on 
clustering genomes by a distance metric (number of mutations to
produce one genotype from another).  This clustering technique
produces curves that more closely approximate the canonical Preston
distribution.  Using that method here might reduce the number of
species in a population, and thus lower the slope of the
species-area curve.  We decided not to use the clustering technique
here because of the problem of defining species by an arbitrary
genetic distance, but it could significantly change the outcome of our
experiment by reducing the number of singletons.

Species-area curves are typically constructed by sampling from the
total number of individuals present, rather than a complete census.
We checked to see if sampling might affect our results by constructing
species-area curves using subsampling.  Rather than including each
individual in a sample with a probability of one, we tried
subsampling both $\frac{1}{10}$ and $\frac{1}{4}$ of the population and
recalculating the species-area curves.  This procedure did not
noticably change the slope of the curve (data not shown).  Sampling
and complete censusing of Echo populations produce similar
species-area curves.

In nature, habitat heterogeneity plays a role in limiting the
distributions of species across the landscape.  No species is so well
adapted as to be equally successful in all habitats.  This mechanism
to limit species success does not exist in the Echo worlds we have
studied to date.  Our Echo worlds are not heterogenous but are made
of similar sites tiled together.  In these simulations, there are no
gradients in resource availability or favorability of climate such as
those that are believed to limit organisms in nature.  Rather, a
species which is successful at one site in the world should be equally
successful throughout the world.  [SO, WHY DON'T WE DO SOME RUNS TO
TEST THIS?]

It is likely that the discrepancy between expected and observed values
for $z$ reflects an imbalance between rates of speciation and
extinction in our model.  New species are formed by mutating
existing species, and by recombining genotypes through mating.
Formation of new species is thus controlled by mutation and
recombination rates.  Species go extinct on the death of the last
surviving individual, whether through competition or by chance.
However, the mechanisms of extinction in natural systems may include
exogenous events never before experienced by the species.

Raup proposes that many extinctions could be caused by events that
occur on a time scale greater than species' evolutionary time scales
\cite{Raup91}.  This idea is promising, because it parsimoniously
explains how widespread species might suffer extinction.  If the
exogenous events were to occur on an evolutionary time scale, natural
selection should favor species that can tolerate such adversity.
Species with small populations are more likely to suffer extinction by
mechanisms such as fluctuations in population size or competition than
widespread species.  Again, this would have the effect of decreasing the 
slope of the species-area curve.  For a firmly established (widespread) 
species to suffer extinction requires some extreme circumstances, or rare
events.  Raup calls such unnatural events ``first strikes'' and argues
that these rare events play a major role in driving common species
to extinction.  Unlike earth, the Echo model we used here is a closed
system; there are no meteorite impacts or global changes in climate or
sea level to drive species nearer to extinction through previously
unencountered adversity.

We conclude from these considerations that the mechanisms in the
current implementation of Echo are not sufficient to limit species'
success.  Competition and random death as means for extinction are
insufficient to produce the quantity of extinctions necessary to agree
with empirically based predictions made by the species-area scaling
relation.  However, Echo does exhibit a robust species-area scaling
relation as well as a log-normal species-abundance curve similar to a
Preston curve.  The degree of quantitative agreement between these
curves and those found in natural ecosystems may be less important
than the fact that we see similar kinds of curves.
[PTH disagrees with this very stongly.  The whole point of the canonical
lognormal distribution is that a special kind of curve is observed.
Random processes can produce lognormal distributions, so it's easy to 
make curves which look like these.  The whole reason for using the neutral 
model is to ask whether these curves result from community interactions or 
noise.  PTH cares about degree of quantitative agreement between these
curves and those found in natural ecosystems.]

More generally, these results represent a first step towards
validating Echo as a model of natural ecological systems.  Echo is
clearly quite a different kind of model than those typically used in
ecological studies, and it is relevant to ask what we can hope to
learn from a highly abstract model that by design does not directly
correspond to any real system.  There are at least three possible
answers to such a question:
\begin{itemize}
\item Echo as a patch of dirt,
\item Echo as a flight simulator, and [STEPH: if we take John's name
off the paper we should either cite this or change the language slightly.]
\item Echo as a theory of complex adaptive systems.
\end{itemize}

By ``Echo as a patch of dirt,'' we mean to suggest the possibility of
constructing Echo worlds that do correspond directly to some real
ecosystem.  This would allow careful quantitative validation of Echo's
behavior and might possibly lead to concrete predictions.  This kind
of modeling is well beyond the current state of the Echo research
program.  Although it would be an important achievement for Echo to
simulate one real ecosystem accurately, this may not be its most
important contribution.  Other modeling systems have traditionally
focused on exactly this problem of making quantitative predictions
about the behavior of specific systems with given parameter settings.

A second use of Echo is as a kind of ``flight simulator'' to be used
for building intuitions about how ecosystems work, what is critical
to their stability, etc.  Under this view, Echo itself is a rich enough
ecology to be worth studying in its own right, along the lines that we
have outlined in the previous sections.
We can study patterns of behavior,
e.g., how resources flow through different kinds of ecologies, how
cooperation among agents can arise through evolution, and arms races
\cite{Holland94}.  We can also use such a model to identify 
parameters or collections of parameters that are critical, i.e., to
perform a kind of sensitivity analysis.  As with any simulation tool, 
it is much easier to run hypothetical what-if experiments than to conduct
experiments on a real system.  If a model like Echo were successful
and correct, it would enable users to build deep intuitions about how
different aspects of an ecological system affect one another,
important dependencies, and an appreciation of how evolution interacts
with the internal dynamics of an ecosystem.  The neutral model that we
introduced in this paper is an example of how these intuitions can be
developed and explored.  This is perhaps the most important
contribution that models like Echo can make.
%[MAYBE A MORE CONCRETE STATEMENT HERE ABOUT PETER'S AMERICAN NATURALIST PAPER HERE.]

A third, and more ambitious, view of Echo is as a theory of complex
adaptive systems \cite{Holland95a}.  Echo can be viewed as a
computational artifact that captures what we believe are the essential
components and interactions in a wide variety of CAS.  As such, it
encodes a theory about which mechanisms are relevant and which are
not.  The theory can be confirmed or disconfirmed by running the model
under a wide variety of operating conditions and observing the extent
to which the expected macro-level behaviors arise spontaneously.  Our
experiments on species diversity are an example of such a confirmation
process, and our introduction of the neutral model provides another
way of testing the extent to which the current Echo mechanisms are all
required to produce the behaviors of interest.  It should be noted
that the version of Echo reported here is an early attempt at a broad
CAS theory, and we expect that many of the details will need to be
modified over time (e.g., to address the problem of extinctions) and
when careful comparisons to other CAS (e.g., economic models) are made
Nevertheless, an important contribution of computational systems like
Echo is to distill features that are common across many CAS in a
testable model.

Although we are optimistic about the insights that artificial-life
models like Echo can contribute to understanding real ecosystems and
other CAS, it is important to note that such models have several
apparent drawbacks.  These include a mapping problem, what concrete
questions they can address, scaling issues, and nonlinear interactions
(already discussed).  We discuss the first three of these issues
briefly in the following paragraphs.

Because CAS models tend to strip away many details, it is often
impossible to say what any component of one of these models
corresponds to in the real world---a mapping problem.  In
the immune system, for example, many theoretical immunologists use string
matching to model receptor/ligand binding
\cite{PerelsonAndOster79}.  Patterns of bits (or other symbols) are
used to represent both molecular shape and electrostatic charge.
Consequently, it is difficult to say what one bit in the model
corresponds to in the immune system.  Because different alphabets and
different matching rules can have very different properties, the
challenge is to select an alphabet and matching rule that have general
properties similar to the real system without worrying too much what
each bit really stands for \cite{Smith94a}.  By contrast, most
theories of modeling are based on the premise that a correspondence
can be established between the modeled system and the primitive
components of its model.

As a consequence of this mapping problem, it is not always clear what
scientific questions are being addressed by CAS models.  In more
conventional simulation-based modeling, models are used to make
quantitative predictions based on certain predicated inputs, for
example, to determine optimal parameter values.  Agent-based models of
CAS are rarely able to make this kind of quantitative prediction, and
as a result the focus is on identifying broad categories of behavior
and critical parameters (but not necessarily the exact critical
parameter values), as we discussed earlier.

A third problem faced by agent-based models is one of scale.  Because
they are simulations, agent-based models typically operate on vastly
different time scales of evolution and with much smaller population
sizes than those of the systems they model.  Also, we tend to be
intolerant of high failure rates such as those often observed in
nature.  For example, consider the selection algorithms typically used
in genetic algorithms.  Selection pressure is maintained at an
artificially high rate and often scaled to maintain increased pressure
near the end of a run.  Evolution thus occurs orders of magnitude more
quickly than in natural systems, and as a result, we may lose some of
the richness of the natural evolutionary process.

Despite these drawbacks, we believe that there is much to be learned
from discrete, agent-based models such as Echo.  [REITERATE the 
discrete/spatial inhomogenetity stuff from intro].