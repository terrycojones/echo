\section{The Echo Cycle}
\label{echo-cycle}

%
% STEPH suggests saying
% ``The previous section described the structral elements of Echo...''
% but I have resisted that because it seems a bit of a lazy way
% to get an extra sentence in before the itemized list. I know what
% sort of thing you want though. I'd just like something that looks less
% like filler.
%
In this section, we describe the sequence of events that takes place in
a single Echo cycle:
\begin{enumerate}
\item Interactions between agents are performed at each site.  These
  include trade, mating, and combat.  For each interaction, one agent
  is selected.  A second agent is then selected in the vicinity of the
  first.  The first agent is moved adjacent to the second in the
  one-dimensional array of agents at the site.  If the first agent
  would attack the second, the second may run away by moving a small
  distance away in the array.  In both cases, distances are likely to
  be small, as the probability of large distances decreases
  exponentially.

  [TERRY: this last sentence is unclear.  What are you
  trying to say here?---Please give the forumla]

\item Agents collect resources from the site if any are available.
  The site produces resources according to its ``site'' parameters,
  and these are distributed as equally as possible among the agents at
  the site that are genetically able to collect them.

\item Each agent at each site is taxed (probabilistically).  Each site
  exacts a resource tax from each agent with a given (worldwide)
  probability.  If an agent does not possess the resources to pay the
  tax, it is deleted (killed) and its resources are returned to the
  environment.  Tax in Echo can be thought of as economic taxation, or
  as the cost required to live at the site.  Biologically, this can be
  thought of as metabolic cost.

\item Agents are killed at random with some small probability.  This
  can be interpreted either as bad luck or as a mechanism that
  prevents agents from living indefinitely.  Agents that are not
  killed some other way (through combat or taxation), will eventually
  be randomly killed.  When an agent at a site dies, for whatever
  reason, its resources are returned to the environment, thereby
  becoming available to other agents at that site.

\item The sites produce resources. Different sites may produce
  different amounts of each resource.  For example, one site might
  produce ten {\em a}'s and ten {\em b}'s on each time step, whereas
  another might produce five {\em b}'s and twenty {\em c}'s.  The
  idea is that agents will replicate frequently if they are located
  at sites whose resources match their genomes, if the site is not too
  crowded.  When an agent at a site dies, its resources are returned
  to the environment and become immediately available to other agents
  at that site.

\item Agents that do not acquire any resources during an Echo cycle
  (either through picking them up or through combat or trade) migrate
  to a neighboring site. The neighboring site is selected at random
  from among those permitted by the geography of the world.  This is
  not the same as the local movement within a site that occurs as the
  result of the agent-agent interactions that are described in
  Section~\ref{agent-agent}.

\item Agents that can replicate do so (asexual reproduction).  An
  agent may replicate when it acquires sufficient resources.  In
  replication, an agent makes a copy of its genome using the resources
  it has stored in its reservoir.  A parameter controls how many
  resources are required to be stored beyond those needed to make an
  exact copy.  The replication process is noisy: random mutations may
  result in genetic differences between parent and child.

\end{enumerate}
This cycle is iterated many times during the course of a run.


