%
% SLIDE 1
%
\begin{slide}{}
\centerline{\bf Ecological Modeling}
\centerline{\bf with Echo}
\vskip 1.0truein
\centerline{Stephanie Forrest}
\centerline{Dept. of Computer Science}
\centerline{University of New Mexico}
\centerline{Albuquerque, New Mexico}
\vskip1.0truein
\centerline{Terry Jones}
\centerline{Santa Fe Institute}
\centerline{Santa Fe, New Mexico}
\end{slide}

%
% SLIDE 2
%
\begin{slide}{}
\centerline{\bf Project Plan}
\begin{itemize}
\itemsep 0pt
\parsep  0pt
\parskip 0pt
\item Learn what Echo is.
\vskip 6pt
\item Implement Echo in C with X-based graphical interface.
\vskip 6pt
\item Produce Echo documentation.
\vskip 6pt
\item Produce reasonably portable Echo distribution.
\vskip 6pt
\item Overview paper.
\vskip 6pt
\item Start model validation.
\begin{itemize}
\itemsep 0pt
\parsep  0pt
\parskip 0pt
\item Jim Brown
\vskip 6pt
\item Leo Buss
\vskip 6pt
\item Jonathon Haas
\vskip 6pt
\item Tim Keitt
\end{itemize}
\end{itemize}
\end{slide}

%
% SLIDE 3
%
\begin{slide}{}
\centerline{\bf Echo}
An abstract model of ecological behavior:
\begin{itemize}
\itemsep 0pt
\parsep  0pt
\parskip 0pt
\item Large numbers of interacting agents.
\vskip 6pt
\item Interactions among agents:
\vskip 6pt
\begin{itemize}
\item Trade
\item Combat
\item Mating
\end{itemize}
\item Agents are distributed over a geography of sites.
\vskip 6pt
\item Each site has renewable resources.
\vskip 6pt
\item Agents can migrate between sites.
\end{itemize}
%Relevant macro-properties for Echo:
%\begin{itemize}
%\item Formation of species
%\item Efficiency
%\item Flexibility
%\end{itemize}
\end{slide}

%
% SLIDE 4
%
\begin{slide}{}
\centerline{\bf Echo}
What can we hope to learn with highly abstract idealized models
like Echo?
\begin{itemize}
\item General patterns of behavior.
\item E.g., flow of resources, cooperation among agents, arms races, 
communities.
\item Critical regions of parameter space.
\item E.g., Sustainability, economies.
\end{itemize}
\end{slide}

%
% SLIDE 5
%
\begin{slide}{}
\centerline{\bf What is Echo?}
\begin{enumerate}
\vskip 15pt
\itemsep 0pt
\parsep  0pt
\parskip 0pt

\item Worlds:
\vskip 6pt
\begin{itemize}
\item System-wide parameters.
\end{itemize}
\item Sites:
\vskip 6pt
\begin{itemize}
\item Locality within a site.
\vskip 2pt
\item Renewable resources (A's, B's, C's, etc.).
\vskip 2pt
\item Migration between sites.
\vskip 4pt
\end{itemize}
\item Agents:
\vskip 6pt
\begin{itemize}
\item Defined entirely by genotypes.
\vskip 2pt
\item Genotypes specify interactions with other agents.
\vskip 2pt
\item Reservoirs of resources (letters).
\vskip 2pt
\item Agents reproduce when they have enough resources to make a complete
copy of their genotype.
\vskip 2pt
\item Agents die randomly.
\vskip 2pt
\item Agents acquire resource by fighting and from their local site.
\vskip 2pt
\end{itemize}
\end{enumerate}
\end{slide}

%
% SLIDE 6
%
\begin{slide}{}
\centerline{\bf The Ecology of Echo}
\begin{itemize}
\vskip 15pt
\itemsep 0pt
\parsep  0pt
\parskip 0pt
\item Stable interactions among variants (fly-ant-catepillar).
\vskip 6pt
\item Flows of resources.
\vskip 6pt
\item Trajectories of number of variants.
\vskip 6pt
\item Duration of variants over time.
\vskip 6pt
\item Cataclysmic events (e.g., meteors).
\vskip 6pt
\item Speciation.
\vskip 6pt
\item Evolution of phenotypic plasticity.
\vskip 6pt
\item Isolation effects.
\vskip 6pt
\item What happens when more resources are added to the system?
\end{itemize}
\end{slide}

\begin{note}
\end{note}

%\begin{description,leftmargin 0.5in, indent -0.5in)

%
% SLIDE 7
%
\begin{slide}{}
\centerline{\bf ``Tropical Rain Forest'' Experiment}
Hypothesis: Diversity of genotypes will increase with resources.

Experiment: Measure diversity in Echo as a function of resources:
\begin{itemize}
\itemsep 0pt
\parsep  0pt
\parskip 0pt
\item One site.
\vskip 6pt
\item Initial resource in env. is 50 (of each letter).
\vskip 6pt
\item Each agent starts with 11 genes (but can grow or shrink).
\vskip 6pt
\item Initialize with 30 agents of 3 different types (10 of each type).
\vskip 6pt
\item No mating (mutation only - 0.0005 per gene)
\vskip 6pt
\item Random death - 0.0001 per indiv. per gen.
\end{itemize}

\end{slide}


%
% SLIDE 8
%
\begin{slide}{}
\centerline{\bf Observations}
\begin{itemize}
\item Experiment apparently confirms the hypothesis.
\item Linear (nearly) relation between resources and diversity.
\item Population continues to be invadable by new agent types.
\item Frequency of different agent types has high variance.
\item In the real world most species that have ever existed no longer exist.
Echo behaves similarly.
\end{itemize}

\end{slide}


%
% SLIDE 9
%
\begin{slide}{}
\centerline{\bf The Echo Cycle}
\begin{itemize}
\vskip 15pt
\itemsep 0pt
\parsep  0pt
\parskip 0pt
\item Pairs of agents are selected for interaction (combat, trading, mating).
\vskip 6pt
\item Agents uptake resources.
\vskip 6pt
\item Sites charge maintenance.
\vskip 6pt
\item Agents self--replicate.
\vskip 6pt
\item Agents migrate if they have acquired nothing recently.
\vskip 6pt
\item Sites produce resources.
\end{itemize}
\end{slide}

%
% SLIDE 10
%
\begin{slide}{}
\centerline{\bf Why Build Models?}
\begin{itemize}
\vskip 15pt
\itemsep 0pt
\parsep  0pt
\parskip 0pt
\item Something in the real world is of interest.
\vskip 6pt
\item A model aims to:
\begin{itemize}
\item Explain the past.
\vskip 3pt
\item Predict the future.
\end{itemize}
\vskip 6pt
\item A model is always a simplification.
\end{itemize}
\end{slide}

%
% SLIDE 11
%
\begin{slide}{}
\centerline{\bf Issues In Model Building}
\begin{itemize}
\vskip 15pt
\itemsep 0pt
\parsep  0pt
\parskip 0pt
\item Predictive ability of the model.
\vskip 6pt
\item Explanatory power of the model.
\vskip 6pt
\item What to discard, what to keep?
\vskip 6pt
\item Ease of construction.
\vskip 6pt
\item Confidence in programming.
\vskip 6pt
\item Ease of experimentation.
\end{itemize}
\end{slide}

%
% SLIDE 12
%
\begin{slide}{}
\centerline{\bf Pessimism}
\begin{itemize}
\vskip 15pt
\itemsep 0pt
\parsep  0pt
\parskip 0pt
\item Explanation is hard.
\vskip 6pt
\item Prediction is hard.
\vskip 6pt
\item Verification is hard.
\vskip 6pt
\item Good hypotheses are many and almost always wrong.
\end{itemize}
\end{slide}

%
% SLIDE 13
%
\begin{slide}{}
\centerline{\bf Optimism}
\begin{itemize}
\vskip 15pt
\itemsep 0pt
\parsep  0pt
\parskip 0pt
\item Computer allows explanation (eventually).
\vskip 6pt
\item Prediction can be probabilistic.
\vskip 6pt
\item Sensitivities and stabilities can be identified.
\vskip 6pt
\item Models can be kites to be shot down.
\vskip 6pt
\item Expectations can be changed.
\vskip 6pt
\item One can always learn to be humble.
\end{itemize}
\end{slide}

%
% SLIDE 14
%
\begin{slide}{}
\centerline{\bf Outline}
\begin{itemize}
\vskip 15pt
\itemsep 0pt
\parsep  0pt
\parskip 0pt
\item Introduction to Echo.
\vskip 6pt
\item Issues in model building.
\vskip 6pt
\item Echo demonstration.
\end{itemize}
\end{slide}

%
% SLIDE 15
%
\begin{slide}{}
\centerline{\bf Outline}
\begin{itemize}
\vskip 15pt
\itemsep 0pt
\parsep  0pt
\parskip 0pt
\item Echo project plan and progress report.
\begin{itemize}
\item Development.
\vskip 6pt
\item Verification.
\vskip 6pt
\item Outlook.
\end{itemize}
\vskip 6pt
\item Computational Landscapes
\begin{itemize}
\item Search algorithms, navigation and landscapes.
\vskip 6pt
\item Genetic algorithms and crossover landscapes.
\vskip 6pt
\end{itemize}
\end{itemize}
\end{slide}

%
% SLIDE 16
%
\begin{slide}{}
\centerline{\bf What can we hope to learn with}
\centerline{\bf abstract models like Echo?}
\begin{itemize}
\item Echo as a patch of dirt

\item Echo as a flight simulator
\begin{itemize}
\item Building intuitions
\item The ecology of Echo
\item Neutral models
\item Species abundance distributions
\end{itemize}

\item A theory of complex adaptive systems

\end{itemize}
\end{slide}
